\PassOptionsToPackage{unicode=true}{hyperref} % options for packages loaded elsewhere
\PassOptionsToPackage{hyphens}{url}
%
\documentclass[]{article}
\usepackage{lmodern}
\usepackage{amssymb,amsmath}
\usepackage{ifxetex,ifluatex}
\usepackage{fixltx2e} % provides \textsubscript
\ifnum 0\ifxetex 1\fi\ifluatex 1\fi=0 % if pdftex
  \usepackage[T1]{fontenc}
  \usepackage[utf8]{inputenc}
  \usepackage{textcomp} % provides euro and other symbols
\else % if luatex or xelatex
  \usepackage{unicode-math}
  \defaultfontfeatures{Ligatures=TeX,Scale=MatchLowercase}
\fi
% use upquote if available, for straight quotes in verbatim environments
\IfFileExists{upquote.sty}{\usepackage{upquote}}{}
% use microtype if available
\IfFileExists{microtype.sty}{%
\usepackage[]{microtype}
\UseMicrotypeSet[protrusion]{basicmath} % disable protrusion for tt fonts
}{}
\IfFileExists{parskip.sty}{%
\usepackage{parskip}
}{% else
\setlength{\parindent}{0pt}
\setlength{\parskip}{6pt plus 2pt minus 1pt}
}
\usepackage{hyperref}
\hypersetup{
            pdftitle={Source code for Using citizen science to parse climatic and landcover influences on bird occupancy within a tropical biodiversity hotspot},
            pdfauthor={Vijay Ramesh; Pratik R. Gupte; Morgan W. Tingley; VV Robin; Ruth DeFries},
            pdfborder={0 0 0},
            breaklinks=true}
\urlstyle{same}  % don't use monospace font for urls
\usepackage[margin=1in]{geometry}
\usepackage{color}
\usepackage{fancyvrb}
\newcommand{\VerbBar}{|}
\newcommand{\VERB}{\Verb[commandchars=\\\{\}]}
\DefineVerbatimEnvironment{Highlighting}{Verbatim}{commandchars=\\\{\}}
% Add ',fontsize=\small' for more characters per line
\newenvironment{Shaded}{}{}
\newcommand{\AlertTok}[1]{\textcolor[rgb]{1.00,0.00,0.00}{#1}}
\newcommand{\AnnotationTok}[1]{\textcolor[rgb]{0.00,0.50,0.00}{#1}}
\newcommand{\AttributeTok}[1]{#1}
\newcommand{\BaseNTok}[1]{#1}
\newcommand{\BuiltInTok}[1]{#1}
\newcommand{\CharTok}[1]{\textcolor[rgb]{0.00,0.50,0.50}{#1}}
\newcommand{\CommentTok}[1]{\textcolor[rgb]{0.00,0.50,0.00}{#1}}
\newcommand{\CommentVarTok}[1]{\textcolor[rgb]{0.00,0.50,0.00}{#1}}
\newcommand{\ConstantTok}[1]{#1}
\newcommand{\ControlFlowTok}[1]{\textcolor[rgb]{0.00,0.00,1.00}{#1}}
\newcommand{\DataTypeTok}[1]{#1}
\newcommand{\DecValTok}[1]{#1}
\newcommand{\DocumentationTok}[1]{\textcolor[rgb]{0.00,0.50,0.00}{#1}}
\newcommand{\ErrorTok}[1]{\textcolor[rgb]{1.00,0.00,0.00}{\textbf{#1}}}
\newcommand{\ExtensionTok}[1]{#1}
\newcommand{\FloatTok}[1]{#1}
\newcommand{\FunctionTok}[1]{#1}
\newcommand{\ImportTok}[1]{#1}
\newcommand{\InformationTok}[1]{\textcolor[rgb]{0.00,0.50,0.00}{#1}}
\newcommand{\KeywordTok}[1]{\textcolor[rgb]{0.00,0.00,1.00}{#1}}
\newcommand{\NormalTok}[1]{#1}
\newcommand{\OperatorTok}[1]{#1}
\newcommand{\OtherTok}[1]{\textcolor[rgb]{1.00,0.25,0.00}{#1}}
\newcommand{\PreprocessorTok}[1]{\textcolor[rgb]{1.00,0.25,0.00}{#1}}
\newcommand{\RegionMarkerTok}[1]{#1}
\newcommand{\SpecialCharTok}[1]{\textcolor[rgb]{0.00,0.50,0.50}{#1}}
\newcommand{\SpecialStringTok}[1]{\textcolor[rgb]{0.00,0.50,0.50}{#1}}
\newcommand{\StringTok}[1]{\textcolor[rgb]{0.00,0.50,0.50}{#1}}
\newcommand{\VariableTok}[1]{#1}
\newcommand{\VerbatimStringTok}[1]{\textcolor[rgb]{0.00,0.50,0.50}{#1}}
\newcommand{\WarningTok}[1]{\textcolor[rgb]{0.00,0.50,0.00}{\textbf{#1}}}
\usepackage{longtable,booktabs}
% Fix footnotes in tables (requires footnote package)
\IfFileExists{footnote.sty}{\usepackage{footnote}\makesavenoteenv{longtable}}{}
\usepackage{graphicx,grffile}
\makeatletter
\def\maxwidth{\ifdim\Gin@nat@width>\linewidth\linewidth\else\Gin@nat@width\fi}
\def\maxheight{\ifdim\Gin@nat@height>\textheight\textheight\else\Gin@nat@height\fi}
\makeatother
% Scale images if necessary, so that they will not overflow the page
% margins by default, and it is still possible to overwrite the defaults
% using explicit options in \includegraphics[width, height, ...]{}
\setkeys{Gin}{width=\maxwidth,height=\maxheight,keepaspectratio}
\setlength{\emergencystretch}{3em}  % prevent overfull lines
\providecommand{\tightlist}{%
  \setlength{\itemsep}{0pt}\setlength{\parskip}{0pt}}
\setcounter{secnumdepth}{5}
% Redefines (sub)paragraphs to behave more like sections
\ifx\paragraph\undefined\else
\let\oldparagraph\paragraph
\renewcommand{\paragraph}[1]{\oldparagraph{#1}\mbox{}}
\fi
\ifx\subparagraph\undefined\else
\let\oldsubparagraph\subparagraph
\renewcommand{\subparagraph}[1]{\oldsubparagraph{#1}\mbox{}}
\fi

% set default figure placement to htbp
\makeatletter
\def\fps@figure{htbp}
\makeatother


\usepackage{fontspec}
% use nice fonts if available else use boring defaults

\usepackage{lineno}
% \KOMAoption{fontsize}{10pt}

\IfFontExistsTF{Times New Roman}{\setmainfont[]{Times New Roman}}{} 
% \IfFontExistsTF{Arial}{\setsansfont[]{Arial}}{}
\IfFontExistsTF{Inconsolata}{\setmonofont{Inconsolata}}

\linenumbers

\title{Source code for \emph{Using citizen science to parse climatic and landcover influences on bird occupancy within a tropical biodiversity hotspot}}
\author{Vijay Ramesh \and Pratik R. Gupte \and Morgan W. Tingley \and VV Robin \and Ruth DeFries}
\date{2020-12-26}

\begin{document}
\maketitle

{
\setcounter{tocdepth}{2}
\tableofcontents
}
\hypertarget{introduction}{%
\section{Introduction}\label{introduction}}

This is the readable version containing analysis that models associations between environmental predictors (climate and landcover) and citizen science observations of birds across the Nilgiri and Anamalai Hills of the Western Ghats Biodiversity Hotspot.

Methods and format are derived from \href{https://cornelllabofornithology.github.io/ebird-best-practices/}{Strimas-Mackey et al.}.

\hypertarget{attribution}{%
\subsection{Attribution}\label{attribution}}

Please contact the following in case of interest in the project.

\begin{itemize}
\tightlist
\item
  Vijay Ramesh (lead author)

  \begin{itemize}
  \tightlist
  \item
    PhD student, Columbia University
  \end{itemize}
\item
  Pratik Gupte (repo maintainer)

  \begin{itemize}
  \tightlist
  \item
    PhD student, University of Groningen
  \end{itemize}
\end{itemize}

\hypertarget{data-access}{%
\subsection{Data access}\label{data-access}}

The data used in this work are available from \href{http://ebird.org/data/download}{eBird}.

\hypertarget{data-processing}{%
\subsection{Data processing}\label{data-processing}}

The data processing for this project is described in the following sections. Navigate through them using the links in the sidebar.

\hypertarget{main-text-figure-1}{%
\subsection{Main Text Figure 1}\label{main-text-figure-1}}

Figure prepared in QGIS 3.10.

\begin{figure}
\centering
\includegraphics{figs/fig_01_elevation.png}
\caption{A shaded relief of the study area - the Nilgiri and the Anamalai hills are shown in this figure. This map was made using the SRTM digital elevation model at a spatial resolution of 1km and data from Natural Earth were used to outline boundaries of water bodies.}
\end{figure}

\hypertarget{preparing-ebird-data}{%
\section{Preparing eBird Data}\label{preparing-ebird-data}}

\hypertarget{prepare-libraries-and-data-sources}{%
\subsection{Prepare libraries and data sources}\label{prepare-libraries-and-data-sources}}

Here, we will load the necessary libraries required for preparing the eBird data. Please download the latest versions of the eBird Basic Dataset (for India) and the eBird Sampling dataset from \url{https://ebird.org/data/download}.

\begin{Shaded}
\begin{Highlighting}[numbers=left,,]
\CommentTok{# load libraries}
\KeywordTok{library}\NormalTok{(tidyverse)}
\KeywordTok{library}\NormalTok{(readr)}
\KeywordTok{library}\NormalTok{(sf)}
\KeywordTok{library}\NormalTok{(auk)}
\KeywordTok{library}\NormalTok{(readxl)}

\CommentTok{# custom sum function}
\NormalTok{sum.no.na <-}\StringTok{ }\ControlFlowTok{function}\NormalTok{(x) \{}
  \KeywordTok{sum}\NormalTok{(x, }\DataTypeTok{na.rm =}\NormalTok{ T)}
\NormalTok{\}}

\CommentTok{# set file paths for auk functions}
\CommentTok{# To use these two datasets, please download the latest versions from https://ebird.org/data/download and set the file path accordingly. Since these two datasets are extremely large, we have not uploaded the same on github.}

\NormalTok{f_in_ebd <-}\StringTok{ }\KeywordTok{file.path}\NormalTok{(}\StringTok{"data/ebd_IN_relApr-2020.txt"}\NormalTok{)}
\NormalTok{f_in_sampling <-}\StringTok{ }\KeywordTok{file.path}\NormalTok{(}\StringTok{"data/ebd_sampling_relApr-2020.txt"}\NormalTok{)}
\end{Highlighting}
\end{Shaded}

\hypertarget{filter-data}{%
\subsection{Filter data}\label{filter-data}}

Insert the list of species that we will be analyzing in this study. We initially chose those species that occurred in at least 5\% of all checklists across 50\% of the 25 x 25 km cells from where they have been reported, resulting in a total of 93 species. To arrive at this final list of species, we carried out further pre-processing which can be found in Section 2 of the Supplementary material.

\begin{Shaded}
\begin{Highlighting}[numbers=left,,]
\CommentTok{# add species of interest}
\NormalTok{specieslist <-}\StringTok{ }\KeywordTok{read.csv}\NormalTok{(}\StringTok{"data/species_list.csv"}\NormalTok{)}
\NormalTok{speciesOfInterest <-}\StringTok{ }\KeywordTok{as.character}\NormalTok{(specieslist}\OperatorTok{$}\NormalTok{scientific_name)}
\end{Highlighting}
\end{Shaded}

Here, we set broad spatial filters for the states of Kerala, Tamil Nadu and Karnataka and keep only those checklists for our list of species that were reported between 1st Jan 2013 and 31st Dec 2019.

\begin{Shaded}
\begin{Highlighting}[numbers=left,,]
\CommentTok{# run filters using auk packages}
\NormalTok{ebd_filters <-}\StringTok{ }\KeywordTok{auk_ebd}\NormalTok{(f_in_ebd, f_in_sampling) }\OperatorTok
\StringTok{  }\KeywordTok{auk_species}\NormalTok{(speciesOfInterest) }\OperatorTok
\StringTok{  }\KeywordTok{auk_country}\NormalTok{(}\DataTypeTok{country =} \StringTok{"IN"}\NormalTok{) }\OperatorTok
\StringTok{  }\KeywordTok{auk_state}\NormalTok{(}\KeywordTok{c}\NormalTok{(}\StringTok{"IN-KL"}\NormalTok{, }\StringTok{"IN-TN"}\NormalTok{, }\StringTok{"IN-KA"}\NormalTok{)) }\OperatorTok\StringTok{ }\CommentTok{# Restricting geography to TamilNadu, Kerala & Karnataka}
\StringTok{  }\KeywordTok{auk_date}\NormalTok{(}\KeywordTok{c}\NormalTok{(}\StringTok{"2013-01-01"}\NormalTok{, }\StringTok{"2019-12-31"}\NormalTok{)) }\OperatorTok
\StringTok{  }\KeywordTok{auk_complete}\NormalTok{()}

\CommentTok{# check filters}
\NormalTok{ebd_filters}
\end{Highlighting}
\end{Shaded}

Below code need not be run if it has been filtered once already and the above path leads to the right dataset. NB: This is a computation heavy process, run with caution.

\begin{Shaded}
\begin{Highlighting}[numbers=left,,]
\CommentTok{# specify output location and perform filter}
\NormalTok{f_out_ebd <-}\StringTok{ "data/01_ebird-filtered-EBD-westernGhats.txt"}
\NormalTok{f_out_sampling <-}\StringTok{ "data/01_ebird-filtered-sampling-westernGhats.txt"}
\end{Highlighting}
\end{Shaded}

\begin{Shaded}
\begin{Highlighting}[numbers=left,,]
\NormalTok{ebd_filtered <-}\StringTok{ }\KeywordTok{auk_filter}\NormalTok{(ebd_filters,}
  \DataTypeTok{file =}\NormalTok{ f_out_ebd,}
  \DataTypeTok{file_sampling =}\NormalTok{ f_out_sampling, }\DataTypeTok{overwrite =} \OtherTok{TRUE}
\NormalTok{)}
\end{Highlighting}
\end{Shaded}

\hypertarget{process-filtered-data}{%
\subsection{Process filtered data}\label{process-filtered-data}}

The data has been filtered above using the auk functions. We will now work with the filtered checklist observations (Please note that we have not yet spatially filtered the checklists to the confines of our study area, which is the Nilgiris and the Anamalai hills. This step is carried out further on).

\begin{Shaded}
\begin{Highlighting}[numbers=left,,]
\CommentTok{# read in the data}
\NormalTok{ebd <-}\StringTok{ }\KeywordTok{read_ebd}\NormalTok{(f_out_ebd)}
\end{Highlighting}
\end{Shaded}

eBird checklists only suggest whether a species was reported at a particular location. To arrive at absence data, we use a process known as zero-filling (Johnston et al. \protect\hyperlink{ref-johnston2019a}{2019}), wherein a new dataframe is created with a 0 marked for each checklist when the bird was not observed.

\begin{Shaded}
\begin{Highlighting}[numbers=left,,]
\CommentTok{# fill zeroes}
\NormalTok{zf <-}\StringTok{ }\KeywordTok{auk_zerofill}\NormalTok{(f_out_ebd, f_out_sampling)}
\NormalTok{new_zf <-}\StringTok{ }\KeywordTok{collapse_zerofill}\NormalTok{(zf)}
\end{Highlighting}
\end{Shaded}

Let us now choose specific columns necessary for further analysis.

\begin{Shaded}
\begin{Highlighting}[numbers=left,,]
\CommentTok{# choose columns of interest}
\NormalTok{columnsOfInterest <-}\StringTok{ }\KeywordTok{c}\NormalTok{(}
  \StringTok{"checklist_id"}\NormalTok{, }\StringTok{"scientific_name"}\NormalTok{, }\StringTok{"common_name"}\NormalTok{,}
  \StringTok{"observation_count"}\NormalTok{, }\StringTok{"locality"}\NormalTok{, }\StringTok{"locality_id"}\NormalTok{,}
  \StringTok{"locality_type"}\NormalTok{, }\StringTok{"latitude"}\NormalTok{, }\StringTok{"longitude"}\NormalTok{,}
  \StringTok{"observation_date"}\NormalTok{, }\StringTok{"time_observations_started"}\NormalTok{,}
  \StringTok{"observer_id"}\NormalTok{, }\StringTok{"sampling_event_identifier"}\NormalTok{,}
  \StringTok{"protocol_type"}\NormalTok{, }\StringTok{"duration_minutes"}\NormalTok{,}
  \StringTok{"effort_distance_km"}\NormalTok{, }\StringTok{"effort_area_ha"}\NormalTok{,}
  \StringTok{"number_observers"}\NormalTok{, }\StringTok{"species_observed"}\NormalTok{,}
  \StringTok{"reviewed"}
\NormalTok{)}

\CommentTok{# make list of presence and absence data and choose cols of interest}
\NormalTok{data <-}\StringTok{ }\KeywordTok{list}\NormalTok{(ebd, new_zf) }\OperatorTok
\StringTok{  }\KeywordTok{map}\NormalTok{(}\ControlFlowTok{function}\NormalTok{(x) \{}
\NormalTok{    x }\OperatorTok\StringTok{ }\KeywordTok{select}\NormalTok{(}\KeywordTok{one_of}\NormalTok{(columnsOfInterest))}
\NormalTok{  \})}

\CommentTok{# remove zerofills to save working memory}
\KeywordTok{rm}\NormalTok{(zf, new_zf)}
\KeywordTok{gc}\NormalTok{()}

\CommentTok{# check for presences and absence in absences df, remove essentially the presences df which may lead to erroneous analysis}
\NormalTok{data[[}\DecValTok{2}\NormalTok{]] <-}\StringTok{ }\NormalTok{data[[}\DecValTok{2}\NormalTok{]] }\OperatorTok\StringTok{ }\KeywordTok{filter}\NormalTok{(species_observed }\OperatorTok{==}\StringTok{ }\NormalTok{F)}
\end{Highlighting}
\end{Shaded}

\hypertarget{spatial-filter}{%
\subsection{Spatial filter}\label{spatial-filter}}

A spatial filter is now supplied to further restrict our list of observations to the confines of the Nilgiris and the Anamalai hills of the Western Ghats biodiversity hotspot.

\begin{Shaded}
\begin{Highlighting}[numbers=left,,]
\CommentTok{# load shapefile of the study area}
\KeywordTok{library}\NormalTok{(sf)}
\NormalTok{hills <-}\StringTok{ }\KeywordTok{st_read}\NormalTok{(}\StringTok{"data/spatial/hillsShapefile/Nil_Ana_Pal.shp"}\NormalTok{)}

\CommentTok{# write a prelim filter by bounding box}
\NormalTok{box <-}\StringTok{ }\KeywordTok{st_bbox}\NormalTok{(hills)}

\CommentTok{# get data spatial coordinates}
\NormalTok{dataLocs <-}\StringTok{ }\NormalTok{data }\OperatorTok
\StringTok{  }\KeywordTok{map}\NormalTok{(}\ControlFlowTok{function}\NormalTok{(x) \{}
    \KeywordTok{select}\NormalTok{(x, longitude, latitude) }\OperatorTok
\StringTok{      }\KeywordTok{filter}\NormalTok{(}\KeywordTok{between}\NormalTok{(longitude, box[}\StringTok{"xmin"}\NormalTok{], box[}\StringTok{"xmax"}\NormalTok{]) }\OperatorTok{&}
\StringTok{        }\KeywordTok{between}\NormalTok{(latitude, box[}\StringTok{"ymin"}\NormalTok{], box[}\StringTok{"ymax"}\NormalTok{]))}
\NormalTok{  \}) }\OperatorTok
\StringTok{  }\KeywordTok{bind_rows}\NormalTok{() }\OperatorTok
\StringTok{  }\KeywordTok{distinct}\NormalTok{() }\OperatorTok
\StringTok{  }\KeywordTok{st_as_sf}\NormalTok{(}\DataTypeTok{coords =} \KeywordTok{c}\NormalTok{(}\StringTok{"longitude"}\NormalTok{, }\StringTok{"latitude"}\NormalTok{)) }\OperatorTok
\StringTok{  }\KeywordTok{st_set_crs}\NormalTok{(}\DecValTok{4326}\NormalTok{) }\OperatorTok
\StringTok{  }\KeywordTok{st_intersection}\NormalTok{(hills)}

\CommentTok{# get simplified data and drop geometry}
\NormalTok{dataLocs <-}\StringTok{ }\KeywordTok{mutate}\NormalTok{(dataLocs, }\DataTypeTok{spatialKeep =}\NormalTok{ T) }\OperatorTok
\StringTok{  }\KeywordTok{bind_cols}\NormalTok{(., }\KeywordTok{as_tibble}\NormalTok{(}\KeywordTok{st_coordinates}\NormalTok{(dataLocs))) }\OperatorTok
\StringTok{  }\KeywordTok{st_drop_geometry}\NormalTok{()}

\CommentTok{# bind to data and then filter}
\NormalTok{data <-}\StringTok{ }\NormalTok{data }\OperatorTok
\StringTok{  }\KeywordTok{map}\NormalTok{(}\ControlFlowTok{function}\NormalTok{(x) \{}
    \KeywordTok{left_join}\NormalTok{(x, dataLocs, }\DataTypeTok{by =} \KeywordTok{c}\NormalTok{(}\StringTok{"longitude"}\NormalTok{ =}\StringTok{ "X"}\NormalTok{, }\StringTok{"latitude"}\NormalTok{ =}\StringTok{ "Y"}\NormalTok{)) }\OperatorTok
\StringTok{      }\KeywordTok{filter}\NormalTok{(spatialKeep }\OperatorTok{==}\StringTok{ }\NormalTok{T) }\OperatorTok
\StringTok{      }\KeywordTok{select}\NormalTok{(}\OperatorTok{-}\NormalTok{Id, }\OperatorTok{-}\NormalTok{spatialKeep)}
\NormalTok{  \})}
\end{Highlighting}
\end{Shaded}

Save temporary data created so far.

\begin{Shaded}
\begin{Highlighting}[numbers=left,,]
\CommentTok{# save a temp data file}
\KeywordTok{save}\NormalTok{(data, }\DataTypeTok{file =} \StringTok{"data/01_data_temp.rdata"}\NormalTok{)}
\end{Highlighting}
\end{Shaded}

\hypertarget{handle-presence-data}{%
\subsection{Handle presence data}\label{handle-presence-data}}

Further pre-processing is required in the case of many checklists where species abundance is often unknown and an `X' is denoted in such cases. Here, we convert all `X' notations to a 1, suggesting a presence (as we are not concerned with abundance data in this analysis). We also removed those checklists where the duration in minutes is either not recorded or listed as zero. Lastly, we added an sampling effort based filter following (Johnston et al. \protect\hyperlink{ref-johnston2019a}{2019}), wherein we considered only those checklists with duration in minutes is less than 300 and distance in kilometres traveled is less than 5km. Lastly, we excluded those group checklists where the number of observers was greater than 10.

\begin{Shaded}
\begin{Highlighting}[numbers=left,,]
\CommentTok{# in the first set, replace X, for presences, with 1}
\NormalTok{data[[}\DecValTok{1}\NormalTok{]] <-}\StringTok{ }\NormalTok{data[[}\DecValTok{1}\NormalTok{]] }\OperatorTok
\StringTok{  }\KeywordTok{mutate}\NormalTok{(}\DataTypeTok{observation_count =} \KeywordTok{ifelse}\NormalTok{(observation_count }\OperatorTok{==}\StringTok{ "X"}\NormalTok{,}
    \StringTok{"1"}\NormalTok{, observation_count}
\NormalTok{  ))}

\CommentTok{# remove records where duration is 0}
\NormalTok{data <-}\StringTok{ }\KeywordTok{map}\NormalTok{(data, }\ControlFlowTok{function}\NormalTok{(x) }\KeywordTok{filter}\NormalTok{(x, duration_minutes }\OperatorTok{>}\StringTok{ }\DecValTok{0}\NormalTok{))}

\CommentTok{# group data by site and sampling event identifier}
\CommentTok{# then, summarise relevant variables as the sum}
\NormalTok{dataGrouped <-}\StringTok{ }\KeywordTok{map}\NormalTok{(data, }\ControlFlowTok{function}\NormalTok{(x) \{}
\NormalTok{  x }\OperatorTok
\StringTok{    }\KeywordTok{group_by}\NormalTok{(sampling_event_identifier) }\OperatorTok
\StringTok{    }\KeywordTok{summarise_at}\NormalTok{(}
      \KeywordTok{vars}\NormalTok{(}
\NormalTok{        duration_minutes, effort_distance_km,}
\NormalTok{        effort_area_ha}
\NormalTok{      ),}
      \KeywordTok{list}\NormalTok{(sum.no.na)}
\NormalTok{    )}
\NormalTok{\})}

\CommentTok{# bind rows combining data frames, and filter}
\NormalTok{dataGrouped <-}\StringTok{ }\KeywordTok{bind_rows}\NormalTok{(dataGrouped) }\OperatorTok
\StringTok{  }\KeywordTok{filter}\NormalTok{(}
\NormalTok{    duration_minutes }\OperatorTok{<=}\StringTok{ }\DecValTok{300}\NormalTok{,}
\NormalTok{    effort_distance_km }\OperatorTok{<=}\StringTok{ }\DecValTok{5}\NormalTok{,}
\NormalTok{    effort_area_ha }\OperatorTok{<=}\StringTok{ }\DecValTok{500}
\NormalTok{  )}

\CommentTok{# get data identifiers, such as sampling identifier etc}
\NormalTok{dataConstants <-}\StringTok{ }\NormalTok{data }\OperatorTok
\StringTok{  }\KeywordTok{bind_rows}\NormalTok{() }\OperatorTok
\StringTok{  }\KeywordTok{select}\NormalTok{(}
\NormalTok{    sampling_event_identifier, time_observations_started,}
\NormalTok{    locality, locality_type, locality_id,}
\NormalTok{    observer_id, observation_date, scientific_name,}
\NormalTok{    observation_count, protocol_type, number_observers,}
\NormalTok{    longitude, latitude}
\NormalTok{  )}

\CommentTok{# join the summarised data with the identifiers,}
\CommentTok{# using sampling_event_identifier as the key}
\NormalTok{dataGrouped <-}\StringTok{ }\KeywordTok{left_join}\NormalTok{(dataGrouped, dataConstants,}
  \DataTypeTok{by =} \StringTok{"sampling_event_identifier"}
\NormalTok{)}

\CommentTok{# remove checklists or seis with more than 10 obervers}
\KeywordTok{count}\NormalTok{(dataGrouped, number_observers }\OperatorTok{>}\StringTok{ }\DecValTok{10}\NormalTok{) }\CommentTok{# count how many have 10+ obs}
\NormalTok{dataGrouped <-}\StringTok{ }\KeywordTok{filter}\NormalTok{(dataGrouped, number_observers }\OperatorTok{<=}\StringTok{ }\DecValTok{10}\NormalTok{)}
\end{Highlighting}
\end{Shaded}

\hypertarget{add-decimal-time}{%
\subsection{Add decimal time}\label{add-decimal-time}}

We added a column where time is denoted in decimal hours since midnight.

\begin{Shaded}
\begin{Highlighting}[numbers=left,,]
\CommentTok{# assign present or not, and get time in decimal hours since midnight}
\KeywordTok{library}\NormalTok{(lubridate)}
\NormalTok{time_to_decimal <-}\StringTok{ }\ControlFlowTok{function}\NormalTok{(x) \{}
\NormalTok{  x <-}\StringTok{ }\KeywordTok{hms}\NormalTok{(x, }\DataTypeTok{quiet =} \OtherTok{TRUE}\NormalTok{)}
  \KeywordTok{hour}\NormalTok{(x) }\OperatorTok{+}\StringTok{ }\KeywordTok{minute}\NormalTok{(x) }\OperatorTok{/}\StringTok{ }\DecValTok{60} \OperatorTok{+}\StringTok{ }\KeywordTok{second}\NormalTok{(x) }\OperatorTok{/}\StringTok{ }\DecValTok{3600}
\NormalTok{\}}

\CommentTok{# will cause issues if using time obs started as a linear effect and not quadratic}
\NormalTok{dataGrouped <-}\StringTok{ }\KeywordTok{mutate}\NormalTok{(dataGrouped,}
  \DataTypeTok{pres_abs =}\NormalTok{ observation_count }\OperatorTok{>=}\StringTok{ }\DecValTok{1}\NormalTok{,}
  \DataTypeTok{decimalTime =} \KeywordTok{time_to_decimal}\NormalTok{(time_observations_started)}
\NormalTok{)}

\CommentTok{# check class of dataGrouped, make sure not sf}
\NormalTok{assertthat}\OperatorTok{::}\KeywordTok{assert_that}\NormalTok{(}\OperatorTok{!}\StringTok{"sf"} \OperatorTok\StringTok{ }\KeywordTok{class}\NormalTok{(dataGrouped))}
\end{Highlighting}
\end{Shaded}

The above data is saved to a file.

\begin{Shaded}
\begin{Highlighting}[numbers=left,,]

\CommentTok{# save a temp data file}
\KeywordTok{save}\NormalTok{(dataGrouped, }\DataTypeTok{file =} \StringTok{"data/01_data_prelim_processing.rdata"}\NormalTok{)}
\end{Highlighting}
\end{Shaded}

\hypertarget{preparing-environmental-predictors}{%
\section{Preparing Environmental Predictors}\label{preparing-environmental-predictors}}

In this script, we processed climatic and landscape predictors for occupancy modeling.

All climatic data was obtained from \url{https://chelsa-climate.org/bioclim/}
All landscape data was derived from a high resolution Sentinel-2 composite image that was classified using Google Earth Engine. This code can be accessed in Section 3 of the Supplementary Material and here: \url{https://code.earthengine.google.com/ec69fc4ffad32a532b25202009243d42}.

The goal here is to resample all rasters so that they have the same resolution of 1km cells. We also tested for spatial autocorrelation among climatic predictors, which can be found in Section 4 of the Supplementary Material.

\hypertarget{prepare-libraries}{%
\subsection{Prepare libraries}\label{prepare-libraries}}

We load some common libraries for raster processing and define a custom mode function.

\begin{Shaded}
\begin{Highlighting}[numbers=left,,]

\CommentTok{# load libs}
\KeywordTok{library}\NormalTok{(raster)}
\KeywordTok{library}\NormalTok{(stringi)}
\KeywordTok{library}\NormalTok{(glue)}
\KeywordTok{library}\NormalTok{(gdalUtils)}
\KeywordTok{library}\NormalTok{(purrr)}

\CommentTok{# prep mode function to aggregate}
\NormalTok{funcMode <-}\StringTok{ }\ControlFlowTok{function}\NormalTok{(x, }\DataTypeTok{na.rm =}\NormalTok{ T) \{}
\NormalTok{  ux <-}\StringTok{ }\KeywordTok{unique}\NormalTok{(x)}
\NormalTok{  ux[}\KeywordTok{which.max}\NormalTok{(}\KeywordTok{tabulate}\NormalTok{(}\KeywordTok{match}\NormalTok{(x, ux)))]}
\NormalTok{\}}

\CommentTok{# a basic test}
\NormalTok{assertthat}\OperatorTok{::}\KeywordTok{assert_that}\NormalTok{(}\KeywordTok{funcMode}\NormalTok{(}\KeywordTok{c}\NormalTok{(}\DecValTok{2}\NormalTok{, }\DecValTok{2}\NormalTok{, }\DecValTok{2}\NormalTok{, }\DecValTok{2}\NormalTok{, }\DecValTok{3}\NormalTok{, }\DecValTok{3}\NormalTok{, }\DecValTok{3}\NormalTok{, }\DecValTok{4}\NormalTok{)) }\OperatorTok{==}\StringTok{ }\KeywordTok{as.character}\NormalTok{(}\DecValTok{2}\NormalTok{),}
  \DataTypeTok{msg =} \StringTok{"problem in the mode function"}
\NormalTok{) }\CommentTok{# works}
\end{Highlighting}
\end{Shaded}

\hypertarget{prepare-spatial-extent}{%
\subsection{Prepare spatial extent}\label{prepare-spatial-extent}}

We prepare a 30km buffer around the boundary of the study area. This buffer will be used to mask the landscape rasters.The buffer procedure is done on data transformed to the UTM 43N CRS to avoid distortions.

\begin{Shaded}
\begin{Highlighting}[numbers=left,,]
\CommentTok{# load hills}
\KeywordTok{library}\NormalTok{(sf)}
\NormalTok{hills <-}\StringTok{ }\KeywordTok{st_read}\NormalTok{(}\StringTok{"data/spatial/hillsShapefile/Nil_Ana_Pal.shp"}\NormalTok{)}
\NormalTok{hills <-}\StringTok{ }\KeywordTok{st_transform}\NormalTok{(hills, }\DecValTok{32643}\NormalTok{)}
\NormalTok{buffer <-}\StringTok{ }\KeywordTok{st_buffer}\NormalTok{(hills, }\FloatTok{3e4}\NormalTok{) }\OperatorTok
\StringTok{  }\KeywordTok{st_transform}\NormalTok{(}\DecValTok{4326}\NormalTok{)}
\end{Highlighting}
\end{Shaded}

\hypertarget{prepare-terrain-rasters}{%
\subsection{Prepare terrain rasters}\label{prepare-terrain-rasters}}

We prepare the elevation data which is an SRTM raster layer, and derive the slope and aspect from it after cropping it to the extent of the study site buffer. Please download the latest version of the SRTM raster layer from \url{https://www.worldclim.org/data/worldclim21.html}

\begin{Shaded}
\begin{Highlighting}[numbers=left,,]
\CommentTok{# load elevation and crop to hills size, then mask by hills}
\NormalTok{alt <-}\StringTok{ }\KeywordTok{raster}\NormalTok{(}\StringTok{"data/spatial/Elevation/alt"}\NormalTok{) }\CommentTok{# this layer is not added to github as a result of its large size and can be downloaded from the above link}
\NormalTok{alt.hills <-}\StringTok{ }\KeywordTok{crop}\NormalTok{(alt, }\KeywordTok{as}\NormalTok{(buffer, }\StringTok{"Spatial"}\NormalTok{))}
\KeywordTok{rm}\NormalTok{(alt)}
\KeywordTok{gc}\NormalTok{()}

\CommentTok{# get slope and aspect}
\NormalTok{slopeData <-}\StringTok{ }\KeywordTok{terrain}\NormalTok{(}\DataTypeTok{x =}\NormalTok{ alt.hills, }\DataTypeTok{opt =} \KeywordTok{c}\NormalTok{(}\StringTok{"slope"}\NormalTok{, }\StringTok{"aspect"}\NormalTok{))}
\NormalTok{elevData <-}\StringTok{ }\NormalTok{raster}\OperatorTok{::}\KeywordTok{stack}\NormalTok{(alt.hills, slopeData)}
\KeywordTok{rm}\NormalTok{(alt.hills)}
\KeywordTok{gc}\NormalTok{()}
\end{Highlighting}
\end{Shaded}

\hypertarget{prepare-chelsa-rasters}{%
\subsection{Prepare CHELSA rasters}\label{prepare-chelsa-rasters}}

We prepare the CHELSA rasters for annual temperature and annual precipitation in the same way, reading them in, cropping them to the study site buffer extent, and handling the temperature layer values which we divide by 10. The CHELSA rasters can be downloaded from \url{https://chelsa-climate.org/bioclim/}

\begin{Shaded}
\begin{Highlighting}[numbers=left,,]
\CommentTok{# list chelsa files}
\CommentTok{# the chelsa data for Annual mean temperature and annual precipitation can be downloaded from the aforementioned link. They haven't been uploaded to github as a result of its large size.}
\NormalTok{chelsaFiles <-}\StringTok{ }\KeywordTok{list.files}\NormalTok{(}\StringTok{"data/chelsa/"}\NormalTok{,}
  \DataTypeTok{full.names =} \OtherTok{TRUE}\NormalTok{,}
  \DataTypeTok{pattern =} \StringTok{"*.tif"}
\NormalTok{)}

\CommentTok{# gather chelsa rasters}
\NormalTok{chelsaData <-}\StringTok{ }\NormalTok{purrr}\OperatorTok{::}\KeywordTok{map}\NormalTok{(chelsaFiles, }\ControlFlowTok{function}\NormalTok{(chr) \{}
\NormalTok{  a <-}\StringTok{ }\KeywordTok{raster}\NormalTok{(chr)}
  \KeywordTok{crs}\NormalTok{(a) <-}\StringTok{ }\KeywordTok{crs}\NormalTok{(elevData)}
\NormalTok{  a <-}\StringTok{ }\KeywordTok{crop}\NormalTok{(a, }\KeywordTok{as}\NormalTok{(buffer, }\StringTok{"Spatial"}\NormalTok{))}
  \KeywordTok{return}\NormalTok{(a)}
\NormalTok{\})}

\CommentTok{# divide temperature by 10}
\NormalTok{chelsaData[[}\DecValTok{1}\NormalTok{]] <-}\StringTok{ }\NormalTok{chelsaData[[}\DecValTok{1}\NormalTok{]] }\OperatorTok{/}\StringTok{ }\DecValTok{10}

\CommentTok{# stack chelsa data}
\NormalTok{chelsaData <-}\StringTok{ }\NormalTok{raster}\OperatorTok{::}\KeywordTok{stack}\NormalTok{(chelsaData)}
\end{Highlighting}
\end{Shaded}

We stack the terrain and climatic rasters.

\begin{Shaded}
\begin{Highlighting}[numbers=left,,]
\CommentTok{# stack rasters for efficient reprojection later}
\NormalTok{env_data <-}\StringTok{ }\KeywordTok{stack}\NormalTok{(elevData, chelsaData)}
\end{Highlighting}
\end{Shaded}

\hypertarget{resample-landcover-from-10m-to-1km}{%
\subsection{Resample landcover from 10m to 1km}\label{resample-landcover-from-10m-to-1km}}

We read in a land cover classified image and resample that using the mode function to a 1km resolution. Please note that the resampling process need not be carried out as it has been done already and the resampled raster can be loaded with the subsequent code chunk.

The classified image can be obtained from: \url{https://code.earthengine.google.com/ec69fc4ffad32a532b25202009243d42}.

\begin{Shaded}
\begin{Highlighting}[numbers=left,,]
\CommentTok{# read in landcover raster location}
\NormalTok{landcover <-}\StringTok{ "data/landUseClassification/classifiedImage-UTM.tif"}

\CommentTok{# get extent}
\NormalTok{e <-}\StringTok{ }\KeywordTok{bbox}\NormalTok{(}\KeywordTok{raster}\NormalTok{(landcover))}

\CommentTok{# init resolution}
\NormalTok{res_init <-}\StringTok{ }\KeywordTok{res}\NormalTok{(}\KeywordTok{raster}\NormalTok{(landcover))}

\CommentTok{# res to transform to 1000m}
\NormalTok{res_final <-}\StringTok{ }\NormalTok{res_init }\OperatorTok{*}\StringTok{ }\DecValTok{100}

\CommentTok{# use gdalutils gdalwarp for resampling transform}
\CommentTok{# to 1km from 10m}
\NormalTok{gdalUtils}\OperatorTok{::}\KeywordTok{gdalwarp}\NormalTok{(}
  \DataTypeTok{srcfile =}\NormalTok{ landcover,}
  \DataTypeTok{dstfile =} \StringTok{"data/landUseClassification/lc_01000m.tif"}\NormalTok{,}
  \DataTypeTok{tr =} \KeywordTok{c}\NormalTok{(res_final), }\DataTypeTok{r =} \StringTok{"mode"}\NormalTok{, }\DataTypeTok{te =} \KeywordTok{c}\NormalTok{(e)}
\NormalTok{)}
\end{Highlighting}
\end{Shaded}

We compare the frequency of landcover classes between the original 10m and the resampled 1km raster to be certain that the resampling has not resulted in drastic misrepresentation of the frequency of any landcover type. This comparison is made using the figure below.

\hypertarget{resample-other-rasters-to-1km}{%
\subsection{Resample other rasters to 1km}\label{resample-other-rasters-to-1km}}

We now resample all other rasters to a resolution of 1km.

\hypertarget{read-in-resampled-landcover}{%
\subsubsection{Read in resampled landcover}\label{read-in-resampled-landcover}}

Here, we read in the 1km landcover raster and set 0 to NA.

\begin{Shaded}
\begin{Highlighting}[numbers=left,,]
\NormalTok{lc_data <-}\StringTok{ }\KeywordTok{raster}\NormalTok{(}\StringTok{"data/landUseClassification/lc_01000m.tif"}\NormalTok{)}
\NormalTok{lc_data[lc_data }\OperatorTok{==}\StringTok{ }\DecValTok{0}\NormalTok{] <-}\StringTok{ }\OtherTok{NA}
\end{Highlighting}
\end{Shaded}

\hypertarget{reproject-environmental-data-using-landcover-as-a-template}{%
\subsubsection{Reproject environmental data using landcover as a template}\label{reproject-environmental-data-using-landcover-as-a-template}}

\begin{Shaded}
\begin{Highlighting}[numbers=left,,]

\CommentTok{# resample to the corresponding landcover data}
\NormalTok{env_data_resamp <-}\StringTok{ }\KeywordTok{projectRaster}\NormalTok{(}
  \DataTypeTok{from =}\NormalTok{ env_data, }\DataTypeTok{to =}\NormalTok{ lc_data,}
  \DataTypeTok{crs =} \KeywordTok{crs}\NormalTok{(lc_data), }\DataTypeTok{res =} \KeywordTok{res}\NormalTok{(lc_data)}
\NormalTok{)}

\CommentTok{# export as raster stack}
\NormalTok{land_stack <-}\StringTok{ }\KeywordTok{stack}\NormalTok{(env_data_resamp, lc_data)}

\CommentTok{# get names}
\NormalTok{land_names <-}\StringTok{ }\KeywordTok{glue}\NormalTok{(}\StringTok{'data/spatial/landscape_resamp\{c("01")\}_km.tif'}\NormalTok{)}

\CommentTok{# write to file}
\KeywordTok{writeRaster}\NormalTok{(land_stack, }\DataTypeTok{filename =} \KeywordTok{as.character}\NormalTok{(land_names), }\DataTypeTok{overwrite =} \OtherTok{TRUE}\NormalTok{)}
\end{Highlighting}
\end{Shaded}

\hypertarget{temperature-and-rainfall-in-relation-to-elevation}{%
\subsection{Temperature and rainfall in relation to elevation}\label{temperature-and-rainfall-in-relation-to-elevation}}

\hypertarget{prepare-libraries-and-ci-function}{%
\subsubsection{Prepare libraries and CI function}\label{prepare-libraries-and-ci-function}}

\begin{Shaded}
\begin{Highlighting}[numbers=left,,]
\CommentTok{# load libs}
\KeywordTok{library}\NormalTok{(dplyr)}
\KeywordTok{library}\NormalTok{(tidyr)}

\KeywordTok{library}\NormalTok{(scales)}
\KeywordTok{library}\NormalTok{(ggplot2)}

\CommentTok{# get ci func}
\NormalTok{ci <-}\StringTok{ }\ControlFlowTok{function}\NormalTok{(x) \{}
  \KeywordTok{qnorm}\NormalTok{(}\FloatTok{0.975}\NormalTok{) }\OperatorTok{*}\StringTok{ }\KeywordTok{sd}\NormalTok{(x, }\DataTypeTok{na.rm =}\NormalTok{ T) }\OperatorTok{/}\StringTok{ }\KeywordTok{sqrt}\NormalTok{(}\KeywordTok{length}\NormalTok{(x))}
\NormalTok{\}}
\end{Highlighting}
\end{Shaded}

\hypertarget{load-resampled-environmental-rasters}{%
\subsubsection{Load resampled environmental rasters}\label{load-resampled-environmental-rasters}}

\begin{Shaded}
\begin{Highlighting}[numbers=left,,]
\CommentTok{# read landscape prepare for plotting}
\NormalTok{landscape <-}\StringTok{ }\KeywordTok{stack}\NormalTok{(}\StringTok{"data/spatial/landscape_resamp01_km.tif"}\NormalTok{)}

\CommentTok{# get proper names}
\NormalTok{elev_names <-}\StringTok{ }\KeywordTok{c}\NormalTok{(}\StringTok{"elev"}\NormalTok{, }\StringTok{"slope"}\NormalTok{, }\StringTok{"aspect"}\NormalTok{)}
\NormalTok{chelsa_names <-}\StringTok{ }\KeywordTok{c}\NormalTok{(}\StringTok{"bio_01"}\NormalTok{, }\StringTok{"bio_12"}\NormalTok{)}

\KeywordTok{names}\NormalTok{(landscape) <-}\StringTok{ }\KeywordTok{as.character}\NormalTok{(}\KeywordTok{glue}\NormalTok{(}\StringTok{'\{c(elev_names, chelsa_names, "landcover")\}'}\NormalTok{))}
\end{Highlighting}
\end{Shaded}

\begin{Shaded}
\begin{Highlighting}[numbers=left,,]
\CommentTok{# make duplicate stack}
\NormalTok{land_data <-}\StringTok{ }\NormalTok{landscape[[}\KeywordTok{c}\NormalTok{(}\StringTok{"elev"}\NormalTok{, chelsa_names)]]}

\CommentTok{# convert to list}
\NormalTok{land_data <-}\StringTok{ }\KeywordTok{as.list}\NormalTok{(land_data)}

\CommentTok{# map get values over the stack}
\NormalTok{land_data <-}\StringTok{ }\NormalTok{purrr}\OperatorTok{::}\KeywordTok{map}\NormalTok{(land_data, getValues)}
\KeywordTok{names}\NormalTok{(land_data) <-}\StringTok{ }\KeywordTok{c}\NormalTok{(}\StringTok{"elev"}\NormalTok{, chelsa_names)}

\CommentTok{# conver to dataframe and round to 100m}
\NormalTok{land_data <-}\StringTok{ }\KeywordTok{bind_cols}\NormalTok{(land_data)}
\NormalTok{land_data <-}\StringTok{ }\KeywordTok{drop_na}\NormalTok{(land_data) }\OperatorTok
\StringTok{  }\KeywordTok{mutate}\NormalTok{(}\DataTypeTok{elev_round =}\NormalTok{ plyr}\OperatorTok{::}\KeywordTok{round_any}\NormalTok{(elev, }\DecValTok{200}\NormalTok{)) }\OperatorTok
\StringTok{  }\NormalTok{dplyr}\OperatorTok{::}\KeywordTok{select}\NormalTok{(}\OperatorTok{-}\NormalTok{elev) }\OperatorTok
\StringTok{  }\KeywordTok{pivot_longer}\NormalTok{(}
    \DataTypeTok{cols =} \KeywordTok{contains}\NormalTok{(}\StringTok{"bio"}\NormalTok{),}
    \DataTypeTok{names_to =} \StringTok{"clim_var"}
\NormalTok{  ) }\OperatorTok
\StringTok{  }\KeywordTok{group_by}\NormalTok{(elev_round, clim_var) }\OperatorTok
\StringTok{  }\KeywordTok{summarise_all}\NormalTok{(}\DataTypeTok{.funs =} \KeywordTok{list}\NormalTok{(}\OperatorTok{~}\StringTok{ }\KeywordTok{mean}\NormalTok{(.), }\OperatorTok{~}\StringTok{ }\KeywordTok{ci}\NormalTok{(.)))}
\end{Highlighting}
\end{Shaded}

Figure code is hidden in versions rendered as HTML or PDF.

\hypertarget{land-cover-type-in-relation-to-elevation}{%
\subsection{Land cover type in relation to elevation}\label{land-cover-type-in-relation-to-elevation}}

\begin{Shaded}
\begin{Highlighting}[numbers=left,,]
\CommentTok{# get data from landscape rasters}
\NormalTok{lc_elev <-}\StringTok{ }\KeywordTok{tibble}\NormalTok{(}
  \DataTypeTok{elev =} \KeywordTok{getValues}\NormalTok{(landscape[[}\StringTok{"elev"}\NormalTok{]]),}
  \DataTypeTok{landcover =} \KeywordTok{getValues}\NormalTok{(landscape[[}\StringTok{"landcover"}\NormalTok{]])}
\NormalTok{)}

\CommentTok{# process data for proportions}
\NormalTok{lc_elev <-}\StringTok{ }\NormalTok{lc_elev }\OperatorTok
\StringTok{  }\KeywordTok{filter}\NormalTok{(}\OperatorTok{!}\KeywordTok{is.na}\NormalTok{(landcover), }\OperatorTok{!}\KeywordTok{is.na}\NormalTok{(elev)) }\OperatorTok
\StringTok{  }\KeywordTok{mutate}\NormalTok{(}\DataTypeTok{elev =}\NormalTok{ plyr}\OperatorTok{::}\KeywordTok{round_any}\NormalTok{(elev, }\DecValTok{100}\NormalTok{)) }\OperatorTok
\StringTok{  }\KeywordTok{count}\NormalTok{(elev, landcover) }\OperatorTok
\StringTok{  }\KeywordTok{group_by}\NormalTok{(elev) }\OperatorTok
\StringTok{  }\KeywordTok{mutate}\NormalTok{(}\DataTypeTok{prop =}\NormalTok{ n }\OperatorTok{/}\StringTok{ }\KeywordTok{sum}\NormalTok{(n))}

\CommentTok{# fill out lc elev}
\NormalTok{lc_elev_canon <-}\StringTok{ }\KeywordTok{crossing}\NormalTok{(}
  \DataTypeTok{elev =} \KeywordTok{unique}\NormalTok{(lc_elev}\OperatorTok{$}\NormalTok{elev),}
  \DataTypeTok{landcover =} \KeywordTok{unique}\NormalTok{(lc_elev}\OperatorTok{$}\NormalTok{landcover)}
\NormalTok{)}

\CommentTok{# bind with lcelev}
\NormalTok{lc_elev <-}\StringTok{ }\KeywordTok{full_join}\NormalTok{(lc_elev, lc_elev_canon)}

\CommentTok{# convert NA to zero}
\NormalTok{lc_elev <-}\StringTok{ }\KeywordTok{replace_na}\NormalTok{(lc_elev, }\DataTypeTok{replace =} \KeywordTok{list}\NormalTok{(}\DataTypeTok{n =} \DecValTok{0}\NormalTok{, }\DataTypeTok{prop =} \DecValTok{0}\NormalTok{))}
\end{Highlighting}
\end{Shaded}

Figure code is hidden in versions rendered as HTML and PDF.

\hypertarget{main-text-figure-2}{%
\subsection{Main Text Figure 2}\label{main-text-figure-2}}

\begin{figure}
\centering
\includegraphics{figs/fig_02_clim_lc_elev.png}
\caption{Annual Mean Temperature varied from \textasciitilde{}28C in the plains to \textless{}14C at higher elevations. Annual precipitation increased at lower elevations (in the plains) to \textasciitilde{}3000mm and ranged between 1500mm and 2200mm at mid- and high elevations across the Nilgiri and the Anamalai hills. (b) The proportion of land cover types varied across the study area as shown in this panel (1 = agriculture; 2 = forests; 3 = grasslands; 4 = plantations; 5 = settlements; 6 = tea; 7 = water bodies).}
\end{figure}

\hypertarget{preparing-observer-expertise-scores}{%
\section{Preparing Observer Expertise Scores}\label{preparing-observer-expertise-scores}}

Differences in local avifaunal expertise among citizen scientists can lead to biased species detection when compared with data collected by a consistent set of trained observers (van Strien, van Swaay, and Termaat \protect\hyperlink{ref-vanstrien2013}{2013}). Including observer expertise as a detection covariate in occupancy models using eBird data can help account for this variation (Johnston et al. \protect\hyperlink{ref-johnston2018}{2018}). Observer-specific expertise in local avifauna was calculated following (Kelling et al. \protect\hyperlink{ref-kelling2015a}{2015}) as the normalized predicted number of species reported by an observer after 60 minutes of sampling across the most common land cover type within the study area. This score was calculated by examining checklists from anonymized observers across the study area. We modified Kelling et al.~(2015) formulation by including only observations of the 93 species of interest in our calculations. An observer with a higher number of species of interest reported within 60 minutes would have a higher observer-specific expertise score, with respect to the study area.

Plots with respect to how observer expertise varied over time (2013-2019) for the list of species considered in this study (across the study area alone) can be accessed in Section 7 of the Supplementary Material.

\hypertarget{prepare-libraries-1}{%
\subsection{Prepare libraries}\label{prepare-libraries-1}}

\begin{Shaded}
\begin{Highlighting}[numbers=left,,]

\CommentTok{# load libs}
\KeywordTok{library}\NormalTok{(data.table)}
\KeywordTok{library}\NormalTok{(readxl)}
\KeywordTok{library}\NormalTok{(magrittr)}
\KeywordTok{library}\NormalTok{(stringr)}
\KeywordTok{library}\NormalTok{(dplyr)}
\KeywordTok{library}\NormalTok{(tidyr)}
\KeywordTok{library}\NormalTok{(auk)}

\CommentTok{# get decimal time function}
\KeywordTok{library}\NormalTok{(lubridate)}
\NormalTok{time_to_decimal <-}\StringTok{ }\ControlFlowTok{function}\NormalTok{(x) \{}
\NormalTok{  x <-}\StringTok{ }\NormalTok{lubridate}\OperatorTok{::}\KeywordTok{hms}\NormalTok{(x, }\DataTypeTok{quiet =} \OtherTok{TRUE}\NormalTok{)}
\NormalTok{  lubridate}\OperatorTok{::}\KeywordTok{hour}\NormalTok{(x) }\OperatorTok{+}\StringTok{ }\NormalTok{lubridate}\OperatorTok{::}\KeywordTok{minute}\NormalTok{(x) }\OperatorTok{/}\StringTok{ }\DecValTok{60} \OperatorTok{+}\StringTok{ }\NormalTok{lubridate}\OperatorTok{::}\KeywordTok{second}\NormalTok{(x) }\OperatorTok{/}\StringTok{ }\DecValTok{3600}
\NormalTok{\}}
\end{Highlighting}
\end{Shaded}

\hypertarget{prepare-data}{%
\subsection{Prepare data}\label{prepare-data}}

Here, we go through the data preparation process again because we might want to assess observer expertise over a larger area than the study site.

\begin{Shaded}
\begin{Highlighting}[numbers=left,,]
\CommentTok{# Read in shapefile of study area to subset by bounding box}
\KeywordTok{library}\NormalTok{(sf)}
\NormalTok{wg <-}\StringTok{ }\KeywordTok{st_read}\NormalTok{(}\StringTok{"data/spatial/hillsShapefile/Nil_Ana_Pal.shp"}\NormalTok{) }\OperatorTok
\StringTok{  }\KeywordTok{st_transform}\NormalTok{(}\DecValTok{32643}\NormalTok{)}

\CommentTok{# set file paths for auk functions}
\NormalTok{f_in_ebd <-}\StringTok{ }\KeywordTok{file.path}\NormalTok{(}\StringTok{"data/01_ebird-filtered-EBD-westernGhats.txt"}\NormalTok{)}
\NormalTok{f_in_sampling <-}\StringTok{ }\KeywordTok{file.path}\NormalTok{(}\StringTok{"data/01_ebird-filtered-sampling-westernGhats.txt"}\NormalTok{)}

\CommentTok{# run filters using auk packages}
\NormalTok{ebd_filters <-}\StringTok{ }\KeywordTok{auk_ebd}\NormalTok{(f_in_ebd, f_in_sampling) }\OperatorTok
\StringTok{  }\KeywordTok{auk_country}\NormalTok{(}\DataTypeTok{country =} \StringTok{"IN"}\NormalTok{) }\OperatorTok
\StringTok{  }\KeywordTok{auk_state}\NormalTok{(}\KeywordTok{c}\NormalTok{(}\StringTok{"IN-KL"}\NormalTok{, }\StringTok{"IN-TN"}\NormalTok{, }\StringTok{"IN-KA"}\NormalTok{)) }\OperatorTok
\StringTok{  }\CommentTok{# Restricting geography to TamilNadu, Kerala & Karnataka}
\StringTok{  }\KeywordTok{auk_date}\NormalTok{(}\KeywordTok{c}\NormalTok{(}\StringTok{"2013-01-01"}\NormalTok{, }\StringTok{"2019-12-31"}\NormalTok{)) }\OperatorTok
\StringTok{  }\KeywordTok{auk_complete}\NormalTok{()}

\CommentTok{# check filters}
\NormalTok{ebd_filters}

\CommentTok{# specify output location and perform filter}
\NormalTok{f_out_ebd <-}\StringTok{ "data/ebird_for_expertise.txt"}
\NormalTok{f_out_sampling <-}\StringTok{ "data/ebird_sampling_for_expertise.txt"}

\NormalTok{ebd_filtered <-}\StringTok{ }\KeywordTok{auk_filter}\NormalTok{(ebd_filters,}
  \DataTypeTok{file =}\NormalTok{ f_out_ebd,}
  \DataTypeTok{file_sampling =}\NormalTok{ f_out_sampling, }\DataTypeTok{overwrite =} \OtherTok{TRUE}
\NormalTok{)}
\end{Highlighting}
\end{Shaded}

Load in the filtered data and columns of interest.

\begin{Shaded}
\begin{Highlighting}[numbers=left,,]
\CommentTok{## Process filtered data}
\CommentTok{# read in the data}
\NormalTok{ebd <-}\StringTok{ }\KeywordTok{fread}\NormalTok{(f_out_ebd)}
\NormalTok{names <-}\StringTok{ }\KeywordTok{names}\NormalTok{(ebd) }\OperatorTok
\StringTok{  }\NormalTok{stringr}\OperatorTok{::}\KeywordTok{str_to_lower}\NormalTok{() }\OperatorTok
\StringTok{  }\NormalTok{stringr}\OperatorTok{::}\KeywordTok{str_replace_all}\NormalTok{(}\StringTok{" "}\NormalTok{, }\StringTok{"_"}\NormalTok{)}

\KeywordTok{setnames}\NormalTok{(ebd, names)}
\CommentTok{# choose columns of interest}
\NormalTok{columnsOfInterest <-}\StringTok{ }\KeywordTok{c}\NormalTok{(}
  \StringTok{"checklist_id"}\NormalTok{, }\StringTok{"scientific_name"}\NormalTok{, }\StringTok{"observation_count"}\NormalTok{,}
  \StringTok{"locality"}\NormalTok{, }\StringTok{"locality_id"}\NormalTok{, }\StringTok{"locality_type"}\NormalTok{, }\StringTok{"latitude"}\NormalTok{,}
  \StringTok{"longitude"}\NormalTok{, }\StringTok{"observation_date"}\NormalTok{,}
  \StringTok{"time_observations_started"}\NormalTok{, }\StringTok{"observer_id"}\NormalTok{,}
  \StringTok{"sampling_event_identifier"}\NormalTok{, }\StringTok{"protocol_type"}\NormalTok{,}
  \StringTok{"duration_minutes"}\NormalTok{, }\StringTok{"effort_distance_km"}\NormalTok{, }\StringTok{"effort_area_ha"}\NormalTok{,}
  \StringTok{"number_observers"}\NormalTok{, }\StringTok{"species_observed"}\NormalTok{, }\StringTok{"reviewed"}
\NormalTok{)}

\NormalTok{ebd <-}\StringTok{ }\KeywordTok{setDF}\NormalTok{(ebd) }\OperatorTok
\StringTok{  }\KeywordTok{as_tibble}\NormalTok{() }\OperatorTok
\StringTok{  }\NormalTok{dplyr}\OperatorTok{::}\KeywordTok{select}\NormalTok{(}\KeywordTok{one_of}\NormalTok{(columnsOfInterest))}

\KeywordTok{setDT}\NormalTok{(ebd)}
\end{Highlighting}
\end{Shaded}

\hypertarget{spatially-explicit-filter-on-checklists}{%
\subsection{Spatially explicit filter on checklists}\label{spatially-explicit-filter-on-checklists}}

\begin{Shaded}
\begin{Highlighting}[numbers=left,,]
\CommentTok{# get checklist locations}
\NormalTok{ebd_locs <-}\StringTok{ }\NormalTok{ebd[, .(longitude, latitude)]}
\NormalTok{ebd_locs <-}\StringTok{ }\KeywordTok{setDF}\NormalTok{(ebd_locs) }\OperatorTok\StringTok{ }\KeywordTok{distinct}\NormalTok{()}
\NormalTok{ebd_locs <-}\StringTok{ }\KeywordTok{st_as_sf}\NormalTok{(ebd_locs,}
  \DataTypeTok{coords =} \KeywordTok{c}\NormalTok{(}\StringTok{"longitude"}\NormalTok{, }\StringTok{"latitude"}\NormalTok{)}
\NormalTok{) }\OperatorTok
\StringTok{  `}\DataTypeTok{st_crs<-}\StringTok{`}\NormalTok{(}\DecValTok{4326}\NormalTok{) }\OperatorTok
\StringTok{  }\KeywordTok{bind_cols}\NormalTok{(}\KeywordTok{as_tibble}\NormalTok{(}\KeywordTok{st_coordinates}\NormalTok{(.))) }\OperatorTok
\StringTok{  }\KeywordTok{st_transform}\NormalTok{(}\DecValTok{32643}\NormalTok{) }\OperatorTok
\StringTok{  }\KeywordTok{mutate}\NormalTok{(}\DataTypeTok{id =} \DecValTok{1}\OperatorTok{:}\KeywordTok{nrow}\NormalTok{(.))}

\CommentTok{# check whether to include}
\NormalTok{to_keep <-}\StringTok{ }\KeywordTok{unlist}\NormalTok{(}\KeywordTok{st_contains}\NormalTok{(wg, ebd_locs))}

\CommentTok{# filter locs}
\NormalTok{ebd_locs <-}\StringTok{ }\KeywordTok{filter}\NormalTok{(ebd_locs, id }\OperatorTok\StringTok{ }\NormalTok{to_keep) }\OperatorTok
\StringTok{  }\KeywordTok{bind_cols}\NormalTok{(}\KeywordTok{as_tibble}\NormalTok{(}\KeywordTok{st_coordinates}\NormalTok{(}\KeywordTok{st_as_sf}\NormalTok{(.)))) }\OperatorTok
\StringTok{  }\KeywordTok{st_drop_geometry}\NormalTok{()}
\end{Highlighting}
\end{Shaded}

\begin{Shaded}
\begin{Highlighting}[numbers=left,,]
\NormalTok{ebd <-}\StringTok{ }\NormalTok{ebd[longitude }\OperatorTok\StringTok{ }\NormalTok{ebd_locs}\OperatorTok{$}\NormalTok{X }\OperatorTok{&}\StringTok{ }\NormalTok{latitude }\OperatorTok\StringTok{ }\NormalTok{ebd_locs}\OperatorTok{$}\NormalTok{Y, ]}
\end{Highlighting}
\end{Shaded}

\hypertarget{prepare-species-of-interest}{%
\subsection{Prepare species of interest}\label{prepare-species-of-interest}}

\begin{Shaded}
\begin{Highlighting}[numbers=left,,]
\CommentTok{# read in species list}
\NormalTok{specieslist <-}\StringTok{ }\KeywordTok{read.csv}\NormalTok{(}\StringTok{"data/species_list.csv"}\NormalTok{)}

\CommentTok{# set species of interest}
\NormalTok{soi <-}\StringTok{ }\NormalTok{specieslist}\OperatorTok{$}\NormalTok{scientific_name}

\NormalTok{ebdSpSum <-}\StringTok{ }\NormalTok{ebd[, .(}
  \DataTypeTok{nSp =}\NormalTok{ .N,}
  \DataTypeTok{totSoiSeen =} \KeywordTok{length}\NormalTok{(}\KeywordTok{intersect}\NormalTok{(scientific_name, soi))}
\NormalTok{),}
\NormalTok{by =}\StringTok{ }\KeywordTok{list}\NormalTok{(sampling_event_identifier)}
\NormalTok{]}

\CommentTok{# write to file and link with checklist id later}
\KeywordTok{fwrite}\NormalTok{(ebdSpSum, }\DataTypeTok{file =} \StringTok{"data/03_data-nspp-per-chk.csv"}\NormalTok{)}
\end{Highlighting}
\end{Shaded}

\hypertarget{prepare-checklists-for-observer-score}{%
\subsection{Prepare checklists for observer score}\label{prepare-checklists-for-observer-score}}

\begin{Shaded}
\begin{Highlighting}[numbers=left,,]
\CommentTok{# 1. add new columns of decimal time and julian date}
\NormalTok{ebd[, }\StringTok{`}\DataTypeTok{:=}\StringTok{`}\NormalTok{(}
  \DataTypeTok{decimalTime =} \KeywordTok{time_to_decimal}\NormalTok{(time_observations_started),}
  \DataTypeTok{julianDate =} \KeywordTok{yday}\NormalTok{(}\KeywordTok{as.POSIXct}\NormalTok{(observation_date))}
\NormalTok{)]}

\NormalTok{ebdEffChk <-}\StringTok{ }\KeywordTok{setDF}\NormalTok{(ebd) }\OperatorTok
\StringTok{  }\KeywordTok{mutate}\NormalTok{(}\DataTypeTok{year =} \KeywordTok{year}\NormalTok{(observation_date)) }\OperatorTok
\StringTok{  }\KeywordTok{distinct}\NormalTok{(}
\NormalTok{    sampling_event_identifier, observer_id,}
\NormalTok{    year,}
\NormalTok{    duration_minutes, effort_distance_km, effort_area_ha,}
\NormalTok{    longitude, latitude,}
\NormalTok{    locality, locality_id,}
\NormalTok{    decimalTime, julianDate, number_observers}
\NormalTok{  ) }\OperatorTok
\StringTok{  }\CommentTok{# drop rows with NAs in cols used in the model}
\StringTok{  }\NormalTok{tidyr}\OperatorTok{::}\KeywordTok{drop_na}\NormalTok{(}
\NormalTok{    sampling_event_identifier, observer_id,}
\NormalTok{    duration_minutes, decimalTime, julianDate}
\NormalTok{  ) }\OperatorTok

\StringTok{  }\CommentTok{# drop years below 2013}
\StringTok{  }\KeywordTok{filter}\NormalTok{(year }\OperatorTok{>=}\StringTok{ }\DecValTok{2013}\NormalTok{)}

\CommentTok{# 3. join to covariates and remove large groups (> 10)}
\NormalTok{ebdChkSummary <-}\StringTok{ }\KeywordTok{inner_join}\NormalTok{(ebdEffChk, ebdSpSum)}

\CommentTok{# remove ebird data}
\KeywordTok{rm}\NormalTok{(ebd)}
\KeywordTok{gc}\NormalTok{()}
\end{Highlighting}
\end{Shaded}

\hypertarget{get-landcover}{%
\subsection{Get landcover}\label{get-landcover}}

Read in land cover type data resampled at 1km resolution.

\begin{Shaded}
\begin{Highlighting}[numbers=left,,]
\CommentTok{# read in 1km landcover and set 0 to NA}
\KeywordTok{library}\NormalTok{(raster)}
\NormalTok{landcover <-}\StringTok{ }\NormalTok{raster}\OperatorTok{::}\KeywordTok{raster}\NormalTok{(}\StringTok{"data/landUseClassification/lc_01000m.tif"}\NormalTok{)}
\NormalTok{landcover[landcover }\OperatorTok{==}\StringTok{ }\DecValTok{0}\NormalTok{] <-}\StringTok{ }\OtherTok{NA}

\CommentTok{# get locs in utm coords}
\NormalTok{locs <-}\StringTok{ }\KeywordTok{distinct}\NormalTok{(}
\NormalTok{  ebdChkSummary, sampling_event_identifier, longitude, latitude,}
\NormalTok{  locality, locality_id}
\NormalTok{)}
\NormalTok{locs <-}\StringTok{ }\KeywordTok{st_as_sf}\NormalTok{(locs, }\DataTypeTok{coords =} \KeywordTok{c}\NormalTok{(}\StringTok{"longitude"}\NormalTok{, }\StringTok{"latitude"}\NormalTok{)) }\OperatorTok
\StringTok{  `}\DataTypeTok{st_crs<-}\StringTok{`}\NormalTok{(}\DecValTok{4326}\NormalTok{) }\OperatorTok
\StringTok{  }\KeywordTok{st_transform}\NormalTok{(}\DecValTok{32643}\NormalTok{) }\OperatorTok
\StringTok{  }\KeywordTok{st_coordinates}\NormalTok{()}

\CommentTok{# get for unique points}
\NormalTok{landcoverVec <-}\StringTok{ }\NormalTok{raster}\OperatorTok{::}\KeywordTok{extract}\NormalTok{(}
  \DataTypeTok{x =}\NormalTok{ landcover,}
  \DataTypeTok{y =}\NormalTok{ locs}
\NormalTok{)}

\CommentTok{# assign to df and overwrite}
\KeywordTok{setDT}\NormalTok{(ebdChkSummary)[, landcover }\OperatorTok{:}\ErrorTok{=}\StringTok{ }\NormalTok{landcoverVec]}
\end{Highlighting}
\end{Shaded}

\hypertarget{filter-checklist-data}{%
\subsection{Filter checklist data}\label{filter-checklist-data}}

\begin{Shaded}
\begin{Highlighting}[numbers=left,,]
\CommentTok{# change names for easy handling}
\KeywordTok{setnames}\NormalTok{(ebdChkSummary, }\KeywordTok{c}\NormalTok{(}
  \StringTok{"sei"}\NormalTok{, }\StringTok{"observer"}\NormalTok{, }\StringTok{"year"}\NormalTok{, }\StringTok{"duration"}\NormalTok{, }\StringTok{"distance"}\NormalTok{,}
  \StringTok{"area"}\NormalTok{, }\StringTok{"longitude"}\NormalTok{, }\StringTok{"latitude"}\NormalTok{, }\StringTok{"locality"}\NormalTok{,}
  \StringTok{"locality_id"}\NormalTok{, }\StringTok{"decimalTime"}\NormalTok{,}
  \StringTok{"julianDate"}\NormalTok{, }\StringTok{"nObs"}\NormalTok{, }\StringTok{"nSp"}\NormalTok{, }\StringTok{"nSoi"}\NormalTok{, }\StringTok{"landcover"}
\NormalTok{))}

\CommentTok{# count data points per observer}
\NormalTok{obscount <-}\StringTok{ }\KeywordTok{count}\NormalTok{(ebdChkSummary, observer) }\OperatorTok
\StringTok{  }\KeywordTok{filter}\NormalTok{(n }\OperatorTok{>=}\StringTok{ }\DecValTok{10}\NormalTok{)}

\CommentTok{# make factor variables and remove obs not in obscount}
\CommentTok{# also remove 0 durations}
\NormalTok{ebdChkSummary <-}\StringTok{ }\NormalTok{ebdChkSummary }\OperatorTok
\StringTok{  }\KeywordTok{mutate}\NormalTok{(}
    \DataTypeTok{distance =} \KeywordTok{ifelse}\NormalTok{(}\KeywordTok{is.na}\NormalTok{(distance), }\DecValTok{0}\NormalTok{, distance),}
    \DataTypeTok{duration =} \KeywordTok{if_else}\NormalTok{(}\KeywordTok{is.na}\NormalTok{(duration), }\FloatTok{0.0}\NormalTok{, }\KeywordTok{as.double}\NormalTok{(duration))}
\NormalTok{  ) }\OperatorTok
\StringTok{  }\KeywordTok{filter}\NormalTok{(}
\NormalTok{    observer }\OperatorTok\StringTok{ }\NormalTok{obscount}\OperatorTok{$}\NormalTok{observer,}
\NormalTok{    duration }\OperatorTok{>}\StringTok{ }\DecValTok{0}\NormalTok{,}
\NormalTok{    duration }\OperatorTok{<=}\StringTok{ }\DecValTok{300}\NormalTok{,}
\NormalTok{    nSoi }\OperatorTok{>=}\StringTok{ }\DecValTok{0}\NormalTok{,}
\NormalTok{    distance }\OperatorTok{<=}\StringTok{ }\DecValTok{5}\NormalTok{,}
    \OperatorTok{!}\KeywordTok{is.na}\NormalTok{(nSoi)}
\NormalTok{  ) }\OperatorTok
\StringTok{  }\KeywordTok{mutate}\NormalTok{(}
    \DataTypeTok{landcover =} \KeywordTok{as.factor}\NormalTok{(landcover),}
    \DataTypeTok{observer =} \KeywordTok{as.factor}\NormalTok{(observer)}
\NormalTok{  ) }\OperatorTok
\StringTok{  }\KeywordTok{drop_na}\NormalTok{(landcover)}


\CommentTok{# save to file for later reuse}
\KeywordTok{fwrite}\NormalTok{(ebdChkSummary, }\DataTypeTok{file =} \StringTok{"data/03_data-covars-perChklist.csv"}\NormalTok{)}
\end{Highlighting}
\end{Shaded}

\hypertarget{model-observer-expertise}{%
\subsection{Model observer expertise}\label{model-observer-expertise}}

Our observer expertise model aims to include the random intercept effect of observer identity, with a random slope effect of duration. This models the different rate of species accumulation by different observers, as well as their different starting points.

\begin{Shaded}
\begin{Highlighting}[numbers=left,,]
\CommentTok{# uses either a subset or all data}
\KeywordTok{library}\NormalTok{(lmerTest)}

\CommentTok{# here we specify a glmm with random effects for observer}
\CommentTok{# time is considered a fixed log predictor and a random slope}
\NormalTok{modObsExp <-}\StringTok{ }\KeywordTok{glmer}\NormalTok{(nSoi }\OperatorTok{~}\StringTok{ }\KeywordTok{sqrt}\NormalTok{(duration) }\OperatorTok{+}
\StringTok{  }\NormalTok{landcover }\OperatorTok{+}
\StringTok{  }\KeywordTok{sqrt}\NormalTok{(decimalTime) }\OperatorTok{+}
\StringTok{  }\KeywordTok{I}\NormalTok{((}\KeywordTok{sqrt}\NormalTok{(decimalTime))}\OperatorTok{^}\DecValTok{2}\NormalTok{) }\OperatorTok{+}
\StringTok{  }\KeywordTok{log}\NormalTok{(julianDate) }\OperatorTok{+}
\StringTok{  }\KeywordTok{I}\NormalTok{((}\KeywordTok{log}\NormalTok{(julianDate)}\OperatorTok{^}\DecValTok{2}\NormalTok{)) }\OperatorTok{+}
\StringTok{  }\NormalTok{(}\DecValTok{1} \OperatorTok{|}\StringTok{ }\NormalTok{observer) }\OperatorTok{+}\StringTok{ }\NormalTok{(}\DecValTok{0} \OperatorTok{+}\StringTok{ }\NormalTok{duration }\OperatorTok{|}\StringTok{ }\NormalTok{observer),}
\DataTypeTok{data =}\NormalTok{ ebdChkSummary, }\DataTypeTok{family =} \StringTok{"poisson"}
\NormalTok{)}
\end{Highlighting}
\end{Shaded}

\begin{Shaded}
\begin{Highlighting}[numbers=left,,]
\CommentTok{# make dir if absent}
\ControlFlowTok{if}\NormalTok{ (}\OperatorTok{!}\KeywordTok{dir.exists}\NormalTok{(}\StringTok{"data/modOutput"}\NormalTok{)) \{}
  \KeywordTok{dir.create}\NormalTok{(}\StringTok{"data/modOutput"}\NormalTok{)}
\NormalTok{\}}

\CommentTok{# write model output to text file}
\NormalTok{\{}
  \KeywordTok{writeLines}\NormalTok{(R.utils}\OperatorTok{::}\KeywordTok{captureOutput}\NormalTok{(}\KeywordTok{list}\NormalTok{(}\KeywordTok{Sys.time}\NormalTok{(), }\KeywordTok{summary}\NormalTok{(modObsExp))),}
    \DataTypeTok{con =} \StringTok{"data/modOutput/03_model-output-expertise.txt"}
\NormalTok{  )}
\NormalTok{\}}
\end{Highlighting}
\end{Shaded}

\begin{Shaded}
\begin{Highlighting}[numbers=left,,]
\CommentTok{# make df with means}
\NormalTok{observer <-}\StringTok{ }\KeywordTok{unique}\NormalTok{(ebdChkSummary}\OperatorTok{$}\NormalTok{observer)}

\CommentTok{# predict at 60 mins on the most common landcover}
\NormalTok{dfPredict <-}\StringTok{ }\NormalTok{ebdChkSummary }\OperatorTok
\StringTok{  }\KeywordTok{summarise_at}\NormalTok{(}\KeywordTok{vars}\NormalTok{(duration, decimalTime, julianDate), }\KeywordTok{list}\NormalTok{(}\OperatorTok{~}\StringTok{ }\KeywordTok{mean}\NormalTok{(.))) }\OperatorTok
\StringTok{  }\KeywordTok{mutate}\NormalTok{(}\DataTypeTok{duration =} \DecValTok{60}\NormalTok{, }\DataTypeTok{landcover =} \KeywordTok{as.factor}\NormalTok{(}\DecValTok{6}\NormalTok{)) }\OperatorTok
\StringTok{  }\NormalTok{tidyr}\OperatorTok{::}\KeywordTok{crossing}\NormalTok{(observer)}

\CommentTok{# run predict from model on it}
\NormalTok{dfPredict <-}\StringTok{ }\KeywordTok{mutate}\NormalTok{(dfPredict,}
  \DataTypeTok{score =} \KeywordTok{predict}\NormalTok{(modObsExp,}
    \DataTypeTok{newdata =}\NormalTok{ dfPredict,}
    \DataTypeTok{type =} \StringTok{"response"}\NormalTok{,}
    \DataTypeTok{allow.new.levels =} \OtherTok{TRUE}
\NormalTok{  )}
\NormalTok{) }\OperatorTok
\StringTok{  }\KeywordTok{mutate}\NormalTok{(}\DataTypeTok{score =}\NormalTok{ scales}\OperatorTok{::}\KeywordTok{rescale}\NormalTok{(score))}
\end{Highlighting}
\end{Shaded}

\begin{Shaded}
\begin{Highlighting}[numbers=left,,]
\KeywordTok{fwrite}\NormalTok{(dfPredict }\OperatorTok\StringTok{ }\NormalTok{dplyr}\OperatorTok{::}\KeywordTok{select}\NormalTok{(observer, score),}
  \DataTypeTok{file =} \StringTok{"data/03_data-obsExpertise-score.csv"}
\NormalTok{)}
\end{Highlighting}
\end{Shaded}

\hypertarget{examining-spatial-sampling-bias}{%
\section{Examining Spatial Sampling Bias}\label{examining-spatial-sampling-bias}}

The goal of this section is to determine how far each checklist location is from the nearest road, and how far each site is from its nearest neighbour. This involves finding the pairwise distance between a large number of unique checklist locations to a vast number of roads, as well as to each other.

Parts of this section are thus implemented in Python, using the R-Python interface package, \texttt{reticulate}.

\hypertarget{prepare-libraries-2}{%
\subsection{Prepare libraries}\label{prepare-libraries-2}}

\begin{Shaded}
\begin{Highlighting}[numbers=left,,]
\CommentTok{# load libraries}
\KeywordTok{library}\NormalTok{(reticulate)}
\KeywordTok{library}\NormalTok{(sf)}
\KeywordTok{library}\NormalTok{(dplyr)}
\KeywordTok{library}\NormalTok{(scales)}
\KeywordTok{library}\NormalTok{(readr)}
\KeywordTok{library}\NormalTok{(purrr)}

\KeywordTok{library}\NormalTok{(ggplot2)}
\KeywordTok{library}\NormalTok{(ggthemes)}
\KeywordTok{library}\NormalTok{(ggspatial)}
\KeywordTok{library}\NormalTok{(scico)}

\CommentTok{# round any function}
\NormalTok{round_any <-}\StringTok{ }\ControlFlowTok{function}\NormalTok{(x, }\DataTypeTok{accuracy =} \DecValTok{20000}\NormalTok{) \{}
  \KeywordTok{round}\NormalTok{(x }\OperatorTok{/}\StringTok{ }\NormalTok{accuracy) }\OperatorTok{*}\StringTok{ }\NormalTok{accuracy}
\NormalTok{\}}
\CommentTok{# ci function}
\NormalTok{ci <-}\StringTok{ }\ControlFlowTok{function}\NormalTok{(x) \{}
  \KeywordTok{qnorm}\NormalTok{(}\FloatTok{0.975}\NormalTok{) }\OperatorTok{*}\StringTok{ }\KeywordTok{sd}\NormalTok{(x, }\DataTypeTok{na.rm =} \OtherTok{TRUE}\NormalTok{) }\OperatorTok{/}\StringTok{ }\KeywordTok{sqrt}\NormalTok{(}\KeywordTok{length}\NormalTok{(x))}
\NormalTok{\}}

\CommentTok{# set python path}
\KeywordTok{use_python}\NormalTok{(}\StringTok{"/usr/bin/python3"}\NormalTok{)}
\end{Highlighting}
\end{Shaded}

Importing python libraries. These libraries need to be installed before use.

\begin{Shaded}
\begin{Highlighting}[numbers=left,,]
\CommentTok{# import classic python libs}
\ImportTok{import}\NormalTok{ itertools}
\ImportTok{from}\NormalTok{ operator }\ImportTok{import}\NormalTok{ itemgetter}
\ImportTok{import}\NormalTok{ numpy }\ImportTok{as}\NormalTok{ np}
\ImportTok{import}\NormalTok{ matplotlib.pyplot }\ImportTok{as}\NormalTok{ plt}
\ImportTok{import}\NormalTok{ math}

\CommentTok{# libs for dataframes}
\ImportTok{import}\NormalTok{ pandas }\ImportTok{as}\NormalTok{ pd}

\CommentTok{# import libs for geodata}
\ImportTok{from}\NormalTok{ shapely.ops }\ImportTok{import}\NormalTok{ nearest_points}
\ImportTok{import}\NormalTok{ geopandas }\ImportTok{as}\NormalTok{ gpd}
\ImportTok{import}\NormalTok{ rasterio}

\CommentTok{# import ckdtree}
\ImportTok{from}\NormalTok{ scipy.spatial }\ImportTok{import}\NormalTok{ cKDTree}
\ImportTok{from}\NormalTok{ shapely.geometry }\ImportTok{import}\NormalTok{ Point, MultiPoint, LineString, MultiLineString}
\end{Highlighting}
\end{Shaded}

\hypertarget{prepare-data-for-processing}{%
\subsection{Prepare data for processing}\label{prepare-data-for-processing}}

First we read in the roads shapefile, which is obtained from the Open Street Map database.
Then we read in the checklist covariates, and extract the unique coordinate pairs.
All data are reprojected to be in the UTM 43N coordinate system.

We define a custom Python function to separate multi-feature geometries (here, roads which are in parts) into single feature geometries.
Then we define a function to use the K-dimensional trees method from \texttt{scipy} to find the distance between two geometries, here, the distance between the locations and the nearest road.
We define another function to find the distance between checklist locations and all other checklist locations.

We use these functions to find the distance between each checklist location and the nearest road and the next nearest site.

\hypertarget{python-functions-and-distance-calculations}{%
\subsubsection{Python functions and distance calculations}\label{python-functions-and-distance-calculations}}

\begin{Shaded}
\begin{Highlighting}[numbers=left,,]
\CommentTok{# read in roads shapefile}
\NormalTok{roads }\OperatorTok{=}\NormalTok{ gpd.read_file(}\StringTok{"data/spatial/roads_studysite_2019/roads_studysite_2019.shp"}\NormalTok{)}
\NormalTok{roads.head()}

\CommentTok{# read in checklist covariates for conversion to gpd}
\CommentTok{# get unique coordinates, assign them to the df}
\CommentTok{# convert df to geo-df}
\NormalTok{chkCovars }\OperatorTok{=}\NormalTok{ pd.read_csv(}\StringTok{"data/03_data-covars-perChklist.csv"}\NormalTok{)}
\NormalTok{unique_locs }\OperatorTok{=}\NormalTok{ chkCovars.drop_duplicates(subset}\OperatorTok{=}\NormalTok{[}\StringTok{'longitude'}\NormalTok{,}
                                         \StringTok{'latitude'}\NormalTok{])[[}\StringTok{'longitude'}\NormalTok{, }\StringTok{'latitude'}\NormalTok{]]}
\NormalTok{unique_locs[}\StringTok{'coordId'}\NormalTok{] }\OperatorTok{=}\NormalTok{ np.arange(}\DecValTok{1}\NormalTok{, unique_locs.shape[}\DecValTok{0}\NormalTok{]}\OperatorTok{+}\DecValTok{1}\NormalTok{)}
\NormalTok{chkCovars }\OperatorTok{=}\NormalTok{ chkCovars.merge(unique_locs, on}\OperatorTok{=}\NormalTok{[}\StringTok{'longitude'}\NormalTok{, }\StringTok{'latitude'}\NormalTok{])}

\NormalTok{unique_locs }\OperatorTok{=}\NormalTok{ gpd.GeoDataFrame(}
\NormalTok{unique_locs, }
\NormalTok{geometry }\OperatorTok{=}\NormalTok{ gpd.points_from_xy(unique_locs.longitude, unique_locs.latitude))}
\NormalTok{unique_locs.crs }\OperatorTok{=}\NormalTok{ \{}\StringTok{'init'}\NormalTok{ :}\StringTok{'epsg:4326'}\NormalTok{\}}

\CommentTok{# reproject spatials to 43n epsg 32643}

\NormalTok{roads }\OperatorTok{=}\NormalTok{ roads.to_crs(\{}\StringTok{'init'}\NormalTok{: }\StringTok{'epsg:32643'}\NormalTok{\})}
\NormalTok{unique_locs }\OperatorTok{=}\NormalTok{ unique_locs.to_crs(\{}\StringTok{'init'}\NormalTok{: }\StringTok{'epsg:32643'}\NormalTok{\})}

\CommentTok{# function to simplify multilinestrings}
\KeywordTok{def}\NormalTok{ simplify_roads(complex_roads):}
\NormalTok{    simpleRoads }\OperatorTok{=}\NormalTok{ []}
    \ControlFlowTok{for}\NormalTok{ i }\KeywordTok{in} \BuiltInTok{range}\NormalTok{(}\BuiltInTok{len}\NormalTok{(complex_roads.geometry)):}
\NormalTok{        feature }\OperatorTok{=}\NormalTok{ complex_roads.geometry.iloc[i]}
        \ControlFlowTok{if}\NormalTok{ feature.geom_type }\OperatorTok{==} \StringTok{"LineString"}\NormalTok{:}
\NormalTok{            simpleRoads.append(feature)}
        \ControlFlowTok{elif}\NormalTok{ feature.geom_type }\OperatorTok{==} \StringTok{"MultiLineString"}\NormalTok{:}
            \ControlFlowTok{for}\NormalTok{ road_level2 }\KeywordTok{in}\NormalTok{ feature:}
\NormalTok{                simpleRoads.append(road_level2)}
    \ControlFlowTok{return}\NormalTok{ simpleRoads}

\CommentTok{# function to use ckdtrees to find the nearest road}
\KeywordTok{def}\NormalTok{ ckdnearest(gdfA, gdfB):}
\NormalTok{    A }\OperatorTok{=}\NormalTok{ np.concatenate(}
\NormalTok{    [np.array(geom.coords) }\ControlFlowTok{for}\NormalTok{ geom }\KeywordTok{in}\NormalTok{ gdfA.geometry.to_list()])}
\NormalTok{    simplified_features }\OperatorTok{=}\NormalTok{ simplify_roads(gdfB)}
\NormalTok{    B }\OperatorTok{=}\NormalTok{ [np.array(geom.coords) }\ControlFlowTok{for}\NormalTok{ geom }\KeywordTok{in}\NormalTok{ simplified_features]}
\NormalTok{    B }\OperatorTok{=}\NormalTok{ np.concatenate(B)}
\NormalTok{    ckd_tree }\OperatorTok{=}\NormalTok{ cKDTree(B)}
\NormalTok{    dist, idx }\OperatorTok{=}\NormalTok{ ckd_tree.query(A, k}\OperatorTok{=}\DecValTok{1}\NormalTok{)}
    \ControlFlowTok{return}\NormalTok{ dist}

\CommentTok{# function to use ckdtrees for nearest other checklist point}
\KeywordTok{def}\NormalTok{ ckdnearest_point(gdfA, gdfB):}
\NormalTok{    A }\OperatorTok{=}\NormalTok{ np.concatenate(}
\NormalTok{    [np.array(geom.coords) }\ControlFlowTok{for}\NormalTok{ geom }\KeywordTok{in}\NormalTok{ gdfA.geometry.to_list()])}
    \CommentTok{#simplified_features = simplify_roads(gdfB)}
\NormalTok{    B }\OperatorTok{=}\NormalTok{ np.concatenate(}
\NormalTok{    [np.array(geom.coords) }\ControlFlowTok{for}\NormalTok{ geom }\KeywordTok{in}\NormalTok{ gdfB.geometry.to_list()])}
    \CommentTok{#B = np.concatenate(B)}
\NormalTok{    ckd_tree }\OperatorTok{=}\NormalTok{ cKDTree(B)}
\NormalTok{    dist, idx }\OperatorTok{=}\NormalTok{ ckd_tree.query(A, k}\OperatorTok{=}\NormalTok{[}\DecValTok{2}\NormalTok{])}
    \ControlFlowTok{return}\NormalTok{ dist}

\CommentTok{# get distance to nearest road}
\NormalTok{unique_locs[}\StringTok{'dist_road'}\NormalTok{] }\OperatorTok{=}\NormalTok{ ckdnearest(unique_locs, roads)}

\CommentTok{# get distance to nearest other site}
\NormalTok{unique_locs[}\StringTok{'nnb'}\NormalTok{] }\OperatorTok{=}\NormalTok{ ckdnearest_point(unique_locs, unique_locs)}

\CommentTok{# write to file}
\NormalTok{unique_locs }\OperatorTok{=}\NormalTok{ pd.DataFrame(unique_locs.drop(columns}\OperatorTok{=}\StringTok{'geometry'}\NormalTok{))}
\NormalTok{unique_locs[}\StringTok{'dist_road'}\NormalTok{] }\OperatorTok{=}\NormalTok{ unique_locs[}\StringTok{'dist_road'}\NormalTok{]}
\NormalTok{unique_locs[}\StringTok{'nnb'}\NormalTok{] }\OperatorTok{=}\NormalTok{ unique_locs[}\StringTok{'nnb'}\NormalTok{]}
\NormalTok{unique_locs.to_csv(path_or_buf }\OperatorTok{=} \StringTok{"data/locs_dist_to_road.csv"}\NormalTok{, index}\OperatorTok{=}\VariableTok{False}\NormalTok{)}

\CommentTok{# merge unique locs with chkCovars}
\NormalTok{chkCovars }\OperatorTok{=}\NormalTok{ chkCovars.merge(unique_locs, on}\OperatorTok{=}\NormalTok{[}\StringTok{'latitude'}\NormalTok{, }
                                             \StringTok{'longitude'}\NormalTok{, }\StringTok{'coordId'}\NormalTok{])}
\end{Highlighting}
\end{Shaded}

\hypertarget{spatially-explicit-filter-on-checklists-1}{%
\subsubsection{Spatially explicit filter on checklists}\label{spatially-explicit-filter-on-checklists-1}}

We filter the checklists by the boundary of the study area. This is \emph{not} the extent.

\begin{Shaded}
\begin{Highlighting}[numbers=left,,]
\CommentTok{# extract data from python}
\NormalTok{chkCovars <-}\StringTok{ }\NormalTok{py}\OperatorTok{$}\NormalTok{chkCovars}
\NormalTok{chkCovars <-}\StringTok{ }\KeywordTok{st_as_sf}\NormalTok{(chkCovars, }\DataTypeTok{coords =} \KeywordTok{c}\NormalTok{(}\StringTok{"longitude"}\NormalTok{, }\StringTok{"latitude"}\NormalTok{)) }\OperatorTok
\StringTok{  `}\DataTypeTok{st_crs<-}\StringTok{`}\NormalTok{(}\DecValTok{4326}\NormalTok{) }\OperatorTok
\StringTok{  }\KeywordTok{st_transform}\NormalTok{(}\DecValTok{32643}\NormalTok{)}

\CommentTok{# read wg}
\NormalTok{wg <-}\StringTok{ }\KeywordTok{st_read}\NormalTok{(}\StringTok{"data/spatial/hillsShapefile/Nil_Ana_Pal.shp"}\NormalTok{) }\OperatorTok
\StringTok{  }\KeywordTok{st_transform}\NormalTok{(}\DecValTok{32643}\NormalTok{)}

\CommentTok{# spatial subset}
\NormalTok{chkCovars <-}\StringTok{ }\NormalTok{chkCovars }\OperatorTok
\StringTok{  }\KeywordTok{mutate}\NormalTok{(}\DataTypeTok{id =} \DecValTok{1}\OperatorTok{:}\KeywordTok{nrow}\NormalTok{(.)) }\OperatorTok
\StringTok{  }\KeywordTok{filter}\NormalTok{(id }\OperatorTok\StringTok{ }\KeywordTok{unlist}\NormalTok{(}\KeywordTok{st_contains}\NormalTok{(wg, chkCovars)))}
\end{Highlighting}
\end{Shaded}

\hypertarget{main-text-figure-3}{%
\subsection{Main Text Figure 3}\label{main-text-figure-3}}

\hypertarget{prepare-histogram-of-distance-to-roads}{%
\subsubsection{Prepare histogram of distance to roads}\label{prepare-histogram-of-distance-to-roads}}

Figure code is hidden in versions rendered as HTML or PDF.

\hypertarget{table-distance-to-roads}{%
\subsubsection{Table: Distance to roads}\label{table-distance-to-roads}}

\begin{Shaded}
\begin{Highlighting}[numbers=left,,]
\CommentTok{# write the mean and ci95 to file}
\NormalTok{chkCovars }\OperatorTok
\StringTok{  }\KeywordTok{st_drop_geometry}\NormalTok{() }\OperatorTok
\StringTok{  }\KeywordTok{select}\NormalTok{(dist_road, nnb) }\OperatorTok
\StringTok{  }\NormalTok{tidyr}\OperatorTok{::}\KeywordTok{pivot_longer}\NormalTok{(}
    \DataTypeTok{cols =} \KeywordTok{c}\NormalTok{(}\StringTok{"dist_road"}\NormalTok{, }\StringTok{"nnb"}\NormalTok{),}
    \DataTypeTok{names_to =} \StringTok{"variable"}
\NormalTok{  ) }\OperatorTok
\StringTok{  }\KeywordTok{group_by}\NormalTok{(variable) }\OperatorTok
\StringTok{  }\KeywordTok{summarise_at}\NormalTok{(}
    \KeywordTok{vars}\NormalTok{(value),}
    \KeywordTok{list}\NormalTok{(}\OperatorTok{~}\StringTok{ }\KeywordTok{mean}\NormalTok{(.), }\OperatorTok{~}\StringTok{ }\KeywordTok{sd}\NormalTok{(.), }\OperatorTok{~}\StringTok{ }\KeywordTok{min}\NormalTok{(.), }\OperatorTok{~}\StringTok{ }\KeywordTok{max}\NormalTok{(.))}
\NormalTok{  ) }\OperatorTok
\StringTok{  }\KeywordTok{write_csv}\NormalTok{(}\StringTok{"data/results/distance_roads_sites.csv"}\NormalTok{)}
\end{Highlighting}
\end{Shaded}

\hypertarget{distance-to-nearest-neighbouring-site}{%
\subsection{Distance to nearest neighbouring site}\label{distance-to-nearest-neighbouring-site}}

\begin{Shaded}
\begin{Highlighting}[numbers=left,,]
\CommentTok{# get unique locations}
\NormalTok{locs <-}\StringTok{ }\NormalTok{py}\OperatorTok{$}\NormalTok{unique_locs}
\end{Highlighting}
\end{Shaded}

Figure code is hidden in versions rendered as HTML and PDF.

\hypertarget{main-text-figure-3-1}{%
\subsection{Main Text Figure 3}\label{main-text-figure-3-1}}

\begin{Shaded}
\begin{Highlighting}[numbers=left,,]
\NormalTok{roads <-}\StringTok{ }\KeywordTok{st_read}\NormalTok{(}\StringTok{"data/spatial/roads_studysite_2019/roads_studysite_2019.shp"}\NormalTok{) }\OperatorTok
\StringTok{  }\KeywordTok{st_transform}\NormalTok{(}\DecValTok{32643}\NormalTok{)}
\NormalTok{points <-}\StringTok{ }\NormalTok{chkCovars }\OperatorTok
\StringTok{  }\KeywordTok{bind_cols}\NormalTok{(}\KeywordTok{as_tibble}\NormalTok{(}\KeywordTok{st_coordinates}\NormalTok{(.))) }\OperatorTok
\StringTok{  }\KeywordTok{st_drop_geometry}\NormalTok{() }\OperatorTok
\StringTok{  }\KeywordTok{mutate}\NormalTok{(}\DataTypeTok{X =} \KeywordTok{round_any}\NormalTok{(X, }\DecValTok{2500}\NormalTok{), }\DataTypeTok{Y =} \KeywordTok{round_any}\NormalTok{(Y, }\DecValTok{2500}\NormalTok{))}

\NormalTok{points <-}\StringTok{ }\KeywordTok{count}\NormalTok{(points, X, Y)}

\CommentTok{# add land}
\KeywordTok{library}\NormalTok{(rnaturalearth)}
\NormalTok{land <-}\StringTok{ }\KeywordTok{ne_countries}\NormalTok{(}
  \DataTypeTok{scale =} \DecValTok{50}\NormalTok{, }\DataTypeTok{type =} \StringTok{"countries"}\NormalTok{, }\DataTypeTok{continent =} \StringTok{"asia"}\NormalTok{,}
  \DataTypeTok{country =} \StringTok{"india"}\NormalTok{,}
  \DataTypeTok{returnclass =} \KeywordTok{c}\NormalTok{(}\StringTok{"sf"}\NormalTok{)}
\NormalTok{) }\OperatorTok
\StringTok{  }\KeywordTok{st_transform}\NormalTok{(}\DecValTok{32643}\NormalTok{)}

\NormalTok{bbox <-}\StringTok{ }\KeywordTok{st_bbox}\NormalTok{(wg)}
\end{Highlighting}
\end{Shaded}

Figure code is hidden in versions rendered as HTML and PDF.

\begin{figure}
\centering
\includegraphics{figs/fig_03_distance_to_roads.png}
\caption{Spatial sampling bias of eBird observations across the Nilgiri and the Anamalai hills. A large proportion of localities/sites were next to roads and were on average only \textasciitilde{}300m from another locality/site. Each cell here is 2.5km x 2.5km}
\end{figure}

\hypertarget{adding-covariates-to-checklist-data}{%
\section{Adding Covariates to Checklist Data}\label{adding-covariates-to-checklist-data}}

In this section, we prepare a final list of covariates, after taking into account spatial sampling bias (examined in the previous section), temporal bias and observer expertise scores.

\hypertarget{prepare-libraries-and-data}{%
\subsection{Prepare libraries and data}\label{prepare-libraries-and-data}}

\begin{Shaded}
\begin{Highlighting}[numbers=left,,]

\CommentTok{# load libs}
\KeywordTok{library}\NormalTok{(dplyr)}
\KeywordTok{library}\NormalTok{(readr)}
\KeywordTok{library}\NormalTok{(stringr)}
\KeywordTok{library}\NormalTok{(purrr)}
\KeywordTok{library}\NormalTok{(raster)}
\KeywordTok{library}\NormalTok{(glue)}
\KeywordTok{library}\NormalTok{(velox)}
\KeywordTok{library}\NormalTok{(tidyr)}
\KeywordTok{library}\NormalTok{(sf)}

\CommentTok{# load saved data object}
\KeywordTok{load}\NormalTok{(}\StringTok{"data/01_ebird_data_prelim_processing.rdata"}\NormalTok{)}
\end{Highlighting}
\end{Shaded}

\hypertarget{spatial-subsampling}{%
\subsection{Spatial subsampling}\label{spatial-subsampling}}

Sampling bias can be introduced into citizen science due to the often ad-hoc nature of data collection (Sullivan et al. \protect\hyperlink{ref-sullivan2014}{2014}). For eBird, this translates into checklists reported when convenient, rather than at regular or random points in time and space, leading to non-independence in the data if observations are spatio-temporally clustered (Johnston et al. \protect\hyperlink{ref-johnston2019a}{2019}). Spatio-temporal autocorrelation in the data can be reduced by sub-sampling at an appropriate spatial resolution, and by avoiding temporal clustering. We estimated two simple measures of spatial clustering: the distance from each site to the nearest road (road data from OpenStreetMap; (OpenStreetMap contributors \protect\hyperlink{ref-OpenStreetMap}{2019})), and the nearest-neighbor distance for each site. Sites were strongly tied to roads (mean distance to road ± SD = 390.77 ± 859.15 m; range = 0.28 m -- 7.64 km) and were on average only 297 m away from another site (SD = 553 m; range = 0.14 m -- 12.85 km) (Figure 3). This analysis was done in the previous section.

Here, to further reduce spatial autocorrelation, we divided the study area into a grid of 1km wide square cells and picked checklists from one site at random within each grid cell.

Prior to running this analysis, we checked how many checklists/data would be retained given a particular value of distance to account for spatial independence. This analysis can be accessed in Section 8 of the Supplementary Material. We show that over 80\% of checklists are retained with a distance cutoff of 1km. In addition, a number of thinning approaches were tested to determine which method retained the highest proportion of points, while accounting for sampling effort (time and distance). This analysis can be accessed in Section 9 of the Supplementary Material.

\begin{Shaded}
\begin{Highlighting}[numbers=left,,]
\CommentTok{# grid based spatial thinning}
\NormalTok{gridsize <-}\StringTok{ }\DecValTok{1000} \CommentTok{# grid size in metres}
\NormalTok{effort_distance_max <-}\StringTok{ }\DecValTok{1000} \CommentTok{# removing checklists with this distance}

\CommentTok{# make grids across the study site}
\NormalTok{hills <-}\StringTok{ }\KeywordTok{st_read}\NormalTok{(}\StringTok{"data/spatial/hillsShapefile/Nil_Ana_Pal.shp"}\NormalTok{) }\OperatorTok
\StringTok{  }\KeywordTok{st_transform}\NormalTok{(}\DecValTok{32643}\NormalTok{)}
\NormalTok{grid <-}\StringTok{ }\KeywordTok{st_make_grid}\NormalTok{(hills, }\DataTypeTok{cellsize =}\NormalTok{ gridsize)}

\CommentTok{# split data by species}
\NormalTok{data_spatial_thin <-}\StringTok{ }\KeywordTok{split}\NormalTok{(}\DataTypeTok{x =}\NormalTok{ dataGrouped, }\DataTypeTok{f =}\NormalTok{ dataGrouped}\OperatorTok{$}\NormalTok{scientific_name)}

\CommentTok{# spatial thinning on each species retains}
\CommentTok{# site with maximum visits per grid cell}
\NormalTok{data_spatial_thin <-}\StringTok{ }\KeywordTok{map}\NormalTok{(data_spatial_thin, }\ControlFlowTok{function}\NormalTok{(df) \{}

  \CommentTok{# count visits per locality}
\NormalTok{  df <-}\StringTok{ }\KeywordTok{group_by}\NormalTok{(df, locality) }\OperatorTok
\StringTok{    }\KeywordTok{mutate}\NormalTok{(}\DataTypeTok{tot_effort =} \KeywordTok{length}\NormalTok{(sampling_event_identifier)) }\OperatorTok
\StringTok{    }\KeywordTok{ungroup}\NormalTok{()}

  \CommentTok{# remove sites with distances above spatial independence}
\NormalTok{  df <-}\StringTok{ }\NormalTok{df }\OperatorTok
\StringTok{    }\KeywordTok{filter}\NormalTok{(effort_distance_km }\OperatorTok{<=}\StringTok{ }\NormalTok{effort_distance_max) }\OperatorTok
\StringTok{    }\KeywordTok{st_as_sf}\NormalTok{(}\DataTypeTok{coords =} \KeywordTok{c}\NormalTok{(}\StringTok{"longitude"}\NormalTok{, }\StringTok{"latitude"}\NormalTok{)) }\OperatorTok
\StringTok{    `}\DataTypeTok{st_crs<-}\StringTok{`}\NormalTok{(}\DecValTok{4326}\NormalTok{) }\OperatorTok
\StringTok{    }\KeywordTok{st_transform}\NormalTok{(}\DecValTok{32643}\NormalTok{) }\OperatorTok
\StringTok{    }\KeywordTok{mutate}\NormalTok{(}\DataTypeTok{coordId =} \DecValTok{1}\OperatorTok{:}\KeywordTok{nrow}\NormalTok{(.)) }\OperatorTok
\StringTok{    }\KeywordTok{bind_cols}\NormalTok{(}\KeywordTok{as_tibble}\NormalTok{(}\KeywordTok{st_coordinates}\NormalTok{(.)))}

  \CommentTok{# whcih cell has which coords}
\NormalTok{  grid_contents <-}\StringTok{ }\KeywordTok{st_contains}\NormalTok{(grid, df) }\OperatorTok
\StringTok{    }\KeywordTok{as_tibble}\NormalTok{() }\OperatorTok
\StringTok{    }\KeywordTok{rename}\NormalTok{(}\DataTypeTok{cell =}\NormalTok{ row.id, }\DataTypeTok{coordId =}\NormalTok{ col.id)}

  \CommentTok{# what's the max point in each grid}
\NormalTok{  points_max <-}\StringTok{ }\KeywordTok{left_join}\NormalTok{(df }\OperatorTok\StringTok{ }\KeywordTok{st_drop_geometry}\NormalTok{(),}
\NormalTok{    grid_contents,}
    \DataTypeTok{by =} \StringTok{"coordId"}
\NormalTok{  ) }\OperatorTok
\StringTok{    }\KeywordTok{group_by}\NormalTok{(cell) }\OperatorTok
\StringTok{    }\KeywordTok{filter}\NormalTok{(tot_effort }\OperatorTok{==}\StringTok{ }\KeywordTok{max}\NormalTok{(tot_effort))}

  \KeywordTok{return}\NormalTok{(points_max)}
\NormalTok{\})}

\CommentTok{# remove old data}
\KeywordTok{rm}\NormalTok{(dataGrouped)}
\end{Highlighting}
\end{Shaded}

\hypertarget{temporal-subsampling}{%
\subsection{Temporal subsampling}\label{temporal-subsampling}}

Additionally, from each selected site, we randomly selected a maximum of 10 checklists, which reduced temporal autocorrelation.

\begin{Shaded}
\begin{Highlighting}[numbers=left,,]
\CommentTok{# subsample data for random 10 observations}
\NormalTok{dataSubsample <-}\StringTok{ }\KeywordTok{map}\NormalTok{(data_spatial_thin, }\ControlFlowTok{function}\NormalTok{(df) \{}
\NormalTok{  df <-}\StringTok{ }\KeywordTok{ungroup}\NormalTok{(df)}
\NormalTok{  df_to_locality <-}\StringTok{ }\KeywordTok{split}\NormalTok{(}\DataTypeTok{x =}\NormalTok{ df, }\DataTypeTok{f =}\NormalTok{ df}\OperatorTok{$}\NormalTok{locality)}
\NormalTok{  df_samples <-}\StringTok{ }\KeywordTok{map_if}\NormalTok{(}
    \DataTypeTok{.x =}\NormalTok{ df_to_locality,}
    \DataTypeTok{.p =} \ControlFlowTok{function}\NormalTok{(x) \{}
      \KeywordTok{nrow}\NormalTok{(x) }\OperatorTok{>}\StringTok{ }\DecValTok{10}
\NormalTok{    \},}
    \DataTypeTok{.f =} \ControlFlowTok{function}\NormalTok{(x) }\KeywordTok{sample_n}\NormalTok{(x, }\DecValTok{10}\NormalTok{, }\DataTypeTok{replace =} \OtherTok{FALSE}\NormalTok{)}
\NormalTok{  )}

  \KeywordTok{return}\NormalTok{(}\KeywordTok{bind_rows}\NormalTok{(df_samples))}
\NormalTok{\})}

\CommentTok{# bind all rows for data frame}
\NormalTok{dataSubsample <-}\StringTok{ }\KeywordTok{bind_rows}\NormalTok{(dataSubsample)}

\CommentTok{# remove previous data}
\KeywordTok{rm}\NormalTok{(data_spatial_thin)}
\end{Highlighting}
\end{Shaded}

\hypertarget{add-checklist-calibration-index}{%
\subsection{Add checklist calibration index}\label{add-checklist-calibration-index}}

Load the CCI computed in the previous section. The CCI was the lone observer's expertise score for single-observer checklists, and the highest expertise score among observers for group checklists.

\begin{Shaded}
\begin{Highlighting}[numbers=left,,]
\CommentTok{# read in obs score and extract numbers}
\NormalTok{expertiseScore <-}\StringTok{ }\KeywordTok{read_csv}\NormalTok{(}\StringTok{"data/03_data-obsExpertise-score.csv"}\NormalTok{) }\OperatorTok
\StringTok{  }\KeywordTok{mutate}\NormalTok{(}\DataTypeTok{numObserver =} \KeywordTok{str_extract}\NormalTok{(observer, }\StringTok{"}\CharTok{\textbackslash{}\textbackslash{}}\StringTok{d+"}\NormalTok{)) }\OperatorTok
\StringTok{  }\NormalTok{dplyr}\OperatorTok{::}\KeywordTok{select}\NormalTok{(}\OperatorTok{-}\NormalTok{observer)}

\CommentTok{# group seis consist of multiple observers}
\CommentTok{# in this case, seis need to have the highest expertise observer score}
\CommentTok{# as the associated covariate}

\CommentTok{# get unique observers per sei}
\NormalTok{dataSeiScore <-}\StringTok{ }\KeywordTok{distinct}\NormalTok{(}
\NormalTok{  dataSubsample, sampling_event_identifier,}
\NormalTok{  observer_id}
\NormalTok{) }\OperatorTok
\StringTok{  }\CommentTok{# make list column of observers}
\StringTok{  }\KeywordTok{mutate}\NormalTok{(}\DataTypeTok{observers =} \KeywordTok{str_split}\NormalTok{(observer_id, }\StringTok{","}\NormalTok{)) }\OperatorTok
\StringTok{  }\KeywordTok{unnest}\NormalTok{(}\DataTypeTok{cols =} \KeywordTok{c}\NormalTok{(observers)) }\OperatorTok
\StringTok{  }\CommentTok{# add numeric observer id}
\StringTok{  }\KeywordTok{mutate}\NormalTok{(}\DataTypeTok{numObserver =} \KeywordTok{str_extract}\NormalTok{(observers, }\StringTok{"}\CharTok{\textbackslash{}\textbackslash{}}\StringTok{d+"}\NormalTok{)) }\OperatorTok
\StringTok{  }\CommentTok{# now get distinct sei and observer id numeric}
\StringTok{  }\KeywordTok{distinct}\NormalTok{(sampling_event_identifier, numObserver)}

\CommentTok{# now add expertise score to sei}
\NormalTok{dataSeiScore <-}\StringTok{ }\KeywordTok{left_join}\NormalTok{(dataSeiScore, expertiseScore,}
  \DataTypeTok{by =} \StringTok{"numObserver"}
\NormalTok{) }\OperatorTok
\StringTok{  }\CommentTok{# get max expertise score per sei}
\StringTok{  }\KeywordTok{group_by}\NormalTok{(sampling_event_identifier) }\OperatorTok
\StringTok{  }\KeywordTok{summarise}\NormalTok{(}\DataTypeTok{expertise =} \KeywordTok{max}\NormalTok{(score))}

\CommentTok{# add to dataCovar}
\NormalTok{dataSubsample <-}\StringTok{ }\KeywordTok{left_join}\NormalTok{(dataSubsample, dataSeiScore,}
  \DataTypeTok{by =} \StringTok{"sampling_event_identifier"}
\NormalTok{)}

\CommentTok{# remove data without expertise score}
\NormalTok{dataSubsample <-}\StringTok{ }\KeywordTok{filter}\NormalTok{(dataSubsample, }\OperatorTok{!}\KeywordTok{is.na}\NormalTok{(expertise))}
\end{Highlighting}
\end{Shaded}

\hypertarget{add-climatic-and-landscape-covariates}{%
\subsection{Add climatic and landscape covariates}\label{add-climatic-and-landscape-covariates}}

Reload climate and land cover predictors prepared previously.

\begin{Shaded}
\begin{Highlighting}[numbers=left,,]

\CommentTok{# list landscape covariate stacks}
\NormalTok{landscape_files <-}\StringTok{ "data/spatial/landscape_resamp01_km.tif"}

\CommentTok{# read in as stacks}
\NormalTok{landscape_data <-}\StringTok{ }\KeywordTok{stack}\NormalTok{(landscape_files)}

\CommentTok{# get proper names}
\NormalTok{elev_names <-}\StringTok{ }\KeywordTok{c}\NormalTok{(}\StringTok{"elev"}\NormalTok{, }\StringTok{"slope"}\NormalTok{, }\StringTok{"aspect"}\NormalTok{)}
\NormalTok{chelsa_names <-}\StringTok{ }\KeywordTok{c}\NormalTok{(}\StringTok{"bio1"}\NormalTok{, }\StringTok{"bio12"}\NormalTok{)}

\KeywordTok{names}\NormalTok{(landscape_data) <-}\StringTok{ }\KeywordTok{as.character}\NormalTok{(}\KeywordTok{glue}\NormalTok{(}\StringTok{'\{c(elev_names, chelsa_names, "landcover")\}'}\NormalTok{))}
\end{Highlighting}
\end{Shaded}

\hypertarget{spatial-buffers-around-selected-checklists}{%
\subsection{Spatial buffers around selected checklists}\label{spatial-buffers-around-selected-checklists}}

Every checklist on eBird is associated with a latitude and longitude. However, the coordinates entered by an observer may not accurately depict the location at which a species was detected. This can occur for two reasons: first, traveling checklists are often associated with a single location along the route travelled by observers; and second, checklist locations could be assigned to a `hotspot' -- a location that is marked by eBird as being frequented by multiple observers. In many cases, an observation might be assigned to a hotspot even though the observation was not made at the precise location of the hotspot (J. \protect\hyperlink{ref-praveenj.2017}{2017}). Johnston et al., (2019) showed that a large proportion of observations occurred within a 3km grid, even for those checklists up to 5km in length. Hence to adjust for spatial precision, we considered a minimum radius of 2.5km around each unique locality when sampling environmental covariate values.

\begin{Shaded}
\begin{Highlighting}[numbers=left,,]
\CommentTok{# assign neighbourhood radius in m}
\NormalTok{sample_radius <-}\StringTok{ }\FloatTok{2.5} \OperatorTok{*}\StringTok{ }\FloatTok{1e3}

\CommentTok{# get distinct points and make buffer}
\NormalTok{ebird_buff <-}\StringTok{ }\NormalTok{dataSubsample }\OperatorTok
\StringTok{  }\KeywordTok{ungroup}\NormalTok{() }\OperatorTok
\StringTok{  }\KeywordTok{distinct}\NormalTok{(X, Y) }\OperatorTok
\StringTok{  }\KeywordTok{mutate}\NormalTok{(}\DataTypeTok{id =} \DecValTok{1}\OperatorTok{:}\KeywordTok{nrow}\NormalTok{(.)) }\OperatorTok
\StringTok{  }\KeywordTok{crossing}\NormalTok{(sample_radius) }\OperatorTok
\StringTok{  }\KeywordTok{arrange}\NormalTok{(id) }\OperatorTok
\StringTok{  }\KeywordTok{group_by}\NormalTok{(sample_radius) }\OperatorTok
\StringTok{  }\KeywordTok{nest}\NormalTok{() }\OperatorTok
\StringTok{  }\KeywordTok{ungroup}\NormalTok{()}


\CommentTok{# convert to spatial features}
\NormalTok{ebird_buff <-}\StringTok{ }\KeywordTok{mutate}\NormalTok{(ebird_buff,}
  \DataTypeTok{data =} \KeywordTok{map2}\NormalTok{(}
\NormalTok{    data, sample_radius,}
    \ControlFlowTok{function}\NormalTok{(df, rd) \{}
\NormalTok{      df_sf <-}\StringTok{ }\KeywordTok{st_as_sf}\NormalTok{(df, }\DataTypeTok{coords =} \KeywordTok{c}\NormalTok{(}\StringTok{"X"}\NormalTok{, }\StringTok{"Y"}\NormalTok{), }\DataTypeTok{crs =} \DecValTok{32643}\NormalTok{) }\OperatorTok
\StringTok{        }\CommentTok{# add long lat}
\StringTok{        }\KeywordTok{bind_cols}\NormalTok{(}\KeywordTok{as_tibble}\NormalTok{(}\KeywordTok{st_coordinates}\NormalTok{(.))) }\OperatorTok
\StringTok{        }\CommentTok{# rename(longitude = X, latitude = Y) %>%}
\StringTok{        }\CommentTok{# # transform to modis projection}
\StringTok{        }\CommentTok{# st_transform(crs = 32643) %>%}
\StringTok{        }\CommentTok{# buffer to create neighborhood around each point}
\StringTok{        }\KeywordTok{st_buffer}\NormalTok{(}\DataTypeTok{dist =}\NormalTok{ rd)}
\NormalTok{    \}}
\NormalTok{  )}
\NormalTok{)}
\end{Highlighting}
\end{Shaded}

\hypertarget{spatial-buffer-wide-covariates}{%
\subsection{Spatial buffer-wide covariates}\label{spatial-buffer-wide-covariates}}

\hypertarget{mean-climatic-covariates}{%
\subsubsection{Mean climatic covariates}\label{mean-climatic-covariates}}

All climatic covariates are sampled by considering the mean values within a 2.5km radius as discussed above and prefixed ``am\_''.

\begin{Shaded}
\begin{Highlighting}[numbers=left,,]
\CommentTok{# get area mean for all preds except landcover, which is the last one}
\NormalTok{env_area_mean <-}\StringTok{ }\NormalTok{purrr}\OperatorTok{::}\KeywordTok{map}\NormalTok{(ebird_buff}\OperatorTok{$}\NormalTok{data, }\ControlFlowTok{function}\NormalTok{(df) \{}
\NormalTok{  stk <-}\StringTok{ }\NormalTok{landscape_data[[}\OperatorTok{-}\KeywordTok{dim}\NormalTok{(landscape_data)[}\DecValTok{3}\NormalTok{]]] }\CommentTok{# removing landcover here}
\NormalTok{  velstk <-}\StringTok{ }\KeywordTok{velox}\NormalTok{(stk)}
\NormalTok{  dextr <-}\StringTok{ }\NormalTok{velstk}\OperatorTok{$}\KeywordTok{extract}\NormalTok{(}
    \DataTypeTok{sp =}\NormalTok{ df, }\DataTypeTok{df =} \OtherTok{TRUE}\NormalTok{,}
    \DataTypeTok{fun =} \ControlFlowTok{function}\NormalTok{(x) }\KeywordTok{mean}\NormalTok{(x, }\DataTypeTok{na.rm =}\NormalTok{ T)}
\NormalTok{  )}

  \CommentTok{# assign names for joining}
  \KeywordTok{names}\NormalTok{(dextr) <-}\StringTok{ }\KeywordTok{c}\NormalTok{(}\StringTok{"id"}\NormalTok{, }\KeywordTok{names}\NormalTok{(stk))}
  \KeywordTok{return}\NormalTok{(}\KeywordTok{as_tibble}\NormalTok{(dextr))}
\NormalTok{\})}

\CommentTok{# join to buffer data}
\NormalTok{ebird_buff <-}\StringTok{ }\NormalTok{ebird_buff }\OperatorTok
\StringTok{  }\KeywordTok{mutate}\NormalTok{(}\DataTypeTok{data =} \KeywordTok{map2}\NormalTok{(data, env_area_mean, inner_join, }\DataTypeTok{by =} \StringTok{"id"}\NormalTok{))}
\end{Highlighting}
\end{Shaded}

\hypertarget{proportions-of-land-cover-type}{%
\subsubsection{Proportions of land cover type}\label{proportions-of-land-cover-type}}

All land cover covariates were sampled by considering the proportion of each land cover type within a 2.5km radius.

\begin{Shaded}
\begin{Highlighting}[numbers=left,,]
\CommentTok{# get the last element of each stack from the list}
\CommentTok{# this is the landcover at that resolution}
\NormalTok{lc_area_prop <-}\StringTok{ }\NormalTok{purrr}\OperatorTok{::}\KeywordTok{map}\NormalTok{(ebird_buff}\OperatorTok{$}\NormalTok{data, }\ControlFlowTok{function}\NormalTok{(df) \{}
\NormalTok{  lc <-}\StringTok{ }\NormalTok{landscape_data[[}\KeywordTok{dim}\NormalTok{(landscape_data)[}\DecValTok{3}\NormalTok{]]] }\CommentTok{# accessing landcover here}
\NormalTok{  lc_velox <-}\StringTok{ }\KeywordTok{velox}\NormalTok{(lc)}
\NormalTok{  lc_vals <-}\StringTok{ }\NormalTok{lc_velox}\OperatorTok{$}\KeywordTok{extract}\NormalTok{(}\DataTypeTok{sp =}\NormalTok{ df, }\DataTypeTok{df =} \OtherTok{TRUE}\NormalTok{)}
  \KeywordTok{names}\NormalTok{(lc_vals) <-}\StringTok{ }\KeywordTok{c}\NormalTok{(}\StringTok{"id"}\NormalTok{, }\StringTok{"lc"}\NormalTok{)}

  \CommentTok{# get landcover proportions}
\NormalTok{  lc_prop <-}\StringTok{ }\KeywordTok{count}\NormalTok{(lc_vals, id, lc) }\OperatorTok
\StringTok{    }\KeywordTok{group_by}\NormalTok{(id) }\OperatorTok
\StringTok{    }\KeywordTok{mutate}\NormalTok{(}
      \DataTypeTok{lc =} \KeywordTok{glue}\NormalTok{(}\StringTok{'lc_\{str_pad(lc, 2, pad = "0")\}'}\NormalTok{),}
      \DataTypeTok{prop =}\NormalTok{ n }\OperatorTok{/}\StringTok{ }\KeywordTok{sum}\NormalTok{(n)}
\NormalTok{    ) }\OperatorTok
\StringTok{    }\NormalTok{dplyr}\OperatorTok{::}\KeywordTok{select}\NormalTok{(}\OperatorTok{-}\NormalTok{n) }\OperatorTok
\StringTok{    }\NormalTok{tidyr}\OperatorTok{::}\KeywordTok{pivot_wider}\NormalTok{(}
      \DataTypeTok{names_from =}\NormalTok{ lc,}
      \DataTypeTok{values_from =}\NormalTok{ prop,}
      \DataTypeTok{values_fill =} \KeywordTok{list}\NormalTok{(}\DataTypeTok{prop =} \DecValTok{0}\NormalTok{)}
\NormalTok{    ) }\OperatorTok
\StringTok{    }\KeywordTok{ungroup}\NormalTok{()}

  \KeywordTok{return}\NormalTok{(lc_prop)}
\NormalTok{\})}

\CommentTok{# join to data}
\NormalTok{ebird_buff <-}\StringTok{ }\NormalTok{ebird_buff }\OperatorTok
\StringTok{  }\KeywordTok{mutate}\NormalTok{(}\DataTypeTok{data =} \KeywordTok{map2}\NormalTok{(data, lc_area_prop, inner_join, }\DataTypeTok{by =} \StringTok{"id"}\NormalTok{))}
\end{Highlighting}
\end{Shaded}

\hypertarget{link-environmental-covariates-to-checklists}{%
\subsubsection{Link environmental covariates to checklists}\label{link-environmental-covariates-to-checklists}}

\begin{Shaded}
\begin{Highlighting}[numbers=left,,]
\CommentTok{# duplicate scale data}
\NormalTok{data_at_scale <-}\StringTok{ }\NormalTok{ebird_buff}

\CommentTok{# join the full data to landscape samples at each scale}
\NormalTok{data_at_scale}\OperatorTok{$}\NormalTok{data <-}\StringTok{ }\KeywordTok{map}\NormalTok{(data_at_scale}\OperatorTok{$}\NormalTok{data, }\ControlFlowTok{function}\NormalTok{(df) \{}
\NormalTok{  df <-}\StringTok{ }\KeywordTok{st_drop_geometry}\NormalTok{(df)}
\NormalTok{  df <-}\StringTok{ }\KeywordTok{inner_join}\NormalTok{(dataSubsample, df, }\DataTypeTok{by =} \KeywordTok{c}\NormalTok{(}\StringTok{"X"}\NormalTok{, }\StringTok{"Y"}\NormalTok{))}
  \KeywordTok{return}\NormalTok{(df)}
\NormalTok{\})}
\end{Highlighting}
\end{Shaded}

Save data to file.

\begin{Shaded}
\begin{Highlighting}[numbers=left,,]
\CommentTok{# write to file}
\KeywordTok{pmap}\NormalTok{(data_at_scale, }\ControlFlowTok{function}\NormalTok{(sample_radius, data) \{}
  \KeywordTok{write_csv}\NormalTok{(data, }\DataTypeTok{path =} \KeywordTok{glue}\NormalTok{(}\StringTok{'data/04_data-covars-\{str_pad(sample_radius/1e3, 2, pad = "0")\}km.csv'}\NormalTok{))}
  \KeywordTok{message}\NormalTok{(}\KeywordTok{glue}\NormalTok{(}\StringTok{'export done: data/04_data-covars-\{str_pad(sample_radius/1e3, 2, pad = "0")\}km.csv'}\NormalTok{))}
\NormalTok{\})}
\end{Highlighting}
\end{Shaded}

\hypertarget{modelling-species-occupancy}{%
\section{Modelling Species Occupancy}\label{modelling-species-occupancy}}

\hypertarget{load-necessary-libraries}{%
\subsubsection{Load necessary libraries}\label{load-necessary-libraries}}

\begin{Shaded}
\begin{Highlighting}[numbers=left,,]
\CommentTok{# Load libraries}
\KeywordTok{library}\NormalTok{(auk)}
\KeywordTok{library}\NormalTok{(lubridate)}
\KeywordTok{library}\NormalTok{(sf)}
\KeywordTok{library}\NormalTok{(unmarked)}
\KeywordTok{library}\NormalTok{(raster)}
\KeywordTok{library}\NormalTok{(ebirdst)}
\KeywordTok{library}\NormalTok{(MuMIn)}
\KeywordTok{library}\NormalTok{(AICcmodavg)}
\KeywordTok{library}\NormalTok{(fields)}
\KeywordTok{library}\NormalTok{(tidyverse)}
\KeywordTok{library}\NormalTok{(doParallel)}
\KeywordTok{library}\NormalTok{(snow)}
\KeywordTok{library}\NormalTok{(openxlsx)}
\KeywordTok{library}\NormalTok{(data.table)}
\KeywordTok{library}\NormalTok{(dplyr)}
\KeywordTok{library}\NormalTok{(ecodist)}

\CommentTok{# Source necessary functions}
\KeywordTok{source}\NormalTok{(}\StringTok{"code/fun_screen_cor.R"}\NormalTok{)}
\KeywordTok{source}\NormalTok{(}\StringTok{"code/fun_model_estimate_collection.r"}\NormalTok{)}
\end{Highlighting}
\end{Shaded}

\hypertarget{load-dataframe-and-scale-covariates}{%
\subsection{Load dataframe and scale covariates}\label{load-dataframe-and-scale-covariates}}

Here, we load the required dataframe that contains 10 random visits to a site and environmental covariates that were prepared at a spatial scale of 2.5 sq.km. We also scaled all covariates (mean around 0 and standard deviation of 1). Next, we ensured that only Traveling and Stationary checklists were considered. Even though stationary counts have no distance traveled, we defaulted all stationary accounts to an effective distance of 100m, which we consider the average maximum detection radius for most bird species in our area. Following this, we excluded predictors with a Pearson coefficient \textgreater{} 0.5 which resulted in the removal of grasslands as it was highly negatively correlated with forests (r = -0.77).

Please note that species-specific plots of probabilities of occupancy as a function of environmental data can be accessed in Section 10 of the Supplementary Material.

\begin{Shaded}
\begin{Highlighting}[numbers=left,,]
\CommentTok{# Load in the prepared dataframe that contains 10 random visits to each site}
\NormalTok{dat <-}\StringTok{ }\KeywordTok{fread}\NormalTok{(}\StringTok{"data/04_data-covars-2.5km.csv"}\NormalTok{, }\DataTypeTok{header =}\NormalTok{ T)}
\KeywordTok{setDF}\NormalTok{(dat)}
\KeywordTok{head}\NormalTok{(dat)}

\CommentTok{# Some more pre-processing to get the right data structures}

\CommentTok{# Ensuring that only Traveling and Stationary checklists were considered}
\KeywordTok{names}\NormalTok{(dat)}
\NormalTok{dat <-}\StringTok{ }\NormalTok{dat }\OperatorTok\StringTok{ }\KeywordTok{filter}\NormalTok{(protocol_type }\OperatorTok\StringTok{ }\KeywordTok{c}\NormalTok{(}\StringTok{"Traveling"}\NormalTok{, }\StringTok{"Stationary"}\NormalTok{))}

\CommentTok{# We take all stationary counts and give them a distance of 100 m (so 0.1 km),}
\CommentTok{# as that's approximately the max normal hearing distance for people doing point counts.}
\NormalTok{dat <-}\StringTok{ }\NormalTok{dat }\OperatorTok
\StringTok{  }\KeywordTok{mutate}\NormalTok{(}\DataTypeTok{effort_distance_km =} \KeywordTok{replace}\NormalTok{(}
\NormalTok{    effort_distance_km,}
    \KeywordTok{which}\NormalTok{(effort_distance_km }\OperatorTok{==}\StringTok{ }\DecValTok{0} \OperatorTok{&}
\StringTok{      }\NormalTok{protocol_type }\OperatorTok{==}\StringTok{ "Stationary"}\NormalTok{),}
    \FloatTok{0.1}
\NormalTok{  ))}

\CommentTok{# Converting time observations started to numeric and adding it as a new column}
\CommentTok{# This new column will be minute_observations_started}
\NormalTok{dat <-}\StringTok{ }\NormalTok{dat }\OperatorTok
\StringTok{  }\KeywordTok{mutate}\NormalTok{(}\DataTypeTok{min_obs_started =} \KeywordTok{strtoi}\NormalTok{(}\KeywordTok{as.difftime}\NormalTok{(time_observations_started,}
    \DataTypeTok{format =} \StringTok{"%H:%M:%S"}\NormalTok{, }\DataTypeTok{units =} \StringTok{"mins"}
\NormalTok{  )))}

\CommentTok{# Adding the julian date to the dataframe}
\NormalTok{dat <-}\StringTok{ }\NormalTok{dat }\OperatorTok\StringTok{ }\KeywordTok{mutate}\NormalTok{(}\DataTypeTok{julian_date =}\NormalTok{ lubridate}\OperatorTok{::}\KeywordTok{yday}\NormalTok{(dat}\OperatorTok{$}\NormalTok{observation_date))}

\CommentTok{# Removing other unnecessary columns from the dataframe and creating a clean one without the rest}
\KeywordTok{names}\NormalTok{(dat)}
\NormalTok{dat <-}\StringTok{ }\NormalTok{dat[, }\OperatorTok{-}\KeywordTok{c}\NormalTok{(}\DecValTok{1}\NormalTok{, }\DecValTok{4}\NormalTok{, }\DecValTok{5}\NormalTok{, }\DecValTok{16}\NormalTok{, }\DecValTok{18}\NormalTok{, }\DecValTok{21}\NormalTok{, }\DecValTok{23}\NormalTok{, }\DecValTok{24}\NormalTok{, }\DecValTok{25}\NormalTok{, }\DecValTok{26}\NormalTok{, }\DecValTok{28}\OperatorTok{:}\DecValTok{37}\NormalTok{, }\DecValTok{39}\OperatorTok{:}\DecValTok{45}\NormalTok{, }\DecValTok{47}\NormalTok{)]}

\CommentTok{# Rename column names:}
\KeywordTok{names}\NormalTok{(dat) <-}\StringTok{ }\KeywordTok{c}\NormalTok{(}
  \StringTok{"duration_minutes"}\NormalTok{, }\StringTok{"effort_distance_km"}\NormalTok{, }\StringTok{"locality"}\NormalTok{,}
  \StringTok{"locality_type"}\NormalTok{, }\StringTok{"locality_id"}\NormalTok{, }\StringTok{"observer_id"}\NormalTok{,}
  \StringTok{"observation_date"}\NormalTok{, }\StringTok{"scientific_name"}\NormalTok{, }\StringTok{"observation_count"}\NormalTok{,}
  \StringTok{"protocol_type"}\NormalTok{, }\StringTok{"number_observers"}\NormalTok{, }\StringTok{"pres_abs"}\NormalTok{, }\StringTok{"tot_effort"}\NormalTok{,}
  \StringTok{"longitude"}\NormalTok{, }\StringTok{"latitude"}\NormalTok{, }\StringTok{"expertise"}\NormalTok{, }\StringTok{"bio_1.y"}\NormalTok{, }\StringTok{"bio_12.y"}\NormalTok{,}
  \StringTok{"lc_02.y"}\NormalTok{, }\StringTok{"lc_06.y"}\NormalTok{, }\StringTok{"lc_01.y"}\NormalTok{, }\StringTok{"lc_07.y"}\NormalTok{, }\StringTok{"lc_04.y"}\NormalTok{,}
  \StringTok{"lc_05.y"}\NormalTok{, }\StringTok{"min_obs_started"}\NormalTok{, }\StringTok{"julian_date"}
\NormalTok{)}

\NormalTok{dat}\FloatTok{.1}\NormalTok{ <-}\StringTok{ }\NormalTok{dat }\OperatorTok
\StringTok{  }\KeywordTok{mutate}\NormalTok{(}
    \DataTypeTok{year =} \KeywordTok{year}\NormalTok{(observation_date),}
    \DataTypeTok{pres_abs =} \KeywordTok{as.integer}\NormalTok{(pres_abs)}
\NormalTok{  ) }\CommentTok{# occupancy modeling requires an integer response}

\CommentTok{# Dividing Annual Mean Temperature by 10 to arrive at accurate values of temperature}
\NormalTok{dat}\FloatTok{.1}\OperatorTok{$}\NormalTok{bio_}\FloatTok{1.}\NormalTok{y <-}\StringTok{ }\NormalTok{dat}\FloatTok{.1}\OperatorTok{$}\NormalTok{bio_}\FloatTok{1.}\NormalTok{y }\OperatorTok{/}\StringTok{ }\DecValTok{10}

\CommentTok{# Scaling detection and occupancy covariates}
\NormalTok{dat.scaled <-}\StringTok{ }\NormalTok{dat}\FloatTok{.1}
\NormalTok{dat.scaled[, }\KeywordTok{c}\NormalTok{(}\DecValTok{1}\NormalTok{, }\DecValTok{2}\NormalTok{, }\DecValTok{11}\NormalTok{, }\DecValTok{16}\OperatorTok{:}\DecValTok{25}\NormalTok{)] <-}\StringTok{ }\KeywordTok{scale}\NormalTok{(dat.scaled[, }\KeywordTok{c}\NormalTok{(}\DecValTok{1}\NormalTok{, }\DecValTok{2}\NormalTok{, }\DecValTok{11}\NormalTok{, }\DecValTok{16}\OperatorTok{:}\DecValTok{25}\NormalTok{)]) }\CommentTok{# Scaling and standardizing detection and site-level covariates}
\KeywordTok{fwrite}\NormalTok{(dat.scaled, }\DataTypeTok{file =} \StringTok{"data/05_scaled-covars-2.5km.csv"}\NormalTok{)}

\CommentTok{# Reload the scaled covariate data}
\NormalTok{dat.scaled <-}\StringTok{ }\KeywordTok{fread}\NormalTok{(}\StringTok{"data/05_scaled-covars-2.5km.csv"}\NormalTok{, }\DataTypeTok{header =}\NormalTok{ T)}
\KeywordTok{setDF}\NormalTok{(dat.scaled)}
\KeywordTok{head}\NormalTok{(dat.scaled)}

\CommentTok{# Ensure observation_date column is in the right format}
\NormalTok{dat.scaled}\OperatorTok{$}\NormalTok{observation_date <-}\StringTok{ }\KeywordTok{format}\NormalTok{(}
  \KeywordTok{as.Date}\NormalTok{(}
\NormalTok{    dat.scaled}\OperatorTok{$}\NormalTok{observation_date,}
    \StringTok{"%m/%d/%Y"}
\NormalTok{  ),}
  \StringTok{"%Y-%m-%d"}
\NormalTok{)}

\CommentTok{# Testing for correlations before running further analyses}
\CommentTok{# Most are uncorrelated since we decided to keep only 2 climatic and 6 land cover predictors}
\KeywordTok{source}\NormalTok{(}\StringTok{"code/screen_cor.R"}\NormalTok{)}
\KeywordTok{names}\NormalTok{(dat.scaled)}
\KeywordTok{screen.cor}\NormalTok{(dat.scaled[, }\KeywordTok{c}\NormalTok{(}\DecValTok{1}\NormalTok{, }\DecValTok{2}\NormalTok{, }\DecValTok{11}\NormalTok{, }\DecValTok{16}\OperatorTok{:}\DecValTok{25}\NormalTok{)], }\DataTypeTok{threshold =} \FloatTok{0.5}\NormalTok{)}
\end{Highlighting}
\end{Shaded}

\hypertarget{running-a-null-model}{%
\subsection{Running a null model}\label{running-a-null-model}}

\begin{Shaded}
\begin{Highlighting}[numbers=left,,]
\CommentTok{# All null models are stored in lists below}
\NormalTok{all_null <-}\StringTok{ }\KeywordTok{list}\NormalTok{()}

\CommentTok{# Add a progress bar for the loop}
\NormalTok{pb <-}\StringTok{ }\KeywordTok{txtProgressBar}\NormalTok{(}
  \DataTypeTok{min =} \DecValTok{0}\NormalTok{,}
  \DataTypeTok{max =} \KeywordTok{length}\NormalTok{(}\KeywordTok{unique}\NormalTok{(dat.scaled}\OperatorTok{$}\NormalTok{scientific_name)),}
  \DataTypeTok{style =} \DecValTok{3}
\NormalTok{) }\CommentTok{# text based bar}

\ControlFlowTok{for}\NormalTok{ (i }\ControlFlowTok{in} \DecValTok{1}\OperatorTok{:}\KeywordTok{length}\NormalTok{(}\KeywordTok{unique}\NormalTok{(dat.scaled}\OperatorTok{$}\NormalTok{scientific_name))) \{}
\NormalTok{  data <-}\StringTok{ }\NormalTok{dat.scaled }\OperatorTok\StringTok{ }
\StringTok{  }\KeywordTok{filter}\NormalTok{(dat.scaled}\OperatorTok{$}\NormalTok{scientific_name }\OperatorTok{==}\StringTok{ }\KeywordTok{unique}\NormalTok{(dat.scaled}\OperatorTok{$}\NormalTok{scientific_name)[i])}

  \CommentTok{# Preparing data for the unmarked model}
\NormalTok{  occ <-}\StringTok{ }\KeywordTok{filter_repeat_visits}\NormalTok{(data,}
    \DataTypeTok{min_obs =} \DecValTok{1}\NormalTok{, }\DataTypeTok{max_obs =} \DecValTok{10}\NormalTok{,}
    \DataTypeTok{annual_closure =} \OtherTok{FALSE}\NormalTok{,}
    \DataTypeTok{n_days =} \DecValTok{2600}\NormalTok{, }\CommentTok{# 7 years is considered a period of closure}
    \DataTypeTok{date_var =} \StringTok{"observation_date"}\NormalTok{,}
    \DataTypeTok{site_vars =} \KeywordTok{c}\NormalTok{(}\StringTok{"locality_id"}\NormalTok{)}
\NormalTok{  )}

\NormalTok{  obs_covs <-}\StringTok{ }\KeywordTok{c}\NormalTok{(}
    \StringTok{"min_obs_started"}\NormalTok{,}
    \StringTok{"duration_minutes"}\NormalTok{,}
    \StringTok{"effort_distance_km"}\NormalTok{,}
    \StringTok{"number_observers"}\NormalTok{,}
    \StringTok{"protocol_type"}\NormalTok{,}
    \StringTok{"expertise"}\NormalTok{,}
    \StringTok{"julian_date"}
\NormalTok{  )}

  \CommentTok{# format for unmarked}
\NormalTok{  occ_wide <-}\StringTok{ }\KeywordTok{format_unmarked_occu}\NormalTok{(occ,}
    \DataTypeTok{site_id =} \StringTok{"site"}\NormalTok{,}
    \DataTypeTok{response =} \StringTok{"pres_abs"}\NormalTok{,}
    \DataTypeTok{site_covs =} \KeywordTok{c}\NormalTok{(}\StringTok{"locality_id"}\NormalTok{, }\StringTok{"lc_01.y"}\NormalTok{, }\StringTok{"lc_02.y"}\NormalTok{, }\StringTok{"lc_04.y"}\NormalTok{, }
    \StringTok{"lc_05.y"}\NormalTok{, }\StringTok{"lc_06.y"}\NormalTok{, }\StringTok{"lc_07.y"}\NormalTok{, }\StringTok{"bio_1.y"}\NormalTok{, }\StringTok{"bio_12.y"}\NormalTok{),}
    \DataTypeTok{obs_covs =}\NormalTok{ obs_covs}
\NormalTok{  )}

  \CommentTok{# Convert this dataframe of observations into an unmarked object to start fitting occupancy models}
\NormalTok{  occ_um <-}\StringTok{ }\KeywordTok{formatWide}\NormalTok{(occ_wide, }\DataTypeTok{type =} \StringTok{"unmarkedFrameOccu"}\NormalTok{)}

  \CommentTok{# Set up the model}
\NormalTok{  all_null[[i]] <-}\StringTok{ }\KeywordTok{occu}\NormalTok{(}\OperatorTok{~}\DecValTok{1} \OperatorTok{~}\StringTok{ }\DecValTok{1}\NormalTok{, }\DataTypeTok{data =}\NormalTok{ occ_um)}
  \KeywordTok{names}\NormalTok{(all_null)[i] <-}\StringTok{ }\KeywordTok{unique}\NormalTok{(dat.scaled}\OperatorTok{$}\NormalTok{scientific_name)[i]}
  \KeywordTok{setTxtProgressBar}\NormalTok{(pb, i)}
\NormalTok{\}}
\KeywordTok{close}\NormalTok{(pb)}

\CommentTok{# Store all the  model outputs for each species}
\KeywordTok{capture.output}\NormalTok{(all_null, }\DataTypeTok{file =} \StringTok{"data}\CharTok{\textbackslash{}r}\StringTok{esults}\CharTok{\textbackslash{}n}\StringTok{ull_models.csv"}\NormalTok{)}
\end{Highlighting}
\end{Shaded}

\hypertarget{identifying-covariates-necessary-to-model-the-detection-process}{%
\subsection{Identifying covariates necessary to model the detection process}\label{identifying-covariates-necessary-to-model-the-detection-process}}

Here, we use the \texttt{unmarked} package in R (Fiske and Chandler \protect\hyperlink{ref-fiske2011}{2011}) to identify detection level covariates that are important for each species. We use AIC criteria to select top models (Burnham, Anderson, and Huyvaert \protect\hyperlink{ref-burnham2011}{2011}).

\begin{Shaded}
\begin{Highlighting}[numbers=left,,]

\CommentTok{# All models are stored in lists below}
\NormalTok{det_dred <-}\StringTok{ }\KeywordTok{list}\NormalTok{()}

\CommentTok{# Subsetting those models whose deltaAIC<2 (Burnham et al., 2011)}
\NormalTok{top_det <-}\StringTok{ }\KeywordTok{list}\NormalTok{()}

\CommentTok{# Getting model averaged coefficients and relative importance scores}
\NormalTok{det_avg <-}\StringTok{ }\KeywordTok{list}\NormalTok{()}
\NormalTok{det_imp <-}\StringTok{ }\KeywordTok{list}\NormalTok{()}

\CommentTok{# Getting model estimates}
\NormalTok{det_modelEst <-}\StringTok{ }\KeywordTok{list}\NormalTok{()}

\CommentTok{# Add a progress bar for the loop}
\NormalTok{pb <-}\StringTok{ }\KeywordTok{txtProgressBar}\NormalTok{(}\DataTypeTok{min =} \DecValTok{0}\NormalTok{, }
  \DataTypeTok{max =} \KeywordTok{length}\NormalTok{(}\KeywordTok{unique}\NormalTok{(dat.scaled}\OperatorTok{$}\NormalTok{scientific_name)), }\DataTypeTok{style =} \DecValTok{3}\NormalTok{) }\CommentTok{# text based bar}

\ControlFlowTok{for}\NormalTok{ (i }\ControlFlowTok{in} \DecValTok{1}\OperatorTok{:}\KeywordTok{length}\NormalTok{(}\KeywordTok{unique}\NormalTok{(dat.scaled}\OperatorTok{$}\NormalTok{scientific_name))) \{}
\NormalTok{  data <-}\StringTok{ }\NormalTok{dat.scaled }\OperatorTok\StringTok{ }
\StringTok{    }\KeywordTok{filter}\NormalTok{(dat.scaled}\OperatorTok{$}\NormalTok{scientific_name }\OperatorTok{==}\StringTok{ }\KeywordTok{unique}\NormalTok{(dat.scaled}\OperatorTok{$}\NormalTok{scientific_name)[i])}

  \CommentTok{# Preparing data for the unmarked model}
\NormalTok{  occ <-}\StringTok{ }\KeywordTok{filter_repeat_visits}\NormalTok{(data,}
    \DataTypeTok{min_obs =} \DecValTok{1}\NormalTok{, }\DataTypeTok{max_obs =} \DecValTok{10}\NormalTok{,}
    \DataTypeTok{annual_closure =} \OtherTok{FALSE}\NormalTok{,}
    \DataTypeTok{n_days =} \DecValTok{2600}\NormalTok{, }\CommentTok{# 6 years is considered a period of closure}
    \DataTypeTok{date_var =} \StringTok{"observation_date"}\NormalTok{,}
    \DataTypeTok{site_vars =} \KeywordTok{c}\NormalTok{(}\StringTok{"locality_id"}\NormalTok{)}
\NormalTok{  )}

\NormalTok{  obs_covs <-}\StringTok{ }\KeywordTok{c}\NormalTok{(}
    \StringTok{"min_obs_started"}\NormalTok{,}
    \StringTok{"duration_minutes"}\NormalTok{,}
    \StringTok{"effort_distance_km"}\NormalTok{,}
    \StringTok{"number_observers"}\NormalTok{,}
    \StringTok{"protocol_type"}\NormalTok{,}
    \StringTok{"expertise"}\NormalTok{,}
    \StringTok{"julian_date"}
\NormalTok{  )}

  \CommentTok{# format for unmarked}
\NormalTok{  occ_wide <-}\StringTok{ }\KeywordTok{format_unmarked_occu}\NormalTok{(occ,}
    \DataTypeTok{site_id =} \StringTok{"site"}\NormalTok{,}
    \DataTypeTok{response =} \StringTok{"pres_abs"}\NormalTok{,}
    \DataTypeTok{site_covs =} \KeywordTok{c}\NormalTok{(}\StringTok{"locality_id"}\NormalTok{, }\StringTok{"lc_01.y"}\NormalTok{, }\StringTok{"lc_02.y"}\NormalTok{, }\StringTok{"lc_04.y"}\NormalTok{, }
      \StringTok{"lc_05.y"}\NormalTok{, }\StringTok{"lc_06.y"}\NormalTok{, }\StringTok{"lc_07.y"}\NormalTok{, }\StringTok{"bio_1.y"}\NormalTok{, }\StringTok{"bio_12.y"}\NormalTok{), }
      \DataTypeTok{obs_covs =}\NormalTok{ obs_covs}
\NormalTok{  )}

  \CommentTok{# Convert this dataframe of observations into an unmarked object to start fitting occupancy models}
\NormalTok{  occ_um <-}\StringTok{ }\KeywordTok{formatWide}\NormalTok{(occ_wide, }\DataTypeTok{type =} \StringTok{"unmarkedFrameOccu"}\NormalTok{)}

  \CommentTok{# Fit a global model with all detection level covariates}
\NormalTok{  global_mod <-}\StringTok{ }\KeywordTok{occu}\NormalTok{(}\OperatorTok{~}\StringTok{ }\NormalTok{min_obs_started }\OperatorTok{+}
\StringTok{    }\NormalTok{julian_date }\OperatorTok{+}
\StringTok{    }\NormalTok{duration_minutes }\OperatorTok{+}
\StringTok{    }\NormalTok{effort_distance_km }\OperatorTok{+}
\StringTok{    }\NormalTok{number_observers }\OperatorTok{+}
\StringTok{    }\NormalTok{protocol_type }\OperatorTok{+}
\StringTok{    }\NormalTok{expertise }\OperatorTok{~}\StringTok{ }\DecValTok{1}\NormalTok{, }\DataTypeTok{data =}\NormalTok{ occ_um)}

  \CommentTok{# Set up the cluster}
\NormalTok{  clusterType <-}\StringTok{ }\ControlFlowTok{if}\NormalTok{ (}\KeywordTok{length}\NormalTok{(}\KeywordTok{find.package}\NormalTok{(}\StringTok{"snow"}\NormalTok{, }\DataTypeTok{quiet =} \OtherTok{TRUE}\NormalTok{))) }\StringTok{"SOCK"} \ControlFlowTok{else} \StringTok{"PSOCK"}
\NormalTok{  clust <-}\StringTok{ }\KeywordTok{try}\NormalTok{(}\KeywordTok{makeCluster}\NormalTok{(}\KeywordTok{getOption}\NormalTok{(}\StringTok{"cl.cores"}\NormalTok{, }\DecValTok{6}\NormalTok{), }\DataTypeTok{type =}\NormalTok{ clusterType))}

  \KeywordTok{clusterEvalQ}\NormalTok{(clust, }\KeywordTok{library}\NormalTok{(unmarked))}
  \KeywordTok{clusterExport}\NormalTok{(clust, }\StringTok{"occ_um"}\NormalTok{)}

\NormalTok{  det_dred[[i]] <-}\StringTok{ }\KeywordTok{pdredge}\NormalTok{(global_mod, clust)}
  \KeywordTok{names}\NormalTok{(det_dred)[i] <-}\StringTok{ }\KeywordTok{unique}\NormalTok{(dat.scaled}\OperatorTok{$}\NormalTok{scientific_name)[i]}

  \CommentTok{# Get the top models, which we'll define as those with deltaAICc < 2}
\NormalTok{  top_det[[i]] <-}\StringTok{ }\KeywordTok{get.models}\NormalTok{(det_dred[[i]], }\DataTypeTok{subset =}\NormalTok{ delta }\OperatorTok{<}\StringTok{ }\DecValTok{2}\NormalTok{, }\DataTypeTok{cluster =}\NormalTok{ clust)}
  \KeywordTok{names}\NormalTok{(top_det)[i] <-}\StringTok{ }\KeywordTok{unique}\NormalTok{(dat.scaled}\OperatorTok{$}\NormalTok{scientific_name)[i]}

  \CommentTok{# Obtaining model averaged coefficients}
  \ControlFlowTok{if}\NormalTok{ (}\KeywordTok{length}\NormalTok{(top_det[[i]]) }\OperatorTok{>}\StringTok{ }\DecValTok{1}\NormalTok{) \{}
\NormalTok{    a <-}\StringTok{ }\KeywordTok{model.avg}\NormalTok{(top_det[[i]], }\DataTypeTok{fit =} \OtherTok{TRUE}\NormalTok{)}
\NormalTok{    det_avg[[i]] <-}\StringTok{ }\KeywordTok{as.data.frame}\NormalTok{(a}\OperatorTok{$}\NormalTok{coefficients)}
    \KeywordTok{names}\NormalTok{(det_avg)[i] <-}\StringTok{ }\KeywordTok{unique}\NormalTok{(dat.scaled}\OperatorTok{$}\NormalTok{scientific_name)[i]}

\NormalTok{    det_modelEst[[i]] <-}\StringTok{ }\KeywordTok{data.frame}\NormalTok{(}
      \DataTypeTok{Coefficient =} \KeywordTok{coefTable}\NormalTok{(a, }\DataTypeTok{full =}\NormalTok{ T)[, }\DecValTok{1}\NormalTok{],}
      \DataTypeTok{SE =} \KeywordTok{coefTable}\NormalTok{(a, }\DataTypeTok{full =}\NormalTok{ T)[, }\DecValTok{2}\NormalTok{],}
      \DataTypeTok{lowerCI =} \KeywordTok{confint}\NormalTok{(a)[, }\DecValTok{1}\NormalTok{],}
      \DataTypeTok{upperCI =} \KeywordTok{confint}\NormalTok{(a)[, }\DecValTok{2}\NormalTok{],}
      \DataTypeTok{z_value =}\NormalTok{ (}\KeywordTok{summary}\NormalTok{(a)}\OperatorTok{$}\NormalTok{coefmat.full)[, }\DecValTok{3}\NormalTok{],}
      \DataTypeTok{Pr_z =}\NormalTok{ (}\KeywordTok{summary}\NormalTok{(a)}\OperatorTok{$}\NormalTok{coefmat.full)[, }\DecValTok{4}\NormalTok{]}
\NormalTok{    )}

    \KeywordTok{names}\NormalTok{(det_modelEst)[i] <-}\StringTok{ }\KeywordTok{unique}\NormalTok{(dat.scaled}\OperatorTok{$}\NormalTok{scientific_name)[i]}

\NormalTok{    det_imp[[i]] <-}\StringTok{ }\KeywordTok{as.data.frame}\NormalTok{(MuMIn}\OperatorTok{::}\KeywordTok{importance}\NormalTok{(a))}
    \KeywordTok{names}\NormalTok{(det_imp)[i] <-}\StringTok{ }\KeywordTok{unique}\NormalTok{(dat.scaled}\OperatorTok{$}\NormalTok{scientific_name)[i]}
\NormalTok{  \} }\ControlFlowTok{else}\NormalTok{ \{}
\NormalTok{    det_avg[[i]] <-}\StringTok{ }\KeywordTok{as.data.frame}\NormalTok{(unmarked}\OperatorTok{::}\KeywordTok{coef}\NormalTok{(top_det[[i]][[}\DecValTok{1}\NormalTok{]]))}
    \KeywordTok{names}\NormalTok{(det_avg)[i] <-}\StringTok{ }\KeywordTok{unique}\NormalTok{(dat.scaled}\OperatorTok{$}\NormalTok{scientific_name)[i]}

\NormalTok{    lowDet <-}\StringTok{ }\KeywordTok{data.frame}\NormalTok{(}\DataTypeTok{lowerCI =} \KeywordTok{confint}\NormalTok{(top_det[[i]][[}\DecValTok{1}\NormalTok{]], }\DataTypeTok{type =} \StringTok{"det"}\NormalTok{)[, }\DecValTok{1}\NormalTok{])}
\NormalTok{    upDet <-}\StringTok{ }\KeywordTok{data.frame}\NormalTok{(}\DataTypeTok{upperCI =} \KeywordTok{confint}\NormalTok{(top_det[[i]][[}\DecValTok{1}\NormalTok{]], }\DataTypeTok{type =} \StringTok{"det"}\NormalTok{)[, }\DecValTok{2}\NormalTok{])}
\NormalTok{    zDet <-}\StringTok{ }\KeywordTok{data.frame}\NormalTok{(}\KeywordTok{summary}\NormalTok{(top_det[[i]][[}\DecValTok{1}\NormalTok{]])}\OperatorTok{$}\NormalTok{det[, }\DecValTok{3}\NormalTok{])}
\NormalTok{    Pr_zDet <-}\StringTok{ }\KeywordTok{data.frame}\NormalTok{(}\KeywordTok{summary}\NormalTok{(top_det[[i]][[}\DecValTok{1}\NormalTok{]])}\OperatorTok{$}\NormalTok{det[, }\DecValTok{4}\NormalTok{])}

\NormalTok{    Coefficient <-}\StringTok{ }\KeywordTok{coefTable}\NormalTok{(top_det[[i]][[}\DecValTok{1}\NormalTok{]])[, }\DecValTok{1}\NormalTok{]}
\NormalTok{    SE <-}\StringTok{ }\KeywordTok{coefTable}\NormalTok{(top_det[[i]][[}\DecValTok{1}\NormalTok{]])[, }\DecValTok{2}\NormalTok{]}

\NormalTok{    det_modelEst[[i]] <-}\StringTok{ }\KeywordTok{data.frame}\NormalTok{(}
      \DataTypeTok{Coefficient =}\NormalTok{ Coefficient[}\DecValTok{2}\OperatorTok{:}\DecValTok{9}\NormalTok{],}
      \DataTypeTok{SE =}\NormalTok{ SE[}\DecValTok{2}\OperatorTok{:}\DecValTok{9}\NormalTok{],}
      \DataTypeTok{lowerCI =}\NormalTok{ lowDet,}
      \DataTypeTok{upperCI =}\NormalTok{ upDet,}
      \DataTypeTok{z_value =}\NormalTok{ zDet,}
      \DataTypeTok{Pr_z =}\NormalTok{ Pr_zDet}
\NormalTok{    )}

    \KeywordTok{names}\NormalTok{(det_modelEst)[i] <-}\StringTok{ }\KeywordTok{unique}\NormalTok{(dat.scaled}\OperatorTok{$}\NormalTok{scientific_name)[i]}
\NormalTok{  \}}
  \KeywordTok{setTxtProgressBar}\NormalTok{(pb, i)}
  \KeywordTok{stopCluster}\NormalTok{(clust)}
\NormalTok{\}}
\KeywordTok{close}\NormalTok{(pb)}

\CommentTok{## Storing output from the above models in excel sheets}

\CommentTok{# 1. Store all the model outputs for each species (variable: det_dred() - see above)}
\KeywordTok{write.xlsx}\NormalTok{(det_dred, }\DataTypeTok{file =} \StringTok{"data}\CharTok{\textbackslash{}r}\StringTok{esults\textbackslash{}det-dred.xlsx"}\NormalTok{)}

\CommentTok{# 2. Store all the model averaged outputs for each species and the relative importance score}
\KeywordTok{write.xlsx}\NormalTok{(det_avg, }\DataTypeTok{file =} \StringTok{"data}\CharTok{\textbackslash{}r}\StringTok{esults\textbackslash{}det-avg.xlsx"}\NormalTok{, }\DataTypeTok{rowNames =}\NormalTok{ T, }\DataTypeTok{colNames =}\NormalTok{ T)}
\KeywordTok{write.xlsx}\NormalTok{(det_imp, }\DataTypeTok{file =} \StringTok{"data}\CharTok{\textbackslash{}r}\StringTok{esults\textbackslash{}det-imp.xlsx"}\NormalTok{, }\DataTypeTok{rowNames =}\NormalTok{ T, }\DataTypeTok{colNames =}\NormalTok{ T)}

\KeywordTok{write.xlsx}\NormalTok{(det_modelEst, }\DataTypeTok{file =} \StringTok{"data}\CharTok{\textbackslash{}r}\StringTok{esults\textbackslash{}det-modelEst.xlsx"}\NormalTok{, }\DataTypeTok{rowNames =}\NormalTok{ T, }\DataTypeTok{colNames =}\NormalTok{ T)}
\end{Highlighting}
\end{Shaded}

\hypertarget{land-cover-and-climate}{%
\subsection{Land Cover and Climate}\label{land-cover-and-climate}}

Occupancy models estimate the probability of occurrence of a given species while controlling for the probability of detection and allow us to model the factors affecting occurrence and detection independently (Johnston et al. \protect\hyperlink{ref-johnston2018}{2018}; MacKenzie et al. \protect\hyperlink{ref-mackenzie2002}{2002}). The flexible eBird observation process contributes to the largest source of variation in the likelihood of detecting a particular species (Johnston et al. \protect\hyperlink{ref-johnston2019a}{2019}); hence, we included seven covariates that influence the probability of detection for each checklist: ordinal day of year, duration of observation, distance travelled, protocol type, time observations started, number of observers and the checklist calibration index (CCI).

Using a multi-model information-theoretic approach, we tested how strongly our occurrence data fit our candidate set of environmental covariates (Burnham and Anderson \protect\hyperlink{ref-burnham2002a}{2002}). We fitted single-species occupancy models for each species, to simultaneously estimate a probability of detection (p) and a probability of occupancy (\(\psi\)) (Fiske and Chandler \protect\hyperlink{ref-fiske2011}{2011}; MacKenzie et al. \protect\hyperlink{ref-mackenzie2002}{2002}). For each species, we fit 256 models, each with a unique combination of the eight (climate and land cover) occupancy covariates and all seven detection covariates (Appendix S5).

Across the 256 models tested for each species, the model with highest support was determined using AICc scores. However, across the majority of the species, no single model had overwhelming support. Hence, for each species, we examined those models which had \(\Delta\)AICc \textless{} 2, as these top models were considered to explain a large proportion of the association between the species-specific probability of occupancy and environmental drivers (Burnham, Anderson, and Huyvaert \protect\hyperlink{ref-burnham2011}{2011}; Elsen et al. \protect\hyperlink{ref-elsen2017}{2017}). Using these restricted model sets for each species; we created a model-averaged coefficient estimate for each predictor and assessed its direction and significance (Bartoń \protect\hyperlink{ref-MuMIn}{2020}). We considered a predictor to be significantly associated with occupancy if the range of the 95\% confidence interval around the model-averaged coefficient did not contain zero. Next, we obtained a measure of relative importance of climatic and landscape predictors by calculating cumulative variable importance scores. These scores were calculated by obtaining the sum of model weights (AIC weights) across all models (including the top models) for each predictor across all species.

\begin{Shaded}
\begin{Highlighting}[numbers=left,,]
\CommentTok{# All models are stored in lists below}
\NormalTok{lc_clim <-}\StringTok{ }\KeywordTok{list}\NormalTok{()}

\CommentTok{# Subsetting those models whose deltaAIC<2 (Burnham et al., 2011)}
\NormalTok{top_lc_clim <-}\StringTok{ }\KeywordTok{list}\NormalTok{()}

\CommentTok{# Getting model averaged coefficients and relative importance scores}
\NormalTok{lc_clim_avg <-}\StringTok{ }\KeywordTok{list}\NormalTok{()}
\NormalTok{lc_clim_imp <-}\StringTok{ }\KeywordTok{list}\NormalTok{()}

\CommentTok{# Storing Model estimates}
\NormalTok{lc_clim_modelEst <-}\StringTok{ }\KeywordTok{list}\NormalTok{()}

\CommentTok{# Add a progress bar for the loop}
\NormalTok{pb <-}\StringTok{ }\KeywordTok{txtProgressBar}\NormalTok{(}\DataTypeTok{min =} \DecValTok{0}\NormalTok{, }\DataTypeTok{max =} \KeywordTok{length}\NormalTok{(}\KeywordTok{unique}\NormalTok{(dat.scaled}\OperatorTok{$}\NormalTok{scientific_name)), }\DataTypeTok{style =} \DecValTok{3}\NormalTok{) }\CommentTok{# text based bar}

\ControlFlowTok{for}\NormalTok{ (i }\ControlFlowTok{in} \DecValTok{1}\OperatorTok{:}\KeywordTok{length}\NormalTok{(}\KeywordTok{unique}\NormalTok{(dat.scaled}\OperatorTok{$}\NormalTok{scientific_name))) \{}
\NormalTok{  data <-}\StringTok{ }\NormalTok{dat.scaled }\OperatorTok\StringTok{ }\KeywordTok{filter}\NormalTok{(dat.scaled}\OperatorTok{$}\NormalTok{scientific_name }\OperatorTok{==}\StringTok{ }\KeywordTok{unique}\NormalTok{(dat.scaled}\OperatorTok{$}\NormalTok{scientific_name)[}\DecValTok{1}\NormalTok{])}

  \CommentTok{# Preparing data for the unmarked model}
\NormalTok{  occ <-}\StringTok{ }\KeywordTok{filter_repeat_visits}\NormalTok{(data,}
    \DataTypeTok{min_obs =} \DecValTok{1}\NormalTok{, }\DataTypeTok{max_obs =} \DecValTok{10}\NormalTok{,}
    \DataTypeTok{annual_closure =} \OtherTok{FALSE}\NormalTok{,}
    \DataTypeTok{n_days =} \DecValTok{2600}\NormalTok{, }\CommentTok{# 6 years is considered a period of closure}
    \DataTypeTok{date_var =} \StringTok{"observation_date"}\NormalTok{,}
    \DataTypeTok{site_vars =} \KeywordTok{c}\NormalTok{(}\StringTok{"locality_id"}\NormalTok{)}
\NormalTok{  )}

\NormalTok{  obs_covs <-}\StringTok{ }\KeywordTok{c}\NormalTok{(}
    \StringTok{"min_obs_started"}\NormalTok{,}
    \StringTok{"duration_minutes"}\NormalTok{,}
    \StringTok{"effort_distance_km"}\NormalTok{,}
    \StringTok{"number_observers"}\NormalTok{,}
    \StringTok{"protocol_type"}\NormalTok{,}
    \StringTok{"expertise"}\NormalTok{,}
    \StringTok{"julian_date"}
\NormalTok{  )}

  \CommentTok{# format for unmarked}
\NormalTok{  occ_wide <-}\StringTok{ }\KeywordTok{format_unmarked_occu}\NormalTok{(occ,}
    \DataTypeTok{site_id =} \StringTok{"site"}\NormalTok{,}
    \DataTypeTok{response =} \StringTok{"pres_abs"}\NormalTok{,}
    \DataTypeTok{site_covs =} \KeywordTok{c}\NormalTok{(}\StringTok{"locality_id"}\NormalTok{, }\StringTok{"lc_01.y"}\NormalTok{, }\StringTok{"lc_02.y"}\NormalTok{, }\StringTok{"lc_04.y"}\NormalTok{, }\StringTok{"lc_05.y"}\NormalTok{, }
      \StringTok{"lc_06.y"}\NormalTok{, }\StringTok{"lc_07.y"}\NormalTok{, }\StringTok{"bio_1.y"}\NormalTok{, }\StringTok{"bio_12.y"}\NormalTok{),}
    \DataTypeTok{obs_covs =}\NormalTok{ obs_covs}
\NormalTok{  )}

  \CommentTok{# Convert this dataframe of observations into an unmarked object to start fitting occupancy models}
\NormalTok{  occ_um <-}\StringTok{ }\KeywordTok{formatWide}\NormalTok{(occ_wide, }\DataTypeTok{type =} \StringTok{"unmarkedFrameOccu"}\NormalTok{)}

\NormalTok{  model_lc_clim <-}\StringTok{ }\KeywordTok{occu}\NormalTok{(}\OperatorTok{~}\StringTok{ }\NormalTok{min_obs_started }\OperatorTok{+}
\StringTok{    }\NormalTok{julian_date }\OperatorTok{+}
\StringTok{    }\NormalTok{duration_minutes }\OperatorTok{+}
\StringTok{    }\NormalTok{effort_distance_km }\OperatorTok{+}
\StringTok{    }\NormalTok{number_observers }\OperatorTok{+}
\StringTok{    }\NormalTok{protocol_type }\OperatorTok{+}
\StringTok{    }\NormalTok{expertise }\OperatorTok{~}\StringTok{ }\NormalTok{lc_}\FloatTok{01.}\NormalTok{y }\OperatorTok{+}\StringTok{ }\NormalTok{lc_}\FloatTok{02.}\NormalTok{y }\OperatorTok{+}\StringTok{ }\NormalTok{lc_}\FloatTok{04.}\NormalTok{y }\OperatorTok{+}
\StringTok{    }\NormalTok{lc_}\FloatTok{05.}\NormalTok{y }\OperatorTok{+}\StringTok{ }\NormalTok{lc_}\FloatTok{06.}\NormalTok{y }\OperatorTok{+}\StringTok{ }\NormalTok{lc_}\FloatTok{07.}\NormalTok{y }\OperatorTok{+}\StringTok{ }\NormalTok{bio_}\FloatTok{1.}\NormalTok{y }\OperatorTok{+}\StringTok{ }\NormalTok{bio_}\FloatTok{12.}\NormalTok{y, }\DataTypeTok{data =}\NormalTok{ occ_um)}

  \CommentTok{# Set up the cluster}
\NormalTok{  clusterType <-}\StringTok{ }\ControlFlowTok{if}\NormalTok{ (}\KeywordTok{length}\NormalTok{(}\KeywordTok{find.package}\NormalTok{(}\StringTok{"snow"}\NormalTok{, }\DataTypeTok{quiet =} \OtherTok{TRUE}\NormalTok{))) }\StringTok{"SOCK"} \ControlFlowTok{else} \StringTok{"PSOCK"}
\NormalTok{  clust <-}\StringTok{ }\KeywordTok{try}\NormalTok{(}\KeywordTok{makeCluster}\NormalTok{(}\KeywordTok{getOption}\NormalTok{(}\StringTok{"cl.cores"}\NormalTok{, }\DecValTok{6}\NormalTok{), }\DataTypeTok{type =}\NormalTok{ clusterType))}

  \KeywordTok{clusterEvalQ}\NormalTok{(clust, }\KeywordTok{library}\NormalTok{(unmarked))}
  \KeywordTok{clusterExport}\NormalTok{(clust, }\StringTok{"occ_um"}\NormalTok{)}

  \CommentTok{# Detection terms are fixed}
\NormalTok{  det_terms <-}\StringTok{ }\KeywordTok{c}\NormalTok{(}
    \StringTok{"p(duration_minutes)"}\NormalTok{, }\StringTok{"p(effort_distance_km)"}\NormalTok{, }\StringTok{"p(expertise)"}\NormalTok{, }
    \StringTok{"p(julian_date)"}\NormalTok{, }\StringTok{"p(min_obs_started)"}\NormalTok{,}
    \StringTok{"p(number_observers)"}\NormalTok{, }\StringTok{"p(protocol_type)"}
\NormalTok{  )}

\NormalTok{  lc_clim[[i]] <-}\StringTok{ }\KeywordTok{pdredge}\NormalTok{(model_lc_clim, clust, }\DataTypeTok{fixed =}\NormalTok{ det_terms)}
  \KeywordTok{names}\NormalTok{(lc_clim)[i] <-}\StringTok{ }\KeywordTok{unique}\NormalTok{(dat.scaled}\OperatorTok{$}\NormalTok{scientific_name)[i]}

  \CommentTok{# Identiying top subset of models based on deltaAIC scores being less than 2 (Burnham et al., 2011)}
\NormalTok{  top_lc_clim[[i]] <-}\StringTok{ }\KeywordTok{get.models}\NormalTok{(lc_clim[[i]], }\DataTypeTok{subset =}\NormalTok{ delta }\OperatorTok{<}\StringTok{ }\DecValTok{2}\NormalTok{, }\DataTypeTok{cluster =}\NormalTok{ clust)}

  \KeywordTok{names}\NormalTok{(top_lc_clim)[i] <-}\StringTok{ }\KeywordTok{unique}\NormalTok{(dat.scaled}\OperatorTok{$}\NormalTok{scientific_name)[i]}

  \CommentTok{# Obtaining model averaged coefficients for both candidate model subsets}
  \ControlFlowTok{if}\NormalTok{ (}\KeywordTok{length}\NormalTok{(top_lc_clim[[i]]) }\OperatorTok{>}\StringTok{ }\DecValTok{1}\NormalTok{) \{}
\NormalTok{    a <-}\StringTok{ }\KeywordTok{model.avg}\NormalTok{(top_lc_clim[[i]], }\DataTypeTok{fit =} \OtherTok{TRUE}\NormalTok{)}
\NormalTok{    lc_clim_avg[[i]] <-}\StringTok{ }\KeywordTok{as.data.frame}\NormalTok{(a}\OperatorTok{$}\NormalTok{coefficients)}
    \KeywordTok{names}\NormalTok{(lc_clim_avg)[i] <-}\StringTok{ }\KeywordTok{unique}\NormalTok{(dat.scaled}\OperatorTok{$}\NormalTok{scientific_name)[i]}

\NormalTok{    lc_clim_modelEst[[i]] <-}\StringTok{ }\KeywordTok{data.frame}\NormalTok{(}
      \DataTypeTok{Coefficient =} \KeywordTok{coefTable}\NormalTok{(a, }\DataTypeTok{full =}\NormalTok{ T)[, }\DecValTok{1}\NormalTok{],}
      \DataTypeTok{SE =} \KeywordTok{coefTable}\NormalTok{(a, }\DataTypeTok{full =}\NormalTok{ T)[, }\DecValTok{2}\NormalTok{],}
      \DataTypeTok{lowerCI =} \KeywordTok{confint}\NormalTok{(a)[, }\DecValTok{1}\NormalTok{],}
      \DataTypeTok{upperCI =} \KeywordTok{confint}\NormalTok{(a)[, }\DecValTok{2}\NormalTok{],}
      \DataTypeTok{z_value =}\NormalTok{ (}\KeywordTok{summary}\NormalTok{(a)}\OperatorTok{$}\NormalTok{coefmat.full)[, }\DecValTok{3}\NormalTok{],}
      \DataTypeTok{Pr_z =}\NormalTok{ (}\KeywordTok{summary}\NormalTok{(a)}\OperatorTok{$}\NormalTok{coefmat.full)[, }\DecValTok{4}\NormalTok{]}
\NormalTok{    )}

    \KeywordTok{names}\NormalTok{(lc_clim_modelEst)[i] <-}\StringTok{ }\KeywordTok{unique}\NormalTok{(dat.scaled}\OperatorTok{$}\NormalTok{scientific_name)[i]}

\NormalTok{    lc_clim_imp[[i]] <-}\StringTok{ }\KeywordTok{as.data.frame}\NormalTok{(MuMIn}\OperatorTok{::}\KeywordTok{importance}\NormalTok{(a))}
    \KeywordTok{names}\NormalTok{(lc_clim_imp)[i] <-}\StringTok{ }\KeywordTok{unique}\NormalTok{(dat.scaled}\OperatorTok{$}\NormalTok{scientific_name)[i]}
\NormalTok{  \} }\ControlFlowTok{else}\NormalTok{ \{}
\NormalTok{    lc_clim_avg[[i]] <-}\StringTok{ }\KeywordTok{as.data.frame}\NormalTok{(unmarked}\OperatorTok{::}\KeywordTok{coef}\NormalTok{(top_lc_clim[[i]][[}\DecValTok{1}\NormalTok{]]))}
    \KeywordTok{names}\NormalTok{(lc_clim_avg)[i] <-}\StringTok{ }\KeywordTok{unique}\NormalTok{(dat.scaled}\OperatorTok{$}\NormalTok{scientific_name)[i]}

\NormalTok{    lowSt <-}\StringTok{ }\KeywordTok{data.frame}\NormalTok{(}\DataTypeTok{lowerCI =} \KeywordTok{confint}\NormalTok{(top_lc_clim[[i]][[}\DecValTok{1}\NormalTok{]], }\DataTypeTok{type =} \StringTok{"state"}\NormalTok{)[, }\DecValTok{1}\NormalTok{])}
\NormalTok{    lowDet <-}\StringTok{ }\KeywordTok{data.frame}\NormalTok{(}\DataTypeTok{lowerCI =} \KeywordTok{confint}\NormalTok{(top_lc_clim[[i]][[}\DecValTok{1}\NormalTok{]], }\DataTypeTok{type =} \StringTok{"det"}\NormalTok{)[, }\DecValTok{1}\NormalTok{])}
\NormalTok{    upSt <-}\StringTok{ }\KeywordTok{data.frame}\NormalTok{(}\DataTypeTok{upperCI =} \KeywordTok{confint}\NormalTok{(top_lc_clim[[i]][[}\DecValTok{1}\NormalTok{]], }\DataTypeTok{type =} \StringTok{"state"}\NormalTok{)[, }\DecValTok{2}\NormalTok{])}
\NormalTok{    upDet <-}\StringTok{ }\KeywordTok{data.frame}\NormalTok{(}\DataTypeTok{upperCI =} \KeywordTok{confint}\NormalTok{(top_lc_clim[[i]][[}\DecValTok{1}\NormalTok{]], }\DataTypeTok{type =} \StringTok{"det"}\NormalTok{)[, }\DecValTok{2}\NormalTok{])}
\NormalTok{    zSt <-}\StringTok{ }\KeywordTok{data.frame}\NormalTok{(}\DataTypeTok{z_value =} \KeywordTok{summary}\NormalTok{(top_lc_clim[[i]][[}\DecValTok{1}\NormalTok{]])}\OperatorTok{$}\NormalTok{state[, }\DecValTok{3}\NormalTok{])}
\NormalTok{    zDet <-}\StringTok{ }\KeywordTok{data.frame}\NormalTok{(}\DataTypeTok{z_value =} \KeywordTok{summary}\NormalTok{(top_lc_clim[[i]][[}\DecValTok{1}\NormalTok{]])}\OperatorTok{$}\NormalTok{det[, }\DecValTok{3}\NormalTok{])}
\NormalTok{    Pr_zSt <-}\StringTok{ }\KeywordTok{data.frame}\NormalTok{(}\DataTypeTok{Pr_z =} \KeywordTok{summary}\NormalTok{(top_lc_clim[[i]][[}\DecValTok{1}\NormalTok{]])}\OperatorTok{$}\NormalTok{state[, }\DecValTok{4}\NormalTok{])}
\NormalTok{    Pr_zDet <-}\StringTok{ }\KeywordTok{data.frame}\NormalTok{(}\DataTypeTok{Pr_z =} \KeywordTok{summary}\NormalTok{(top_lc_clim[[i]][[}\DecValTok{1}\NormalTok{]])}\OperatorTok{$}\NormalTok{det[, }\DecValTok{4}\NormalTok{])}

\NormalTok{    lc_clim_modelEst[[i]] <-}\StringTok{ }\KeywordTok{data.frame}\NormalTok{(}
      \DataTypeTok{Coefficient =} \KeywordTok{coefTable}\NormalTok{(top_lc_clim[[i]][[}\DecValTok{1}\NormalTok{]])[, }\DecValTok{1}\NormalTok{],}
      \DataTypeTok{SE =} \KeywordTok{coefTable}\NormalTok{(top_lc_clim[[i]][[}\DecValTok{1}\NormalTok{]])[, }\DecValTok{2}\NormalTok{],}
      \DataTypeTok{lowerCI =} \KeywordTok{rbind}\NormalTok{(lowSt, lowDet),}
      \DataTypeTok{upperCI =} \KeywordTok{rbind}\NormalTok{(upSt, upDet),}
      \DataTypeTok{z_value =} \KeywordTok{rbind}\NormalTok{(zSt, zDet),}
      \DataTypeTok{Pr_z =} \KeywordTok{rbind}\NormalTok{(Pr_zSt, Pr_zDet)}
\NormalTok{    )}

    \KeywordTok{names}\NormalTok{(lc_clim_modelEst)[i] <-}\StringTok{ }\KeywordTok{unique}\NormalTok{(dat.scaled}\OperatorTok{$}\NormalTok{scientific_name)[i]}
\NormalTok{  \}}
  \KeywordTok{setTxtProgressBar}\NormalTok{(pb, i)}
  \KeywordTok{stopCluster}\NormalTok{(clust)}
\NormalTok{\}}
\KeywordTok{close}\NormalTok{(pb)}

\CommentTok{# 1. Store all the model outputs for each species (for both landcover and climate)}
\KeywordTok{write.xlsx}\NormalTok{(lc_clim, }\DataTypeTok{file =} \StringTok{"data}\CharTok{\textbackslash{}r}\StringTok{esults\textbackslash{}lc-clim.xlsx"}\NormalTok{)}

\CommentTok{# 2. Store all the model averaged outputs for each species and relative importance scores}
\KeywordTok{write.xlsx}\NormalTok{(lc_clim_avg, }\DataTypeTok{file =} \StringTok{"data}\CharTok{\textbackslash{}r}\StringTok{esults\textbackslash{}lc-clim-avg.xlsx"}\NormalTok{, }\DataTypeTok{rowNames =}\NormalTok{ T, }\DataTypeTok{colNames =}\NormalTok{ T)}
\KeywordTok{write.xlsx}\NormalTok{(lc_clim_imp, }\DataTypeTok{file =} \StringTok{"data}\CharTok{\textbackslash{}r}\StringTok{esults\textbackslash{}lc-clim-imp.xlsx"}\NormalTok{, }\DataTypeTok{rowNames =}\NormalTok{ T, }\DataTypeTok{colNames =}\NormalTok{ T)}

\CommentTok{# 3. Store all model estimates}
\KeywordTok{write.xlsx}\NormalTok{(lc_clim_modelEst, }\DataTypeTok{file =} \StringTok{"data}\CharTok{\textbackslash{}r}\StringTok{esults\textbackslash{}lc-clim-modelEst.xlsx"}\NormalTok{, }\DataTypeTok{rowNames =}\NormalTok{ T, }\DataTypeTok{colNames =}\NormalTok{ T)}
\end{Highlighting}
\end{Shaded}

\hypertarget{goodness-of-fit-tests}{%
\subsection{Goodness-of-fit tests}\label{goodness-of-fit-tests}}

Adequate model fit was assessed using a chi-square goodness-of-fit test using 5000 parametric bootstrap simulations on a global model that included all occupancy and detection covariates (MacKenzie \& Bailey, 2004).

\begin{Shaded}
\begin{Highlighting}[numbers=left,,]
\NormalTok{goodness_of_fit <-}\StringTok{ }\KeywordTok{data.frame}\NormalTok{()}

\CommentTok{# Add a progress bar for the loop}
\NormalTok{pb <-}\StringTok{ }\KeywordTok{txtProgressBar}\NormalTok{(}\DataTypeTok{min =} \DecValTok{0}\NormalTok{, }\DataTypeTok{max =} \KeywordTok{length}\NormalTok{(}\KeywordTok{unique}\NormalTok{(dat.scaled}\OperatorTok{$}\NormalTok{scientific_name)), }\DataTypeTok{style =} \DecValTok{3}\NormalTok{) }\CommentTok{# text based bar}

\ControlFlowTok{for}\NormalTok{ (i }\ControlFlowTok{in} \DecValTok{1}\OperatorTok{:}\KeywordTok{length}\NormalTok{(}\KeywordTok{unique}\NormalTok{(dat.scaled}\OperatorTok{$}\NormalTok{scientific_name))) \{}
\NormalTok{  data <-}\StringTok{ }\NormalTok{dat.scaled }\OperatorTok\StringTok{ }\KeywordTok{filter}\NormalTok{(dat.scaled}\OperatorTok{$}\NormalTok{scientific_name }\OperatorTok{==}\StringTok{ }\KeywordTok{unique}\NormalTok{(dat.scaled}\OperatorTok{$}\NormalTok{scientific_name)[i])}

  \CommentTok{# Preparing data for the unmarked model}
\NormalTok{  occ <-}\StringTok{ }\KeywordTok{filter_repeat_visits}\NormalTok{(data,}
    \DataTypeTok{min_obs =} \DecValTok{1}\NormalTok{, }\DataTypeTok{max_obs =} \DecValTok{10}\NormalTok{,}
    \DataTypeTok{annual_closure =} \OtherTok{FALSE}\NormalTok{,}
    \DataTypeTok{n_days =} \DecValTok{2600}\NormalTok{, }\CommentTok{# 6 years is considered a period of closure}
    \DataTypeTok{date_var =} \StringTok{"observation_date"}\NormalTok{,}
    \DataTypeTok{site_vars =} \KeywordTok{c}\NormalTok{(}\StringTok{"locality_id"}\NormalTok{)}
\NormalTok{  )}

\NormalTok{  obs_covs <-}\StringTok{ }\KeywordTok{c}\NormalTok{(}
    \StringTok{"min_obs_started"}\NormalTok{,}
    \StringTok{"duration_minutes"}\NormalTok{,}
    \StringTok{"effort_distance_km"}\NormalTok{,}
    \StringTok{"number_observers"}\NormalTok{,}
    \StringTok{"protocol_type"}\NormalTok{,}
    \StringTok{"expertise"}\NormalTok{,}
    \StringTok{"julian_date"}
\NormalTok{  )}

  \CommentTok{# format for unmarked}
\NormalTok{  occ_wide <-}\StringTok{ }\KeywordTok{format_unmarked_occu}\NormalTok{(occ,}
    \DataTypeTok{site_id =} \StringTok{"site"}\NormalTok{,}
    \DataTypeTok{response =} \StringTok{"pres_abs"}\NormalTok{,}
    \DataTypeTok{site_covs =} \KeywordTok{c}\NormalTok{(}\StringTok{"locality_id"}\NormalTok{, }\StringTok{"lc_01.y"}\NormalTok{, }\StringTok{"lc_02.y"}\NormalTok{, }\StringTok{"lc_04.y"}\NormalTok{, }\StringTok{"lc_05.y"}\NormalTok{, }\StringTok{"lc_06.y"}\NormalTok{, }\StringTok{"lc_07.y"}\NormalTok{, }\StringTok{"bio_1.y"}\NormalTok{, }\StringTok{"bio_12.y"}\NormalTok{),}
    \DataTypeTok{obs_covs =}\NormalTok{ obs_covs}
\NormalTok{  )}

  \CommentTok{# Convert this dataframe of observations into an unmarked object to start fitting occupancy models}
\NormalTok{  occ_um <-}\StringTok{ }\KeywordTok{formatWide}\NormalTok{(occ_wide, }\DataTypeTok{type =} \StringTok{"unmarkedFrameOccu"}\NormalTok{)}

\NormalTok{  model_lc_clim <-}\StringTok{ }\KeywordTok{occu}\NormalTok{(}\OperatorTok{~}\StringTok{ }\NormalTok{min_obs_started }\OperatorTok{+}
\StringTok{    }\NormalTok{julian_date }\OperatorTok{+}
\StringTok{    }\NormalTok{duration_minutes }\OperatorTok{+}
\StringTok{    }\NormalTok{effort_distance_km }\OperatorTok{+}
\StringTok{    }\NormalTok{number_observers }\OperatorTok{+}
\StringTok{    }\NormalTok{protocol_type }\OperatorTok{+}
\StringTok{    }\NormalTok{expertise }\OperatorTok{~}\StringTok{ }\NormalTok{lc_}\FloatTok{01.}\NormalTok{y }\OperatorTok{+}\StringTok{ }\NormalTok{lc_}\FloatTok{02.}\NormalTok{y }\OperatorTok{+}\StringTok{ }\NormalTok{lc_}\FloatTok{04.}\NormalTok{y }\OperatorTok{+}
\StringTok{    }\NormalTok{lc_}\FloatTok{05.}\NormalTok{y }\OperatorTok{+}\StringTok{ }\NormalTok{lc_}\FloatTok{06.}\NormalTok{y }\OperatorTok{+}\StringTok{ }\NormalTok{lc_}\FloatTok{07.}\NormalTok{y }\OperatorTok{+}\StringTok{ }\NormalTok{bio_}\FloatTok{1.}\NormalTok{y }\OperatorTok{+}\StringTok{ }\NormalTok{bio_}\FloatTok{12.}\NormalTok{y, }\DataTypeTok{data =}\NormalTok{ occ_um)}

\NormalTok{  occ_gof <-}\StringTok{ }\KeywordTok{mb.gof.test}\NormalTok{(model_lc_clim, }\DataTypeTok{nsim =} \DecValTok{5000}\NormalTok{, }\DataTypeTok{plot.hist =} \OtherTok{FALSE}\NormalTok{)}

\NormalTok{  p.value <-}\StringTok{ }\NormalTok{occ_gof}\OperatorTok{$}\NormalTok{p.value}
\NormalTok{  c.hat <-}\StringTok{ }\NormalTok{occ_gof}\OperatorTok{$}\NormalTok{c.hat.est}
\NormalTok{  scientific_name <-}\StringTok{ }\KeywordTok{unique}\NormalTok{(data}\OperatorTok{$}\NormalTok{scientific_name)}

\NormalTok{  a <-}\StringTok{ }\KeywordTok{data.frame}\NormalTok{(scientific_name, p.value, c.hat)}

\NormalTok{  goodness_of_fit <-}\StringTok{ }\KeywordTok{rbind}\NormalTok{(a, goodness_of_fit)}

  \KeywordTok{setTxtProgressBar}\NormalTok{(pb, i)}
\NormalTok{\}}
\KeywordTok{close}\NormalTok{(pb)}

\KeywordTok{write.csv}\NormalTok{(goodness_of_fit, }\StringTok{"data}\CharTok{\textbackslash{}r}\StringTok{esults}\CharTok{\textbackslash{}05}\StringTok{_goodness-of-fit-2.5km.csv"}\NormalTok{)}
\end{Highlighting}
\end{Shaded}

\hypertarget{visualizing-occupancy-predictor-effects}{%
\section{Visualizing Occupancy Predictor Effects}\label{visualizing-occupancy-predictor-effects}}

In this section, we will visualize the cumulative AIC weights and the magnitude and direction of species-specific probability of occupancy.

To get cumulative AIC weights, we first obtained a measure of relative importance of climatic and landscape predictors by calculating cumulative variable importance scores. These scores were calculated by obtaining the sum of model weights (AIC weights) across all models (including the top models) for each predictor across all species. We then calculated the mean cumulative variable importance score and a standard deviation for each predictor (Burnham and Anderson \protect\hyperlink{ref-burnham2002a}{2002}).

\hypertarget{prepare-libraries-3}{%
\subsection{Prepare libraries}\label{prepare-libraries-3}}

\begin{Shaded}
\begin{Highlighting}[numbers=left,,]
\CommentTok{# to load data}
\KeywordTok{library}\NormalTok{(readxl)}

\CommentTok{# to handle data}
\KeywordTok{library}\NormalTok{(dplyr)}
\KeywordTok{library}\NormalTok{(readr)}
\KeywordTok{library}\NormalTok{(forcats)}
\KeywordTok{library}\NormalTok{(tidyr)}
\KeywordTok{library}\NormalTok{(purrr)}
\KeywordTok{library}\NormalTok{(stringr)}
\KeywordTok{library}\NormalTok{(data.table)}

\CommentTok{# to wrangle models}
\KeywordTok{source}\NormalTok{(}\StringTok{"code/fun_model_estimate_collection.r"}\NormalTok{)}
\KeywordTok{source}\NormalTok{(}\StringTok{"code/fun_make_resp_data.r"}\NormalTok{)}

\CommentTok{# nice tables}
\KeywordTok{library}\NormalTok{(knitr)}
\KeywordTok{library}\NormalTok{(kableExtra)}

\CommentTok{# plotting}
\KeywordTok{library}\NormalTok{(ggplot2)}
\KeywordTok{library}\NormalTok{(patchwork)}
\KeywordTok{source}\NormalTok{(}\StringTok{"code/fun_plot_interaction.r"}\NormalTok{)}
\end{Highlighting}
\end{Shaded}

\hypertarget{load-species-list}{%
\subsection{Load species list}\label{load-species-list}}

\begin{Shaded}
\begin{Highlighting}[numbers=left,,]
\CommentTok{# list of species}
\NormalTok{species <-}\StringTok{ }\KeywordTok{read_csv}\NormalTok{(}\StringTok{"data/species_list.csv"}\NormalTok{)}
\NormalTok{list_of_species <-}\StringTok{ }\KeywordTok{as.character}\NormalTok{(species}\OperatorTok{$}\NormalTok{scientific_name)}
\end{Highlighting}
\end{Shaded}

\hypertarget{show-aic-weight-importance}{%
\subsection{Show AIC weight importance}\label{show-aic-weight-importance}}

\hypertarget{read-in-aic-weight-data}{%
\subsubsection{Read in AIC weight data}\label{read-in-aic-weight-data}}

\begin{Shaded}
\begin{Highlighting}[numbers=left,,]
\CommentTok{# which files to read}
\NormalTok{file_names <-}\StringTok{ }\KeywordTok{c}\NormalTok{(}\StringTok{"data/results/lc-clim-imp.xlsx"}\NormalTok{)}

\CommentTok{# read in sheets by species}
\NormalTok{model_imp <-}\StringTok{ }\KeywordTok{map}\NormalTok{(file_names, }\ControlFlowTok{function}\NormalTok{(f) \{}
\NormalTok{  md_list <-}\StringTok{ }\KeywordTok{map}\NormalTok{(list_of_species, }\ControlFlowTok{function}\NormalTok{(sn) \{}

    \CommentTok{# some sheets are not found}

    \KeywordTok{tryCatch}\NormalTok{(}
\NormalTok{      \{}
\NormalTok{        readxl}\OperatorTok{::}\KeywordTok{read_excel}\NormalTok{(f, }\DataTypeTok{sheet =}\NormalTok{ sn) }\OperatorTok
\StringTok{          `}\DataTypeTok{colnames<-}\StringTok{`}\NormalTok{(}\KeywordTok{c}\NormalTok{(}\StringTok{"predictor"}\NormalTok{, }\StringTok{"AIC_weight"}\NormalTok{)) }\OperatorTok
\StringTok{          }\KeywordTok{filter}\NormalTok{(}\KeywordTok{str_detect}\NormalTok{(predictor, }\StringTok{"psi"}\NormalTok{)) }\OperatorTok
\StringTok{          }\KeywordTok{mutate}\NormalTok{(}
            \DataTypeTok{predictor =}\NormalTok{ stringr}\OperatorTok{::}\KeywordTok{str_extract}\NormalTok{(predictor,}
              \DataTypeTok{pattern =}\NormalTok{ stringr}\OperatorTok{::}\KeywordTok{regex}\NormalTok{(}\StringTok{"}\CharTok{\textbackslash{}\textbackslash{}}\StringTok{((.*?)}\CharTok{\textbackslash{}\textbackslash{}}\StringTok{)"}\NormalTok{)}
\NormalTok{            ),}
            \DataTypeTok{predictor =}\NormalTok{ stringr}\OperatorTok{::}\KeywordTok{str_replace_all}\NormalTok{(predictor, }\StringTok{"[//(//)]"}\NormalTok{, }\StringTok{""}\NormalTok{),}
            \DataTypeTok{predictor =}\NormalTok{ stringr}\OperatorTok{::}\KeywordTok{str_remove}\NormalTok{(predictor, }\StringTok{"}\CharTok{\textbackslash{}\textbackslash{}}\StringTok{.y"}\NormalTok{)}
\NormalTok{          )}
\NormalTok{      \},}
      \DataTypeTok{error =} \ControlFlowTok{function}\NormalTok{(e) \{}
        \KeywordTok{message}\NormalTok{(}\KeywordTok{as.character}\NormalTok{(e))}
\NormalTok{      \}}
\NormalTok{    )}
\NormalTok{  \})}
  \KeywordTok{names}\NormalTok{(md_list) <-}\StringTok{ }\NormalTok{list_of_species}

  \KeywordTok{return}\NormalTok{(md_list)}
\NormalTok{\})}
\end{Highlighting}
\end{Shaded}

\hypertarget{prepare-cumulative-aic-weight-data}{%
\subsubsection{Prepare cumulative AIC weight data}\label{prepare-cumulative-aic-weight-data}}

\begin{Shaded}
\begin{Highlighting}[numbers=left,,]
\CommentTok{# assign scale - minimum spatial scale at which the analysis was carried out to account for observer effort}
\KeywordTok{names}\NormalTok{(model_imp) <-}\StringTok{ }\KeywordTok{c}\NormalTok{(}\StringTok{"2.5km"}\NormalTok{)}
\NormalTok{model_imp <-}\StringTok{ }\KeywordTok{imap}\NormalTok{(model_imp, }\ControlFlowTok{function}\NormalTok{(.x, .y) \{}
\NormalTok{  .x <-}\StringTok{ }\KeywordTok{bind_rows}\NormalTok{(.x)}
\NormalTok{  .x}\OperatorTok{$}\NormalTok{scale <-}\StringTok{ }\NormalTok{.y}
  \KeywordTok{return}\NormalTok{(.x)}
\NormalTok{\})}

\CommentTok{# bind rows}
\NormalTok{model_imp <-}\StringTok{ }\KeywordTok{map}\NormalTok{(model_imp, bind_rows) }\OperatorTok
\StringTok{  }\KeywordTok{bind_rows}\NormalTok{()}

\CommentTok{# convert to numeric}
\NormalTok{model_imp}\OperatorTok{$}\NormalTok{AIC_weight <-}\StringTok{ }\KeywordTok{as.numeric}\NormalTok{(model_imp}\OperatorTok{$}\NormalTok{AIC_weight)}
\NormalTok{model_imp}\OperatorTok{$}\NormalTok{scale <-}\StringTok{ }\KeywordTok{as.factor}\NormalTok{(model_imp}\OperatorTok{$}\NormalTok{scale)}
\KeywordTok{levels}\NormalTok{(model_imp}\OperatorTok{$}\NormalTok{scale) <-}\StringTok{ }\KeywordTok{c}\NormalTok{(}\StringTok{"2.5km"}\NormalTok{)}

\CommentTok{# Let's get a summary of cumulative variable importance}
\NormalTok{model_imp <-}\StringTok{ }\KeywordTok{group_by}\NormalTok{(model_imp, predictor) }\OperatorTok
\StringTok{  }\KeywordTok{summarise}\NormalTok{(}
    \DataTypeTok{mean_AIC =} \KeywordTok{mean}\NormalTok{(AIC_weight),}
    \DataTypeTok{sd_AIC =} \KeywordTok{sd}\NormalTok{(AIC_weight),}
    \DataTypeTok{min_AIC =} \KeywordTok{min}\NormalTok{(AIC_weight),}
    \DataTypeTok{max_AIC =} \KeywordTok{max}\NormalTok{(AIC_weight),}
    \DataTypeTok{med_AIC =} \KeywordTok{median}\NormalTok{(AIC_weight)}
\NormalTok{  )}

\CommentTok{# write to file}
\KeywordTok{write_csv}\NormalTok{(model_imp,}
  \DataTypeTok{file =} \StringTok{"data/results/cumulative_AIC_weights.csv"}
\NormalTok{)}
\end{Highlighting}
\end{Shaded}

Read data back in.

\begin{Shaded}
\begin{Highlighting}[numbers=left,,]
\CommentTok{# read data and make factor}
\NormalTok{model_imp <-}\StringTok{ }\KeywordTok{read_csv}\NormalTok{(}\StringTok{"data/results/cumulative_AIC_weights.csv"}\NormalTok{)}
\NormalTok{model_imp}\OperatorTok{$}\NormalTok{predictor <-}\StringTok{ }\KeywordTok{as_factor}\NormalTok{(model_imp}\OperatorTok{$}\NormalTok{predictor)}
\end{Highlighting}
\end{Shaded}

\begin{Shaded}
\begin{Highlighting}[numbers=left,,]
\CommentTok{# make nice names}
\NormalTok{predictor_name <-}\StringTok{ }\KeywordTok{tibble}\NormalTok{(}
  \DataTypeTok{predictor =} \KeywordTok{levels}\NormalTok{(model_imp}\OperatorTok{$}\NormalTok{predictor),}
  \DataTypeTok{pred_name =} \KeywordTok{c}\NormalTok{(}
    \StringTok{"Annual Mean Temperature (°C)"}\NormalTok{,}
    \StringTok{"Annual Precipitation (mm)"}\NormalTok{,}
    \StringTok{"% Agriculture"}\NormalTok{, }\StringTok{"% Forests"}\NormalTok{,}
    \StringTok{"% Plantations"}\NormalTok{, }\StringTok{"% Settlements"}\NormalTok{,}
    \StringTok{"% Tea"}\NormalTok{, }\StringTok{"% Water Bodies"}
\NormalTok{  )}
\NormalTok{)}

\CommentTok{# rename predictor}
\NormalTok{model_imp <-}\StringTok{ }\KeywordTok{left_join}\NormalTok{(model_imp, predictor_name)}
\end{Highlighting}
\end{Shaded}

Prepare figure for cumulative AIC weight. Figure code is hidden in versions rendered as HTML and PDF.

\begin{Shaded}
\begin{Highlighting}[numbers=left,,]
\NormalTok{fig_aic <-}
\StringTok{  }\KeywordTok{ggplot}\NormalTok{(model_imp) }\OperatorTok{+}
\StringTok{  }\KeywordTok{geom_pointrange}\NormalTok{(}\KeywordTok{aes}\NormalTok{(}
    \DataTypeTok{x =} \KeywordTok{reorder}\NormalTok{(predictor, mean_AIC),}
    \DataTypeTok{y =}\NormalTok{ mean_AIC,}
    \DataTypeTok{ymin =}\NormalTok{ mean_AIC }\OperatorTok{-}\StringTok{ }\NormalTok{sd_AIC,}
    \DataTypeTok{ymax =}\NormalTok{ mean_AIC }\OperatorTok{+}\StringTok{ }\NormalTok{sd_AIC}
\NormalTok{  )) }\OperatorTok{+}
\StringTok{  }\KeywordTok{geom_text}\NormalTok{(}\KeywordTok{aes}\NormalTok{(}
    \DataTypeTok{x =}\NormalTok{ predictor,}
    \DataTypeTok{y =} \FloatTok{0.2}\NormalTok{,}
    \DataTypeTok{label =}\NormalTok{ pred_name}
\NormalTok{  ),}
  \DataTypeTok{angle =} \DecValTok{0}\NormalTok{,}
  \DataTypeTok{hjust =} \StringTok{"inward"}\NormalTok{,}
  \DataTypeTok{vjust =} \DecValTok{2}
\NormalTok{  ) }\OperatorTok{+}
\StringTok{  }\CommentTok{# scale_y_continuous(breaks = seq(45, 75, 10))+}
\StringTok{  }\KeywordTok{scale_x_discrete}\NormalTok{(}\DataTypeTok{labels =} \OtherTok{NULL}\NormalTok{) }\OperatorTok{+}
\StringTok{  }\CommentTok{# scale_color_brewer(palette = "RdBu", values = c(0.5, 1))+}
\StringTok{  }\KeywordTok{coord_flip}\NormalTok{(}
    \CommentTok{# ylim = c(45, 75)}
\NormalTok{  ) }\OperatorTok{+}
\StringTok{  }\KeywordTok{theme_test}\NormalTok{() }\OperatorTok{+}
\StringTok{  }\KeywordTok{theme}\NormalTok{(}\DataTypeTok{legend.position =} \StringTok{"none"}\NormalTok{) }\OperatorTok{+}
\StringTok{  }\KeywordTok{labs}\NormalTok{(}
    \DataTypeTok{x =} \StringTok{"Predictor"}\NormalTok{,}
    \DataTypeTok{y =} \StringTok{"Cumulative AIC weight"}
\NormalTok{  )}

\KeywordTok{ggsave}\NormalTok{(fig_aic,}
  \DataTypeTok{filename =} \StringTok{"figs/fig_aic_weight.png"}\NormalTok{,}
  \DataTypeTok{device =} \KeywordTok{png}\NormalTok{(),}
  \DataTypeTok{dpi =} \DecValTok{300}\NormalTok{,}
  \DataTypeTok{width =} \DecValTok{79}\NormalTok{, }\DataTypeTok{height =} \DecValTok{120}\NormalTok{, }\DataTypeTok{units =} \StringTok{"mm"}
\NormalTok{)}
\end{Highlighting}
\end{Shaded}

\hypertarget{prepare-model-coefficient-data}{%
\subsection{Prepare model coefficient data}\label{prepare-model-coefficient-data}}

For each species, we examined those models which had ΔAICc \textless{} 2, as these top models were considered to explain a large proportion of the association between the species-specific probability of occupancy and environmental drivers (Burnham, Anderson, and Huyvaert \protect\hyperlink{ref-burnham2011}{2011}; Elsen et al. \protect\hyperlink{ref-elsen2017}{2017}). Using these restricted model sets for each species; we created a model-averaged coefficient estimate for each predictor and assessed its direction and significance (Bartoń \protect\hyperlink{ref-MuMIn}{2020}). We considered a predictor to be significantly associated with occupancy if the range of the 95\% confidence interval around the model-averaged coefficient did not contain zero.

\begin{Shaded}
\begin{Highlighting}[numbers=left,,]
\NormalTok{file_read <-}\StringTok{ }\KeywordTok{c}\NormalTok{(}\StringTok{"data/results/lc-clim-modelEst.xlsx"}\NormalTok{)}

\CommentTok{# read data as list column}
\NormalTok{model_est <-}\StringTok{ }\KeywordTok{map}\NormalTok{(file_read, }\ControlFlowTok{function}\NormalTok{(fr) \{}
\NormalTok{  md_list <-}\StringTok{ }\KeywordTok{map}\NormalTok{(list_of_species, }\ControlFlowTok{function}\NormalTok{(sn) \{}
\NormalTok{    readxl}\OperatorTok{::}\KeywordTok{read_excel}\NormalTok{(fr, }\DataTypeTok{sheet =}\NormalTok{ sn)}
\NormalTok{  \})}
  \KeywordTok{names}\NormalTok{(md_list) <-}\StringTok{ }\NormalTok{list_of_species}
  \KeywordTok{return}\NormalTok{(md_list)}
\NormalTok{\})}

\CommentTok{# prepare model data}
\NormalTok{scales <-}\StringTok{ }\KeywordTok{c}\NormalTok{(}\StringTok{"2.5km"}\NormalTok{)}
\NormalTok{model_data <-}\StringTok{ }\KeywordTok{tibble}\NormalTok{(}
  \DataTypeTok{scale =}\NormalTok{ scales,}
  \DataTypeTok{scientific_name =}\NormalTok{ list_of_species}
\NormalTok{) }\OperatorTok
\StringTok{  }\KeywordTok{arrange}\NormalTok{(}\KeywordTok{desc}\NormalTok{(scale))}

\CommentTok{# rename model data components and separate predictors}
\NormalTok{names <-}\StringTok{ }\KeywordTok{c}\NormalTok{(}
  \StringTok{"predictor"}\NormalTok{, }\StringTok{"coefficient"}\NormalTok{, }\StringTok{"se"}\NormalTok{, }\StringTok{"ci_lower"}\NormalTok{,}
  \StringTok{"ci_higher"}\NormalTok{, }\StringTok{"z_value"}\NormalTok{, }\StringTok{"p_value"}
\NormalTok{)}

\CommentTok{# get data for plotting:}
\NormalTok{model_est <-}\StringTok{ }\KeywordTok{map}\NormalTok{(model_est, }\ControlFlowTok{function}\NormalTok{(l) \{}
  \KeywordTok{map}\NormalTok{(l, }\ControlFlowTok{function}\NormalTok{(df) \{}
    \KeywordTok{colnames}\NormalTok{(df) <-}\StringTok{ }\NormalTok{names}
\NormalTok{    df <-}\StringTok{ }\KeywordTok{separate_interaction_terms}\NormalTok{(df)}
\NormalTok{    df <-}\StringTok{ }\KeywordTok{make_response_data}\NormalTok{(df)}
    \KeywordTok{return}\NormalTok{(df)}
\NormalTok{  \})}
\NormalTok{\})}

\CommentTok{# add names and scales}
\NormalTok{model_est <-}\StringTok{ }\KeywordTok{map}\NormalTok{(model_est, }\ControlFlowTok{function}\NormalTok{(l) \{}
  \KeywordTok{imap}\NormalTok{(l, }\ControlFlowTok{function}\NormalTok{(.x, .y) \{}
    \KeywordTok{mutate}\NormalTok{(.x, }\DataTypeTok{scientific_name =}\NormalTok{ .y)}
\NormalTok{  \})}
\NormalTok{\})}

\CommentTok{# add names to model estimates}
\KeywordTok{names}\NormalTok{(model_est) <-}\StringTok{ }\NormalTok{scales}
\NormalTok{model_est <-}\StringTok{ }\KeywordTok{imap}\NormalTok{(model_est, }\ControlFlowTok{function}\NormalTok{(.x, .y) \{}
  \KeywordTok{bind_rows}\NormalTok{(.x) }\OperatorTok
\StringTok{    }\KeywordTok{mutate}\NormalTok{(}\DataTypeTok{scale =}\NormalTok{ .y)}
\NormalTok{\})}

\CommentTok{# remove modulators}
\NormalTok{model_est <-}\StringTok{ }\KeywordTok{bind_rows}\NormalTok{(model_est) }\OperatorTok
\StringTok{  }\KeywordTok{select}\NormalTok{(}\OperatorTok{-}\KeywordTok{matches}\NormalTok{(}\StringTok{"modulator"}\NormalTok{))}

\CommentTok{# join data to species name}
\NormalTok{model_data <-}\StringTok{ }\NormalTok{model_data }\OperatorTok
\StringTok{  }\KeywordTok{left_join}\NormalTok{(model_est)}

\CommentTok{# Keep only those predictors whose p-values are significant:}
\NormalTok{model_data <-}\StringTok{ }\NormalTok{model_data }\OperatorTok
\StringTok{  }\KeywordTok{filter}\NormalTok{(p_value }\OperatorTok{<}\StringTok{ }\FloatTok{0.05}\NormalTok{)}
\end{Highlighting}
\end{Shaded}

Export predictor effects.

\begin{Shaded}
\begin{Highlighting}[numbers=left,,]
\CommentTok{# get predictor effect data}
\NormalTok{data_predictor_effect <-}\StringTok{ }\KeywordTok{distinct}\NormalTok{(}
\NormalTok{  model_data,}
\NormalTok{  scientific_name,}
\NormalTok{  se,}
\NormalTok{  predictor, coefficient}
\NormalTok{)}

\CommentTok{# write to file}
\KeywordTok{write_csv}\NormalTok{(data_predictor_effect,}
  \DataTypeTok{path =} \StringTok{"data/results/data_predictor_effect.csv"}
\NormalTok{)}
\end{Highlighting}
\end{Shaded}

Export model data.

\begin{Shaded}
\begin{Highlighting}[numbers=left,,]
\NormalTok{model_data_to_file <-}\StringTok{ }\NormalTok{model_data }\OperatorTok
\StringTok{  }\KeywordTok{select}\NormalTok{(}
\NormalTok{    predictor, data,}
\NormalTok{    scientific_name, scale}
\NormalTok{  ) }\OperatorTok
\StringTok{  }\KeywordTok{unnest}\NormalTok{(}\DataTypeTok{cols =} \StringTok{"data"}\NormalTok{)}

\CommentTok{# remove .y}
\NormalTok{model_data_to_file <-}\StringTok{ }\NormalTok{model_data_to_file }\OperatorTok
\StringTok{  }\KeywordTok{mutate}\NormalTok{(}\DataTypeTok{predictor =} \KeywordTok{str_remove}\NormalTok{(predictor, }\StringTok{"}\CharTok{\textbackslash{}\textbackslash{}}\StringTok{.y"}\NormalTok{))}

\KeywordTok{write_csv}\NormalTok{(}
\NormalTok{  model_data_to_file,}
  \StringTok{"data/results/data_occupancy_predictors.csv"}
\NormalTok{)}
\end{Highlighting}
\end{Shaded}

Read in data after clearing R session.

\begin{Shaded}
\begin{Highlighting}[numbers=left,,]
\CommentTok{# read from file}
\NormalTok{model_data <-}\StringTok{ }\KeywordTok{read_csv}\NormalTok{(}\StringTok{"data/results/data_predictor_effect.csv"}\NormalTok{)}
\end{Highlighting}
\end{Shaded}

Fix predictor name.

\begin{Shaded}
\begin{Highlighting}[numbers=left,,]
\CommentTok{# remove .y from predictors}
\NormalTok{model_data <-}\StringTok{ }\NormalTok{model_data }\OperatorTok
\StringTok{  }\KeywordTok{mutate_at}\NormalTok{(}\DataTypeTok{.vars =} \KeywordTok{c}\NormalTok{(}\StringTok{"predictor"}\NormalTok{), }\DataTypeTok{.funs =} \ControlFlowTok{function}\NormalTok{(x) \{}
\NormalTok{    stringr}\OperatorTok{::}\KeywordTok{str_remove}\NormalTok{(x, }\StringTok{".y"}\NormalTok{)}
\NormalTok{  \})}
\end{Highlighting}
\end{Shaded}

\hypertarget{get-predictor-effects}{%
\subsection{Get predictor effects}\label{get-predictor-effects}}

\begin{Shaded}
\begin{Highlighting}[numbers=left,,]
\CommentTok{# is the coeff positive? how many positive per scale per predictor per axis of split?}
\NormalTok{data_predictor <-}\StringTok{ }\KeywordTok{mutate}\NormalTok{(model_data,}
  \DataTypeTok{direction =}\NormalTok{ coefficient }\OperatorTok{>}\StringTok{ }\DecValTok{0}
\NormalTok{) }\OperatorTok
\StringTok{  }\KeywordTok{count}\NormalTok{(}
\NormalTok{    predictor,}
\NormalTok{    direction}
\NormalTok{  ) }\OperatorTok
\StringTok{  }\KeywordTok{mutate}\NormalTok{(}\DataTypeTok{mag =}\NormalTok{ n }\OperatorTok{*}\StringTok{ }\NormalTok{(}\KeywordTok{if_else}\NormalTok{(direction, }\DecValTok{1}\NormalTok{, }\DecValTok{-1}\NormalTok{)))}

\CommentTok{# wrangle data to get nice bars}
\NormalTok{data_predictor <-}\StringTok{ }\NormalTok{data_predictor }\OperatorTok
\StringTok{  }\KeywordTok{select}\NormalTok{(}\OperatorTok{-}\NormalTok{n) }\OperatorTok
\StringTok{  }\KeywordTok{drop_na}\NormalTok{(direction) }\OperatorTok
\StringTok{  }\KeywordTok{mutate}\NormalTok{(}\DataTypeTok{direction =} \KeywordTok{ifelse}\NormalTok{(direction, }\StringTok{"positive"}\NormalTok{, }\StringTok{"negative"}\NormalTok{)) }\OperatorTok
\StringTok{  }\KeywordTok{pivot_wider}\NormalTok{(}\DataTypeTok{values_from =} \StringTok{"mag"}\NormalTok{, }\DataTypeTok{names_from =} \StringTok{"direction"}\NormalTok{) }\OperatorTok
\StringTok{  }\KeywordTok{mutate_at}\NormalTok{(}
    \KeywordTok{vars}\NormalTok{(positive, negative),}
    \OperatorTok{~}\StringTok{ }\KeywordTok{if_else}\NormalTok{(}\KeywordTok{is.na}\NormalTok{(.), }\DecValTok{0}\NormalTok{, .)}
\NormalTok{  )}

\NormalTok{data_predictor_long <-}\StringTok{ }\NormalTok{data_predictor }\OperatorTok
\StringTok{  }\KeywordTok{pivot_longer}\NormalTok{(}
    \DataTypeTok{cols =} \KeywordTok{c}\NormalTok{(}\StringTok{"negative"}\NormalTok{, }\StringTok{"positive"}\NormalTok{),}
    \DataTypeTok{names_to =} \StringTok{"effect"}\NormalTok{,}
    \DataTypeTok{values_to =} \StringTok{"magnitude"}
\NormalTok{  )}

\CommentTok{# write}
\KeywordTok{write_csv}\NormalTok{(data_predictor_long,}
  \DataTypeTok{path =} \StringTok{"data/results/data_predictor_direction_nSpecies.csv"}
\NormalTok{)}
\end{Highlighting}
\end{Shaded}

Prepare data to determine the direction (positive or negative) of the effect of each predictor. How many species are affected in either direction?

\begin{Shaded}
\begin{Highlighting}[numbers=left,,]
\CommentTok{# join with predictor names and relative AIC}
\NormalTok{data_predictor_long <-}\StringTok{ }\KeywordTok{left_join}\NormalTok{(data_predictor_long, model_imp)}
\end{Highlighting}
\end{Shaded}

Prepare figure of the number of species affected in each direction. Figure code is hidden in versions rendered as HTML and PDF.

\hypertarget{main-text-figure-4}{%
\subsection{Main Text Figure 4}\label{main-text-figure-4}}

Figure code is hidden in versions rendered as HTML and PDF.

\begin{figure}
\centering
\includegraphics{figs/fig_04_aic_weight_effect.png}
\caption{(a) Cumulative AIC weights suggest that climatic predictors have higher relative importance when compared to landscape predictors. (b) The direction of association between species-specific probability of occupancy and climatic and landscape is shown here. While climatic predictors were both positively and negatively associated with the probability of occupancy for a number of species, human-associated land cover types were largely negatively associated with species-specific probability of occupancy.}
\end{figure}

\clearpage

\hypertarget{references}{%
\section{References}\label{references}}

\hypertarget{refs}{}
\leavevmode\hypertarget{ref-MuMIn}{}%
Bartoń, Kamil. 2020. \emph{MuMIn: Multi-Model Inference}. Manual.

\leavevmode\hypertarget{ref-burnham2002a}{}%
Burnham, Kenneth P., and David R. Anderson. 2002. \emph{Model Selection and Multimodel Inference: A Practical Information-Theoretic Approach}. Second. New York: Springer-Verlag. \url{https://doi.org/10.1007/b97636}.

\leavevmode\hypertarget{ref-burnham2011}{}%
Burnham, Kenneth P., David R. Anderson, and Kathryn P. Huyvaert. 2011. ``AIC Model Selection and Multimodel Inference in Behavioral Ecology: Some Background, Observations, and Comparisons.'' \emph{Behavioral Ecology and Sociobiology} 65 (1): 23--35. \url{https://doi.org/10.1007/s00265-010-1029-6}.

\leavevmode\hypertarget{ref-elsen2017}{}%
Elsen, Paul R., Morgan W. Tingley, Ramnarayan Kalyanaraman, Krishnamurthy Ramesh, and David S. Wilcove. 2017. ``The Role of Competition, Ecotones, and Temperature in the Elevational Distribution of Himalayan Birds.'' \emph{Ecology} 98 (2): 337--48. \url{https://doi.org/10.1002/ecy.1669}.

\leavevmode\hypertarget{ref-fiske2011}{}%
Fiske, Ian, and Richard Chandler. 2011. ``\emph{Unmarked} : An \emph{R} Package for Fitting Hierarchical Models of Wildlife Occurrence and Abundance.'' \emph{Journal of Statistical Software} 43 (10). \url{https://doi.org/10.18637/jss.v043.i10}.

\leavevmode\hypertarget{ref-praveenj.2017}{}%
J., Praveen. 2017. ``On the Geo-Precision of Data for Modelling Home Range of a Species Commentary on Ramesh et Al. (2017).'' \emph{Biological Conservation} 213 (September): 245--46. \url{https://doi.org/10.1016/j.biocon.2017.07.017}.

\leavevmode\hypertarget{ref-johnston2019a}{}%
Johnston, A, Wm Hochachka, Me Strimas-Mackey, V Ruiz Gutierrez, Oj Robinson, Et Miller, T Auer, St Kelling, and D Fink. 2019. ``Analytical Guidelines to Increase the Value of Citizen Science Data: Using eBird Data to Estimate Species Occurrence.'' Preprint. Ecology. \url{https://doi.org/10.1101/574392}.

\leavevmode\hypertarget{ref-johnston2018}{}%
Johnston, Alison, Daniel Fink, Wesley M. Hochachka, and Steve Kelling. 2018. ``Estimates of Observer Expertise Improve Species Distributions from Citizen Science Data.'' Edited by Nick Isaac. \emph{Methods in Ecology and Evolution} 9 (1): 88--97. \url{https://doi.org/10.1111/2041-210X.12838}.

\leavevmode\hypertarget{ref-kelling2015a}{}%
Kelling, Steve, Alison Johnston, Wesley M. Hochachka, Marshall Iliff, Daniel Fink, Jeff Gerbracht, Carl Lagoze, et al. 2015. ``Can Observation Skills of Citizen Scientists Be Estimated Using Species Accumulation Curves?'' Edited by Stefano Goffredo. \emph{PLOS ONE} 10 (10): e0139600. \url{https://doi.org/10.1371/journal.pone.0139600}.

\leavevmode\hypertarget{ref-mackenzie2002}{}%
MacKenzie, Darryl I., James D. Nichols, Gideon B. Lachman, Sam Droege, J. Andrew Royle, and Catherine A. Langtimm. 2002. ``Estimating Site Occupancy Rates When Detection Probabilities Are Less Than One.'' \emph{Ecology} 83 (8): 2248--55. \href{https://doi.org/10.1890/0012-9658(2002)083\%5B2248:ESORWD\%5D2.0.CO;2}{https://doi.org/10.1890/0012-9658(2002)083{[}2248:ESORWD{]}2.0.CO;2}.

\leavevmode\hypertarget{ref-OpenStreetMap}{}%
OpenStreetMap contributors. 2019. ``Planet Dump Retrieved from https://planet.osm.org.''

\leavevmode\hypertarget{ref-sullivan2014}{}%
Sullivan, Brian L., Jocelyn L. Aycrigg, Jessie H. Barry, Rick E. Bonney, Nicholas Bruns, Caren B. Cooper, Theo Damoulas, et al. 2014. ``The eBird Enterprise: An Integrated Approach to Development and Application of Citizen Science.'' \emph{Biological Conservation} 169 (January): 31--40. \url{https://doi.org/10.1016/j.biocon.2013.11.003}.

\leavevmode\hypertarget{ref-vanstrien2013}{}%
van Strien, Arco J., Chris A.M. van Swaay, and Tim Termaat. 2013. ``Opportunistic Citizen Science Data of Animal Species Produce Reliable Estimates of Distribution Trends If Analysed with Occupancy Models.'' Edited by Vincent Devictor. \emph{Journal of Applied Ecology} 50 (6): 1450--8. \url{https://doi.org/10.1111/1365-2664.12158}.

\end{document}
